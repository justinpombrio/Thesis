% Related work:
% Object representation of scope during translation (coined "scope graph"):
%   http://www.lirmm.fr/~ducour/Doc-objets/ECOOP/papers/0276/02760071.pdf
% MetaOcaml & related work by Walid Taha
% Interesting sugary languages: Dylan, Julia, Nim
% From Joe, "See S8. And also the whole paper.":
%   https://www.microsoft.com/en-us/research/wp-content/uploads/2016/11/compiling-without-continuations.pdf
% PHOAS?
%   Chipala ICFP'08 https://archive.alvb.in/msc/10_infomepcs1/literature/PHOAS_Chlipala.pdf
% Extensible Grammars for Language Specialization (must CITE)
%   Cardelli, Matthes, Abadi - DigEquipRC

\RequirePackage{silence} % :-\
    \WarningFilter{scrbook}{Usage of package `titlesec'}
    \WarningFilter{titlesec}{Non standard sectioning command detected}
\documentclass[
  11pt,
  paper=letter,
  footinclude=true,
  headinclude=true,
  american
]{scrbook}

\usepackage[T1]{fontenc}
\usepackage[
  linedheaders=true,
  parts=true
]{classicthesis/classicthesis} % ,manychapters

\usepackage{amsthm}
\usepackage{amsmath}
\usepackage{amssymb}
\usepackage{xargs}
\usepackage{xspace}
\usepackage{semantic}
\usepackage{cleveref}
\usepackage{color}
\usepackage{alltt}
\usepackage{listings}
\usepackage{multicol}
\usepackage{bussproofs}



% Theorems
\newtheorem{definition}{Definition}
\newtheorem{theorem}{Theorem}
\newtheorem{lemma}{Lemma}
\newtheorem{assumption}{Assumption}

\lstset{basicstyle=\ttfamily,breaklines=true}
\lstset{framextopmargin=50pt,frame=bottomline}

% Formatting
\newcommand{\desc}[1]{\noindent\textit{#1:} }
\newenvironment{jtable}
{\begin{center}\begin{tabular}{l l l @{\quad}l}}
{\end{tabular}\end{center}}
               

% Common math notation
\newcommand{\ddd}{\;\dots\;}
\newcommand{\dd}{\,...\,}

% Common math notation
\newcommand{\Exists}[1]{\exists{#1}.\,\,}
\newcommand{\ExistsUnique}[1]{\exists!{#1}.\,\,}
\newcommand{\NotExists}[1]{\not\exists{#1}.\,\,}
\newcommand{\Forall}[1]{\forall{#1}.\,\,}
\newcommand{\NotForall}[1]{\not\forall{#1}.\,\,}
\newcommand{\SetSuchThat}[2]{\{#1 \;|\; #2\}}
\newcommand{\domain}[1]{\textit{domain}(#1)}
\newcommand{\DisjUnion}{\,\dot\cup\,}
\newcommand{\To}{\Rightarrow}

% Common code notation
\newcommand{\code}[1]{\texttt{#1}}
\newenvironment{codes}
  {\begin{alltt}\leftskip=1.5em} % \small
  {\end{alltt}}

% Math names
\newcommand{\op}[2]{\textit{#1}(#2)}
\newcommand{\opName}[1]{\textit{#1}}
\newcommand{\name}[1]{\textit{#1}}
\newcommand{\constName}[1]{\texttt{#1}}

% Expressions & Grammars
\newcommand{\lit}[1]{\textbf{#1}}
\newcommand{\expr}[2]{(#1\,#2)}
\newcommand{\var}[1]{\textrm{\textsc{#1}}}

% Grammars
\newcommand{\exprs}[3]{(#1\,#2\,#3^{*})}
\newcommand{\production}[2]{#1 \leftarrow #2}
\newcommand{\saysG}[3]{#1 \vdash #2\,:\,#3}

\definecolor{goodRed}{rgb} {0.70, 0.37, 0.41} % LAB (50, 35, 10)
\definecolor{goodBlue}{rgb}{0.04, 0.50, 0.70} % LAB (50, -10, -35)
\hypersetup{
  colorlinks=false,
  %citecolor=goodBlue,
  %linkcolor=goodRed
  citebordercolor=goodBlue,
  linkbordercolor=goodRed
  }


\begin{document}

\author{Justin Pombrio}
\title{Resugaring: Restoring the Abstractions that Syntactic Sugar is Supposed to Provide}
\maketitle

\part{Syntactic Sugar}
\chapter{Syntactic Sugar}

\section{What is it?}

% http://www.cs.cmu.edu/~crary/819-f09/Landin64.pdf
The term \emph{syntactic sugar} was introduced by Peter Landin in
1964[CITE]. It refers to surface syntactic forms that are provided for
convenience, but could instead be written using the syntax of the rest
of the language. This captures the spirit and purpose of syntactic
sugar, but is not discriminating enough to be a useful definition.
I will define syntactic sugar as follows:
%% [TODO: definition must (i) rule out functions, (ii) rule out
%%   metaprogramming, (iii) include Racket macros, which are non-local
%%   and non-phase-specific.]
\begin{quote}
  A syntactic construct in an implementation of a programming language
  is \emph{syntactic sugar} if it is translated at compile time into
  the syntax of the rest of the language.
\end{quote}
%% \begin{quote}
%%   'A construct in a language is called ``syntactic sugar'' if it can be
%%   removed from the language without any effect on what the language
%%   can do: functionality and expressive power will remain the same.' (Wikipedia)
%% \end{quote}

The name suggests that syntactic sugar is inessential: it ``sweetens''
the language to make it more palatable, but does not otherwise change
its substance. The name also naturally leads to related terminology:
\emph{desugaring} is the removal of syntactic sugar by expanding it;
and \emph{resugaring} is a term I am introducing for the restoration
of information that was lost during desugaring.

\section{What is it Good for?}

Syntactic sugar is used to define abstractions. But languages have
other ways to define abstractions already: functions, classes, data
definitions, etc. If an abstraction can be implemented using these
features, it's almost always better to do so. Thus:
\begin{quote}
  Syntactic sugar should only be used to implement an abstraction if
  it cannot be implemented in the core language directly.
\end{quote}

Therefore, to find places where it is a \emph{good} idea to use sugar,
we can just look for things that most programming languages
\emph{cannot} abstract over:
\begin{enumerate}
  \item In most languages, variable names are first order and cannot
    be manipulated at run-time (e.g., a variable cannot be passed as
    an argument to a function: if you attempt to do so, the thing the
    variable is bound to will be passed instead). Therefore, creating
    new binding constructs is a good use for syntactic sugar in most
    languages. However, in R[CITE] variable names can be abstracted
    over (e.g. \code{assign("x", 3); x} prints 3), and so sugar isn't
    necessary for this purpose.
  \item Most languages force the arguments to a function to be
    evaluated when it is called, rather than allowing the function to
    choose whether to
    evaluate them or not. Thus if an abstraction requires delaying
    evaluation, it is a good candidate to be a sugar. However, Haskell
    has lazy evaluation, and thus does not need sugars for this purpose.
  \item Most languages cannot manipulate data definitions at run-time:
    e.g., field names cannot be dynamically constructed. Thus creating
    new data definition constructs (e.g., a way to define state
    machines) is a good use case for sugars. However, in Python field
    names are first class (e.g., they can be added or assigned using
    \code{setattr}), so sugars are not needed to abstract over fields
    in data definitions.
    % e.g.:
    %\code{class C: pass; c = C(); setattr(c, ``x'', 3); print(c.x)}
    \footnote{
    It may sound like I am suggesting that everything should be
    manipulatable at run-time. I am not. The more
    things which are fixed at compile time (variable names, field
    names, etc.), the more (i) programmers can reason about their
    programs; (ii) tools can reason about programs; (iii) compilers
    can optimize programs (without herculean effort). It is
    \emph{good} that sugar is sometimes necessary.
  }
\end{enumerate}

Overall, syntactic sugar is a way to \emph{extend a language}.
In a limited sense, this is what functions are for as well.
Functions, however, are limited: they cannot take a variable as an
argument, delay evaluation, introduce new syntax, etc. This is where
sugar shines.


\subsection{A Thousand Grains of Sugar}

There are many axes on which desugaring mechanisms vary:
\begin{enumerate}
  \item Desugaring is a syntax-to-syntax transformation, but what is
    the representation of syntax? There is a big difference between
    transformations on the \emph{text} of the program vs. its
    \emph{concrete surface syntax} vs. its \emph{abstract surface
      syntax}.
  \item Are sugars defined by developers of the language (and thus
    relatively fixed), or by users of the language (and thus flexible)? % macros?
  \item What is the metalanguage? That is, in what language are sugars
    written? Is it the same language the programs are written in, thus
    allowing sugars and code to be interspersed, or a different
    language?
  \item In what order are constructs desugared? Most importantly, are
    nested sugars evaluated from the \emph{innermost to outermost}, or
    from \emph{outermost to innermost}?
  \item How many \emph{phases} of desugaring are there? Can there be
    more than one?
  \item How \emph{safe} is it? Can desugaring produce syntactically
    invalid code? Can it produce an unbound variable, or accidentally
    capture a variable? Can it produce code that contains a type error?
\end{enumerate}

There is one big cluster of desugaring systems that should be called
out by name: macro systems. They can be loosely defined as:
\begin{quote}
  \emph{Macros} are user-defined sugars.
\end{quote}
In practice, macro systems tend to share several features: (i) by
definition, they are user-defined; (ii) the metalanguage is the
programming language itself [CHECK]; and (iii) the evaluation order is
usually outside-in [CHECK]. [CHECK: others?]

\subsection{Syntax Representation}

There are many ways to represent a program. The most prevalent are as
\emph{text} and as a \emph{tree}. Programs are most commonly
\emph{saved as} and \emph{edited as} text (notable exceptions include
visual and block based language/editors), and they are most commonly
\emph{internally represented as} trees (notable exceptions include
assembly, whose code is linear, and Forth, which does not have a
parsing phase).

Desugaring systems may be based on either representation.
\emph{However, text is a terrible representation for desugaring rules.}
The semantics of a language is almost always defined in terms of its
(abstract syntax) tree representation. Thus, insofar as a programmer
as a programmer is forced to think of their program as text rather
than as a tree, they are being distracted from its semantics. There
are well-known examples of bugs that arise in text-based desugaring
rules unless they are written in a very defensive style: I discuss
these in \ref{sec:cpre}.

There are variations among tree representations as well: desugaring
rules may work over the concrete syntax of the language, or over its
abstract syntax. I discuss this further in [REF].

\subsection{Language-defined or User-defined}

Sugars may either be specified and implemented as part of the
language, or they may be defined by users. For example, Haskell list
comprehensions are defined by the Haskell spec [CITE] and implemented
in the compiler(s); thus they are language-defined. Template Haskell
sugars [CITE], on the other hand, can be defined (and used) in any
Haskell program; thus they are user-defined. The difference between the two
is whether sugar is only a convenient method of simplifying language
design, or whether users are given the power to extend the language
themselves.

When sugars are user-defined, it raises difficulties with tools such
as editors that need to support the new syntax. For example, if sugars
are language-defined then an editor can just support the full
language. If they are user-defined, however, how can an editor provide
correct indentation, syntax highlighting, etc.?

Different languages work around this problem in different ways. Lisps
mostly avoid the problem by mostly not having syntax. They don't
\emph{entirely} avoid the problem, though. For example, the DrRacket editor for
Racket [CITE] allows indentation schemes to be set on a per-macro
basis (because different syntactic forms, while purely parenthetical,
still have varying nesting patterns that should be indented
differently), and it displays arrows showing where variables are bound
by being macro-aware and expanding the program (see [REF]).
[FILL: SugarJ, other examples]

[FILL: resugaring in general]

[TODO: macro-defining-macros]


\subsection{Metalanguage}


\subsection{Evaluation Strategy}
% https://dl.acm.org/citation.cfm?id=1440085
% GRAMMARS WITH MACRO-LIKE PRODUCTIONS (Fischer)

There are two major evaluation strategies used in desugaring systems.
They loosely correspond to eager and lazy runtime evaluation, but
differ in some important ways described below, so I will instead refer
to them by their original names [CITE]: Outside-in (\textsc{oi}) and
Inside-out (\textsc{io}) evaluation:
\begin{description}
\item[\textsc{oi}] evaluation is similar to lazy evaluation.
  However, it has an unusual property: [FILL]
\item[\textsc{io}] evaluation is similar to eager evaluation. It has [FILL]
\end{description}

\subsection{Number of Phases}

\subsection{Safety} [TODO]: AST safety, scope-safety, type-safety

\chapter{Desugaring in the Wild}

.[TODO: State version of each system discussed]

\section{C Preprocessor} \label{sec:cpre}


\desc{Evaluation Strategy} IO

\desc{Authorship} User-defined

\desc{Representation} Token stream

\desc{Safety} [FILL]

% https://gcc.gnu.org/onlinedocs/cpp/
% Not Turing complete: https://gcc.gnu.org/onlinedocs/cpp/Self-Referential-Macros.html
% Evaluation strategy: https://gcc.gnu.org/onlinedocs/cpp/Macro-Arguments.html
\desc{Discussion}
The C Preprocessor (hereafter \textsc{cpp}) [CITE] is a \emph{text
  preprocessor}: a source-to-source transformation that operates at
the level of text. (More precisely, it operates on a token stream, in
which the tokens are approximately those of the C language). It is
usually run before compilation for C or C++ programs, but it is not
very language specific, and can be used for other purposes as well.
\textsc{Cpp} is not Turing complete, by a simple mechanism: if a macro
invokes itself (directly or indirectly), the recursive invocation
will not be expanded.

%https://stackoverflow.com/questions/14041453/why-are-preprocessor-macros-evil-and-what-are-the-alternatives
A number of issues arise from the fact that \textsc{cpp} operates on tokens, and
is thus unaware of the higher-level syntax of C [CITE].
As an example, consider this innocent looking
\textsc{cpp} desugaring rule that defines an alias for subtraction:
\begin{codes}
  #define SUB(a, b) a - b
\end{codes}
This rule is completely broken. Suppose it is used as follows:
\begin{codes}
  SUB(0, 2 - 1))
\end{codes}
This will expand to \code{0 - 2 - 1} and evaluate to \code{-3}.
We can revise the rule to fix this:
\begin{codes}
  #define SUB(a, b) (a) - (b)
\end{codes}
This will fix the last example, but it is still broken. Consider:
\begin{codes}
  SUB(5, 3) * 2
\end{codes}
This will expand to \code{5 - 3 * 2} and evaluate to \code{-1}.
The rule can be fully fixed by another set of parentheses:
\begin{codes}
  #define SUB(a, b) ((a) - (b))
\end{codes}
In general, both the inside boundary of a rule (the arguments \code{a}
and \code{b}), and the outside boundary (the whole \textsc{rhs}) need
to be protected to ensure that the expansion is parsed correctly. If
the sugar is used in expression position, as in the \code{SUB}
example, this can be done with parentheses. In other positions,
different tricks must be used: e.g., a rule meant to be used in
statement position can be wrapped in \code{do \{...\} while(0)}.
Software developers should not need to know this.

There are other issues that arise with text-based transformations as
well, such as variable capture. Furthermore, all of these issues are
inherent to text-based transformations, and essentially cannot be
fixed from within the paradigm. \emph{Overall, code transformations
  should never operate at the level of text.}

\section{C++ Templates} \label{sec:cpp}

\desc{Representation} Concrete Syntax

\desc{Authorship} User-defined

\desc{Evaluation Strategy} IO

\desc{Safety} [FILL]

% http://www.open-std.org/jtc1/sc22/wg21/docs/papers/2017/n4659.pdf
\desc{Discussion} C++ templates [CITE] are not general-purpose sugars,
because they cannot take code as an argument.  Instead, they are used
primarily to instantiate polymorphic code by replacing type parameters
with concrete types.  Let's use the following template declaration,
taken from [CITE: pg344], as a running example. It declares a function
to compute the area of a circle, that can be instantiated with
different possible (presumably numeric) types \code{T}:
\begin{codes}
template<class T>
T circular_area(T r) \{
  return pi<T> * r * r;
\}
\end{codes}

Besides function definitions, several other kinds of declarations can
be templated, including methods, classes, structs, and type aliases.
The behavior of each is similar. A template may be invoked by passing
arguments in angle brackets. An invoked template acts like the
kind of thing the template declared, and can be used in the same
positions. Thus, e.g., a \code{struct} template should be invoked in type
position; and our running function template example should be invoked
in expression position to make a function, which can then be called:
\begin{codes}
  float area = circular_area<float>(1);
\end{codes}
When a template is invoked like this, a copy of the template
definition is made, with the template parameters replaced with the
concrete arguments.\marginpar{
  If a template is invoked multiple times with the same parameters,
  only one copy of the code will be made, however.
}
In our example, this produces the code:
\begin{codes}
float circular_area(float r) \{
  return pi<float> * r * r;
\}
\end{codes}

So far we have only described type parameters, but templates can also
take other kinds of parameters, including primitive values (such as
numbers) and other templates. The ability to manipulate numbers and
invoke other templates at compile time make C++ templates powerful
and, unsurprisingly, Turing complete. However, templates \emph{cannot}
be parameterized over code, and thus are not general-purpose sugars.
For example, most of the examples in this thesis cannot be written as
C++ templates.

Template expansion uses IO evaluation order. This is important because
it is possible
to define both a generic template, that applies most of the time, and
a specialized template, that applies if a parameter has a particular
value. For example, this could be used to make a \code{HashMap} use a different
implementation if its keys are \code{int}s. Thus it is important that
a template see the concrete type (e.g. \code{int}) that is passed to
it, even if this type is the result of another template expansion.


\section{Rust Macros} \label{sec:rust}

\desc{Representation} Concrete Syntax

\desc{Authorship} User-defined

\desc{Evaluation Strategy} OI

\desc{Safety} [FILL]

%https://doc.rust-lang.org/1.2.0/book/macros.html
\desc{Discussion}


\section{Haskell Templates} \label{sec:haskell}

\desc{Representation} Concrete or Abstract Syntax

\desc{Authorship} User-defined

\desc{Evaluation Strategy} IO

\desc{Safety} AST safe. Scope unsafe. Type unsafe.

%https://hackage.haskell.org/package/template-haskell-2.10.0.0/docs/Language-Haskell-TH-Syntax.html#t:Lit
%https://stackoverflow.com/questions/10857030/whats-so-bad-about-template-haskell
\desc{Discussion}
NOTES:
\begin{itemize}
  \item Must be defined in separate file
\end{itemize}


% Formatting
\newenvironment{prooftable}
{~\\\begin{tabular}{r l @{\quad} l}}
{\end{tabular}~\\}

% Terminology
\newcommand{\AST}{\textsc{ast}\xspace}

% Derivations
\newcommand{\saysE}[3]{#1 \vdash #2 : #3}
\newcommand{\saysP}[4]{#1;#2 \vdash #3 : #4}
\newcommand{\saysR}[4]{#1 \vdash #2 : #3 \to #4}
\newcommand{\saysS}[3]{#1 \vdash #2 : #3}
\newcommand{\saysStep}[3]{#1 \vdash #2 \rightsquigarrow #3}
\newcommand{\saysSteps}[3]{#1 \vdash #2 \rightsquigarrow^{*} #3}
\newcommand{\saysCase}[4]{#1 \vdash #2 \rightsquigarrow #3 \text{ by  case } #4}
\newcommand{\saysNotCase}[4]{#1 \not\vdash #2 \rightsquigarrow #3 \text{ by  case } #4}
\newcommand{\saysScope}[6]{#1 \vdash #2 : #3 ; #4 ; #5 ; #6}
\newcommand{\saysCanBind}[3]{#1 \vdash #2 \sim_{bind} #3}
\newcommand{\saysCanShadow}[3]{#1 \vdash #2 \sim_{shadow} #3}

\newcommandx*{\Inference}[3][1=\empty]{\inference[#1]{#2}{#3}\vspace{1.5em}}
\newcommand{\IS}{\vspace{0.25em}} % Not enough space by default...
\newcommandx*{\RZero}[1]{\AxiomC{$#1$}}
\newcommandx*{\ROne}[2][1=\empty]{\LeftLabel{\small{#1}}\UnaryInfC{$#2$}}
\newcommandx*{\RTwo}[2][1=\empty]{\LeftLabel{\small{#1}}\BinaryInfC{$#2$}}
\newcommandx*{\RThree}[2][1=\empty]{\LeftLabel{\small{#1}}\TrinaryInfC{$#2$}}
\newcommandx*{\RFour}[2][1=\empty]{\LeftLabel{\small{#1}}\QuaternaryInfC{$#2$}}
\newcommandx*{\RFive}[2][1=\empty]{\LeftLabel{\small{#1}}\QuinaryInfC{$#2$}}

% Terms, Patterns, Rules, Envs
\newcommand{\con}[2]{\{#1\,#2\}}
\newcommand{\app}[2]{(#1\,#2)}
\newcommand{\variable}[4]{
  {#3}\textrm{\textsc{$#2$}}_\mathit{#1}^\textsc{#4}}
\newcommandx*{\decl}[3][1=\empty, 3=\empty]{\variable{#1}{#2}{#3}{d}}
\newcommandx*{\refn}[3][1=\empty, 3=\empty]{\variable{#1}{#2}{#3}{r}}
\newcommand{\cons}[2]{#1, #2}
\newcommand{\rep}[2]{#1\ {*}\_{#2}}
\newcommand{\varindex}[2]{{#1}\_{#2}}
\newcommand{\pvarA}{\alpha}
\newcommand{\pvarB}{\beta}
\newcommand{\pvarC}{\gamma}
\newcommand{\dsrule}[2]{\code{sugar}\;#1 = \{#2\}}
\newcommand{\dsrulefancy}[3]{\code{sugar}\;#1 =
  \begin{cases}
    #2 \\
    \quad ... \\
    #3
  \end{cases}
}
\newcommand{\case}[4]{#1;#2;#3 \To #4}
\newcommand{\emptyEnv}{\{\}}
\newcommand{\emptySubs}{\{\}}
\newcommand{\consEnv}[3]{#1:#2,\,#3}

% Scope
\newcommand{\bind}[3]{\texttt{bind } #3 \texttt{ in } #2 \ensuremath{\in #1}}
\newcommand{\prov}[2]{\texttt{provide } #2 \ensuremath{\in #1}}

% Evaluation
\newcommand{\saysSubs}[4]{#1 \vdash #2 \bullet #3 = #4}
\newcommand{\saysMatch}[4]{#1 \vdash #2 / #3 = #4}

% Types
\newcommand{\Refn}{\textrm{Refn}}
\newcommand{\Decl}{\textrm{Decl}}
\newcommand{\String}{\textrm{String}}


\chapter{Resugaring for Pyret}

\section{Example}

\subsection{Define-struct}

\paragraph{Core AST}
\begin{codes}
Stmts:
| [\{splicing-begin stmts:Stmts\} @rest:Stmts]
   binding stmts in rest
   providing stmts, rest

| [\{let x:Var v:Expr\} @rest:Stmts]
   binding x in rest

| [\{fun f:Var args:Args body:Expr\} @rest:Stmts]
   binding args in body, rest in body
   providing f, rest

ALTERNATIVELY:

Stmts:
| \{splicing-begin stmts:Stmts rest:Stmts\}
   binding stmts in rest
   providing stmts, rest

| \{let x:Var v:Expr rest:Stmts\}
   binding x in rest

| \{fun f:Var args:Args body:Expr rest:Stmts\}
   binding args in body, rest in body
   providing f, rest

| \{end\}

Params:
| \{param x:Var rest:Params\}
| \{end\}
\end{codes}

\paragraph{Auxiliary AST}
\begin{codes}
IStructFields:
| [field:IStructField ...fields:IStructFields]
  providing field, fields

IStructField:
| \{i-struct-field field:Str get:Var set:Var\}
  providing get, set
\end{codes}

\paragraph{Surface AST}
\begin{codes}
SurfStmts:
  .....
| [(define-struct name:Var fields:StructFields) @rest:SurfStmts]
  binding name in rest, fields in rest
  providing name, fields, rest

StructFields:
| [field:StructField ...fields:StructFields]
  providing field, fields

StructField:
| (struct-field field:Str get:Var set:Var)
  providing get, set
\end{codes}

\paragraph{Desugaring Rules}
\begin{codes}
   [(struct-field field:Str get:Var set:Var) @rest:IStructFields]
=> [\{i-struct-field field get set\} @rest]
  
   [(define-struct name:Var
      [(struct-field field:Str get:Var set:Var) ...]) @rest:SurfStmts]
=> [\{fun name [x ...] \{record [\{record-field field x\} ...]\}\}
    \{splicing-begin [\{fun get [rec] \{record-get rec field\}\} ...]\}
    \{splicing-begin [\{fun set [rec val] \{record-set rec field val\}\} ...]\}
    @rest]
\end{codes}

\subsection{Pyret For Expressions}

To handle Pyret for-expressions, we need to do two things.
First, when a for-expression binding (e.g. \code{n from 0}) desugars,
it will simply return its binding (\code{n}) and its value (\code{0})
to the for-expression. It can do so with the desugaring rule:
\begin{codes}
   (s-for-bind l:Loc b:Bind v:Expr)
=> \{for-bind b v\}
\end{codes}
where \code{ForBind} is a new type:
\begin{codes}
  ForBind ::= \{for-bind Bind Expr\}

with list scope:
  [\{for-bind b v\} ...]
  export b
  export ...
\end{codes}

Then for-expressions can be implemented with the desugaring rule:
\begin{codes}
   (s-for l:Loc
          iter:Expr
          [\{ForBind bind:Bind value:Expr\} ...]@binds
          ann:Ann
          body:Expr
          blocky:Bool)
=> \{Lambda l (CONCAT "for-body<" (FORMAT l false) ">")
     [] [bind ...] ann "" body None None blocky\}
with scope:
  bind binds in body
\end{codes}

Notice that \code{s-for} is pattern matching against the results of
desugaring the \code{s-for-bind}s. The \code{(CONCAT ...)} stuff is to
compute at compile time a name for this lambda, which is what Pyret
currently does.

\section{Expressions}

\begin{jtable}
core name $C$ &$::=$& \textit{name} & core syntactic construct name \\
surface name $m$ &$::=$& \textit{name} & surface syntactic construct name \\
expression $e$ &$::=$& $\con{C}{e_1 ... e_n}$ & core syntactic construct \\
  &$|$& $\app{m}{e_1 ... e_n}$ & surface syntactic construct \\
  &$|$& $[e_1 ... e_n]$ & list \\
  &$|$& $string$ & string literal \\
  &$|$& $\refn[i]{x}$ $|$ $\decl[i]{x}$  & variable \\
value $v$ &$::=$& $e$ & with no sugar invocations \\
\end{jtable}



\section{Expansion}

\begin{jtable}
ellipsis label $l$ &$::=$& \textit{name} & ellipsis label \\
%shape $\dot{e}$ &$::=$& ...e... & (same cases as $e$) \\
%  &$|$& $\bullet$ & hole \\
pattern $p$ &$::=$& $\pvarA$ & pattern variable \\
  &$|$& $\con{C}{p_1 ... p_n}$ & syntactic construct \\
  &$|$& $\app{m}{p_1 ... p_n}$ & sugar invocation \\
  &$|$& $[ps]$ & list \\
  &$|$& $string$ & string literal \\
  &$|$& $\refn[i]{x}$ $|$ $\decl[i]{x}$  & variable \\
seq. pattern $ps$ &$::=$& $\epsilon$ & empty sequence \\
  &$|$& $\cons{p}{ps}$ & cons \\
  &$|$& $\rep{p}{l}$ & ellipsis with label $l$ \\
fresh vars $F$ &$::=$& $\{x,...\}$ & fresh variable set \\
type env. $\Gamma$ &$::=$&
$\begin{cases}
  \pvarA:t, ... \\
  i \mapsto [\Gamma], ...
\end{cases}$ \\
substitution $\gamma$ &$::=$&
$\begin{cases}
  \pvarA \mapsto e, ... \\
  l \mapsto [\gamma ... \gamma], ... \\
  x \mapsto x, ...
\end{cases}$
\end{jtable}

\begin{jtable}
rewrite case $c$ &$::=$&
  $\case{(p_1,\,...,\,p_k)}{\Gamma}{F}{p'}$ \\
desugaring rule $r$ &$::=$&
  $\dsrule{m}{c_1,...,c_n}$ \\
desugaring rules $rs$ &$::=$& $\{r_1, ..., r_n\}$
\end{jtable}

\subsection{Matching and Substitution}

\begin{figure}
\fbox{$\saysMatch{F}{e}{p}{\gamma}$}
\begin{multicols}{2}
  \Inference[m-pvar]{}{
    \saysMatch{F}{e}{\pvarA}{\{\pvarA \mapsto e\}}
  }

  \Inference[m-capture]{
    x \not\in F
  }{
    \saysMatch{F}{x}{x}{\{\}}
  }

  \Inference[m-fresh]{
    x \in F
  }{
    \saysMatch{F}{y}{x}{\{x \mapsto y\}}
  }

  \Inference[m-str]{}{
    \saysMatch{F}{string}{string}{\{\}}
  }

  \Inference[m-empty]{}{
    \saysMatch{F}{[\phantom{.}]}{\epsilon}{\{\}}
  }

  \Inference[m-cons]{
    \saysMatch{F}{e_1}{p}{\gamma_1} \\
    \saysMatch{F}{[e_2,...,e_n]}{ps}{\gamma_s} \\
    \gamma_1 \DisjUnion \gamma_2 = \gamma
  }{
    \saysMatch{F}{[e_1 ... e_n]}{p,ps}{\gamma}
  }
\end{multicols}
\vspace{1em}

\Inference[m-con]{
  \saysMatch{F}{e_1}{p_1}{\gamma_1} \;\cdots\; \saysMatch{F}{e_n}{p_n}{\gamma_n}
  \gamma_1 \DisjUnion ... \DisjUnion \gamma_n = \gamma
}{
  \saysMatch{F}{\con{C}{e_1 ... e_n}}{\con{C}{p_1 ... p_n}}{\gamma}
}

\Inference[m-star]{
  \saysMatch{F}{e_1}{p}{\gamma_1} \;\cdots\; \saysMatch{F}{e_n}{p}{\gamma_n}
}{
  \saysMatch{F}{[e_1 ... e_n]}{[\rep{p}{l}]}{\{l \mapsto [\gamma_1 ... \gamma_n]\}}
}

\fbox{$\saysSubs{F}{\gamma}{p}{e}$}
\begin{multicols}{2}
  \Inference[s-pvar]{
    \pvarA \mapsto e \in \gamma
  }{
    \saysSubs{F}{\gamma}{\pvarA}{e}
  }

  \Inference[s-str]{}{
    \saysSubs{F}{\gamma}{string}{string}
  }

  \Inference[s-capture]{
    x \not\in F
  }{
    \saysSubs{F}{\gamma}{x}{x}
  }

  \Inference[s-fresh]{
    x \in F & x \mapsto y \in \gamma
  }{
    \saysSubs{F}{\gamma}{x}{y}
  }

  \Inference[s-empty]{}{
    \saysSubs{F}{\gamma}{[\epsilon]}{[\phantom{.}]}
  }

  \Inference[s-cons]{
    \saysSubs{F}{\gamma}{p}{e_1} \\
    \saysSubs{F}{\gamma}{[ps]}{[e_2,...,e_n]}
  }{
    \saysSubs{F}{\gamma}{[p,ps]}{[e_1 e_2 ... e_n]}
  }
\end{multicols}
\vspace{1em}

\Inference[s-star]{
  l \mapsto [\gamma_1 ... \gamma_n] \in \gamma \\
  \saysSubs{F}{\gamma_1}{p}{e_1} \;\cdots\; \saysSubs{F}{\gamma_n}{p}{e_n}
}{
  \saysSubs{F}{\gamma}{[\rep{p}{l}]}{[e_1 ... e_n]}
}

\Inference[s-con]{
  \saysSubs{F}{\gamma}{p_1}{e_1} \;\cdots\; \saysSubs{F}{\gamma}{p_n}{e_n}
}{
  \saysSubs{F}{\gamma}{\con{C}{p_1 ... p_n}}{\con{C}{e_1 ... e_n}}
}

\Inference[s-sugar]{
  \saysSubs{F}{\gamma}{p_1}{e_1} \;\cdots\; \saysSubs{F}{\gamma}{p_n}{e_n}
}{
  \saysSubs{F}{\gamma}{\app{m}{p_1 ... p_n}}{\app{m}{e_1 ... e_n}}
}

\caption{Matching and Substitution}
\end{figure}

\begin{lemma}[matching and substitution]
  Matching and substitution are inverses:
%  $\saysSubs{F}{\gamma}{p}{e}$, then $\saysMatch{F}{e}{p}{\gamma}$.
\end{lemma}
\begin{proof}
  Induct on $p$.
  [FILL]
\end{proof}
%However, the reverse is not true. Matching does not undo substitution,
%because substitution in non-deterministic (because it generates fresh
%variables).

\subsection{Expansion}

See \cref{fig:expansion}.
[TODO: Replace step with something that looks like desugaring.]
[TODO: Replace $v$ with something that looks like core terms.]

\begin{figure}
  \Inference[eval-ctx]{
    \saysStep{L}{e}{e'}
  }{
    \saysStep{L}{E[e]}{E[e']}
  }
  \Inference[eval-expand]{
    L = G,rs &
    \dsrule{m}{c_1 ... c_n} \in G \IS\\
    \saysCase{L}{\app{m}{e_1 ... e_n}}{e''}{c_i} \IS\\
    \saysNotCase{L}{\app{m}{e_1 ... e_n}}{e'}{c_j} \text{ for any } j<i
  }{
    \saysStep{L}{\app{m}{e_1 ... e_n}}{e'}
  }
  \Inference[eval-case]{
    \saysMatch{L}{e_i}{p_i}{\gamma_i} \text{ for each $i$} \IS \\
    \gamma' \text{ gives fresh names to the variables in $F$} \\
    \gamma_1 \DisjUnion ... \DisjUnion \gamma_n \DisjUnion \gamma' = \gamma \\
    \saysSubs{F}{\gamma}{p'}{e'}
  }{
    \saysCase{L}{\app{m}{e_1 ... e_n}}{e'}{(p_1,...,p_n);\Gamma;F \To p'}
  }
  \caption{Expansion}
  \label{fig:expansion}
\end{figure}

[FILL] One expansion rule. Note expansion contexts.



\section{AST Checking} % Or Syntype Checking

\begin{jtable}
ast defn. $G$ &$::=$& $A \mapsto \{t_1, ... t_n\}$
  & (with no production $A_1 \mapsto A_2$) \\
syntactic category $A$ &$::=$& \textit{name} \\
syntax type $t$ &$::=$& $A$ & syntactic category \\
  &$|$& $\con{C}{t_1 ... t_n}$ & syntactic construct \\
  &$|$& $[t]$ & list \\
  &$|$& String & string literal \\
  &$|$& Decl & variable declaration \\
  &$|$& Refn & variable reference \\
language $L$ &$::=$& $G, rs$
\end{jtable}

Lemma: If rules grammar check, then e obeys Surf implies ds(e) obeys
Core.

Lemma: Normalizing a grammar does not change its language.

\paragraph{Exhaustion Checking}
We perform exhaustion checking to make sure that sugars cover all
possible cases of their arguments, but do not give the algorithm here.
It works by looking at \emph{shapes}: a shape is a pattern that
contains types in place of pattern variables. It is straightforward to
check whether an expression matches a shape, and to convert a pattern
into a shape. Exhaustion checking uses the fact that the expressions
that do \emph{not} match a shape can be expressed as a union of shapes.
[TODO: prove]

\begin{figure}

\fbox{$\saysE{L}{e}{t}$}

\begin{multicols}{2}
  
  \Inference[e-con]{
    A \mapsto \con{C}{t_1 ... t_n} \in L \\
    \saysE{L}{e_1}{t_1} \;\cdots\; \saysE{L}{e_n}{t_n}
  }{
    \saysE{L}{\con{C}{e_1 ... e_n}}{A}
  }

  \Inference[e-refn]{}{
    \saysE{L}{\refn{x}}{\Refn}
  }

  \Inference[e-decl]{}{
    \saysE{L}{\decl{x}}{\Decl}
  }

  \Inference[e-str]{}{
    \saysE{L}{\textit{string}}{\String}
  }

  \Inference[e-list]{
    \saysE{L}{e_1}{t} \;\cdots\; \saysE{L}{e_n}{t}
  }{
    \saysE{L}{[e_1 ... e_n]}{[t]}
  }

  \Inference[e-sugar]{
    \saysR{L}{m}{t_1,...,t_n}{t} \\
    \saysE{L}{e_1}{t_1} \;\cdots\; \saysE{L}{e_n}{t_n}
  }{
    \saysE{L}{\app{m}{e_1 ... e_n}}{t}
  }
\end{multicols}

\fbox{$\saysP{L}{\Gamma}{p}{t}$}

\begin{multicols}{2}

  \Inference[p-pvar]{
    \pvarA : t \in \Gamma
  }{
    \saysP{L}{\Gamma}{\pvarA}{t}
  }

  \Inference[p-refn]{}{
    \saysP{L}{\Gamma}{x}{\Refn}
  }

  \Inference[p-decl]{}{
    \saysP{L}{\Gamma}{x}{\Decl}
  }

  \Inference[p-str]{}{
    \saysP{L}{\Gamma}{\textit{string}}{\String}
  }

  \Inference[p-con]{
    A \mapsto \con{C}{t_1 ... t_n} \in L \\
    \saysP{L}{\Gamma}{p_1}{t_1} \;\cdots\; \saysP{L}{\Gamma}{p_n}{t_n}
  }{
    \saysP{L}{\Gamma}{\con{C}{p_1 ... p_n}}{A}
  }

  \Inference[p-sugar]{
    \saysR{L}{m}{t_1,...,t_n}{t} \\
    \saysP{L}{\Gamma}{p_1}{t_1} \;\cdots\; \saysP{L}{\Gamma}{p_n}{t_n}
  }{
    \saysP{L}{\Gamma}{\app{m}{p_1 ... p_n}}{t}
  }

  \Inference[p-empty]{}{
    \saysP{L}{\Gamma}{[\epsilon]}[t]
  }

  \Inference[p-cons]{
    \saysP{L}{\Gamma}{p}{t} \\
    \saysP{L}{\Gamma}{[ps]}{[t]}
  }{
    \saysP{L}{\Gamma}{[\cons{p}{ps}]}{[t]}
  }

  \Inference[p-star]{
    l \mapsto [\Gamma'] \in \Gamma & \saysP{L}{\Gamma'}{p}{t}
  }{
    \saysP{L}{\Gamma}{[\rep{p}{l}]}{[t]}
  }
\end{multicols}

\fbox{$\saysR{L}{m}{t,...,t}{t}$}
\Inference[g-sugar]{
  \dsrulefancy
      {m}
      {(p_{11},...,p_{1n});\Gamma_1;F_1 \To p_1'}
      {(p_{k1},...,p_{kn});\Gamma_k;F_k \To p_k'}
      \in L \\
  \text{The cases are exhaustive over $t_1,...,t_n$ in $G$} \\
  \saysP{L}{\Gamma_{i}}{p_{ij}}{t_j}
    \text{ for each $i \in 1..k, j \in 1..n$} \\
  \saysP{L}{\Gamma_{i}}{p_i'}{t}
    \text{ for each $i \in 1..k$}
}{
  \saysR{L}{m}{t_1,...,t_n}{t}
}

\fbox{$\saysS{L}{\gamma}{\Gamma}$}
\Inference[$\gamma$-env]{
  \saysE{L}{e_1}{t_1} \;\cdots\; \saysE{e_n}{t_n} \\
  \saysS{L}{\gamma_{11}}{\Gamma_1} \;\cdots\; \saysS{L}{\gamma_{1j}}{\Gamma_1} \\
  ... \\
  \saysS{L}{\gamma_{m1}}{\Gamma_m} \;\cdots\; \saysS{L}{\gamma_{mk}}{\Gamma_m} \\
}{
  \saysS{L}{
    \begin{cases}
      \pvarA_1 \mapsto e_1,\,...,\,\pvarA_n \mapsto e_n \\
      i_1 \mapsto [\gamma_{11},\,...,\,\gamma_{1j}] \\
      ... \\
      i_m \mapsto [\gamma_{m1},\,...,\,\gamma_{mk}]
    \end{cases}
  }{
    \Gamma',
    \begin{cases}
      \pvarA_1: t_1,\,...,\,\pvarA_n: t_n, \\
      %\pvarA_1': t_1',\,...,\,\pvarA_{n'}': t_{n'}' \\
      i_1 \mapsto [\Gamma_1],\,...,\,i_m \mapsto [\Gamma_m], \\
      %i_1' \mapsto [\Gamma_1'],\,...,\,i_{m'}' \mapsto [\Gamma_{m'}']
    \end{cases}
  }
}

\caption{Grammar Checking}
\end{figure}

\subsection{Type Soundness}

We prove soundness by way of progress + preservation:
\begin{theorem}[Soundness]
  If $\saysE{L}{e}{t}$, then
  $\saysSteps{L}{e}{v}$ where $\saysE{L}{v}{t}$, or $e$ runs forever.
\end{theorem}
\begin{proof}
\Cref{thm:progress} (progress) and \cref{thm:preservation}
(preservation) together imply that either:
(i) $e$ is a value, or (ii) $\saysStep{L}{e}{e'}$ and $\saysE{L}{e'}{t}$.
Apply this repeatedly. Either $e$ eventually steps to a value $v$, and
has remained the same type $t$ throughout the evaluation, or $e$ never
halts.
\end{proof}

\begin{lemma}[Progress] \label{thm:progress}
  If $\saysE{L}{e}{t}$, then
  $\saysStep{L}{e}{e'}$, or $e$ is a value.
\end{lemma}
\begin{proof}
  First, verify that our evaluation contexts include every case that
  isn't a value. Thus either $e$ is a value and we are done, or $e$
  contains a redex: $e=E[\app{m}{e_1 ... e_n}]$.
  In the latter case, we will show that $e$ can take a step because
  the eval-expand rule applies. There are two premises that need to be
  satisfied:
  \begin{itemize}
    \item First, we must show that $m$ is bound in $L$. Since $e$
      type-checked, it must be: the only rule which can type-check a
      sugar invocation is p-sugar; this in turn must use rule
      g-sugar; finally g-sugar requires that $m \in L$.
    \item Second, we must show that the pattern match of $e_1,...,e_n$
      succeeds on any case $(p_1,...,p_n);\Gamma;F$ of the desugaring
      rule. By \cref{thm:exhaustion}, it does.
  \end{itemize}
\end{proof}

\begin{assumption}[Exhaustion] \label{thm:exhaustion}
  If the set of cases in a desugaring rule are exhaustive over
  $t_1,...,t_n$ according to our exhaustion checking algorithm, then
  for every possible argument list $e_1,...,e_n$ that matches the
  given types (i.e., $\saysE{L}{e_1}{t_1},...,\saysE{L}{e_n}{t_n}$),
  there is a case $c_i$ such that $e_1,...,e_n$ successfully matches
  against $c_i$. [TODO: prove]
\end{assumption}
\begin{proof}
  \emph{Not given}. We have not stated our exhaustion checking
  algorithm here, and so cannot prove it correct. We believe it is
  straightforward (if tedious).
\end{proof}

\begin{lemma}[Preservation] \label{thm:preservation}
  If $\saysE{L}{e}{t}$ and $\saysStep{L}{e}{e'}$, then $\saysE{L}{e'}{t}$.
\end{lemma}
\begin{proof}
  Since $e$ can take an expansion step, it must have a redex (via
  eval-ctx): $e = E[\app{m}{e_1 ... e_n}]$. And furthermore (by eval-expand) $m$
  must be bound in $L$, and there must be a first case of $m$ that
  matches $e$.  Call it $c_i = (p_1,...,p_n);\Gamma \To p'$. Then:
  \begin{prooftable}
  By eval-case: & $\saysMatch{L}{e_i}{p_i}{\gamma_i}$
    for some $\gamma_i$ for each $i$ & (1) \\
  and & $\saysSubs{F}{\gamma_i \DisjUnion ...
    \DisjUnion \gamma_n}{p'}{e'}$ & (2) \\
  and & $\saysStep{L}{E[e]}{E[e']}$ \\
  By e-sugar: & $\saysE{L}{\app{m}{e_1 ... e_n}}{t}$ \\
  and & $\saysR{L}{m}{t_1 ... t_n}{t}$ \\
  and & $\saysE{L}{e_i}{t_i}$ for each $i$ & (3) \\
  By g-sugar: & $\saysP{L}{\Gamma}{p_i}{t_i}$ for each $i$ & (4) \\
  and & $\saysP{L}{\Gamma}{p'}{t}$ & (5)
  \end{prooftable}
  By \cref{thm:matching} with (1), (3), and (4),
  $\saysS{\gamma_i}{\Gamma}$ for each $i$. By \cref{thm:union},
  $\saysS{\gamma_1 \DisjUnion ... \gamma_n}{\Gamma}$.
  Finally, by \cref{thm:substitution} with that last fact together
  with (2) and (5), $\saysE{L}{e'}{t}$.
\end{proof}

\begin{lemma}[Union of Substitutions] \label{thm:union}
  If $\saysS{L}{\gamma_1}{\Gamma}$ and $\saysS{L}{\gamma_2}{\Gamma}$,
  then $\saysS{L}{\gamma_1 \DisjUnion \gamma_2}{\Gamma}$.
\end{lemma}
\begin{proof}
  [TODO]
\end{proof}

\begin{lemma}[Matching] \label{thm:matching}
  If $\saysP{L}{\Gamma}{p}{t}$
  and $\saysE{L}{e}{t}$
  and $\saysMatch{F}{e}{p}{\gamma}$,
  then $\saysS{L}{\gamma}{\Gamma}$
\end{lemma}
\begin{proof}
  Induction on $p$.
  \begin{description}
  \item[$p = string$]
    \begin{prooftable}
      By p-str: & $\saysP{L}{\Gamma}{string}{\String}$ & fixes $t$ \\
      By m-str: & $\saysMatch{F}{string}{\String}{\emptySubs}$
        & fixes $\gamma$
    \end{prooftable}
    Finally, by $\gamma$-env, $\saysS{F}{\emptySubs}{\Gamma}$
    (this applies for any $\Gamma$).
  \item[$p = x \not\in F$] (Analogous.)
  \item[$p = x \in F$] By p-refn or p-decl, 
    $\Gamma = \{\}$ and $t$ is {\Refn} or {\Decl}.
    By m-fresh, $e = y$ for some fresh name $y$, and $\gamma = \{\}$.
    And the conclusion follows: $\saysS{L}{\{\}}{\{\}}$. [TODO]
  \item[$p = \pvarA$]
    \begin{prooftable}
      By p-pvar: & $\saysP{L}{\Gamma}{\pvarA}{t}$ & fixes $t$ \\
      and & $\pvarA: t \in \Gamma$ & (1) \\
      By m-pvar: & $\saysMatch{F}{e}{\alpha}{\{\pvarA \mapsto e\}}$
        & fixes $\gamma$
    \end{prooftable}
    Finally, using $\gamma$-env on the premise $\saysE{L}{e}{t}$
    gives that $\saysS{L}{\gamma}{\{\pvarA: t\}},\Gamma'$ for any
    $\Gamma'$. By (1), this is the form of $\Gamma$, so we can set
    $\Gamma'$ such that $\Gamma = {\{\pvarA: t\}},\Gamma'$, and we are done.
  \item[$p = \con{C}{p_1 ... p_n}$]
    \begin{prooftable}
      By p-con: & $\saysP{L}{\Gamma}{\con{C}{p_1 ... p_n}}{A}$ & fixes $t$ \\
      and & $A \mapsto \con{C}{t_1 ... t_n} \in L$ \\
      and & $\saysP{L}{\Gamma}{p_i}{t_i}$ for each $i$ & (1) \\
      By m-con: &
        $\saysMatch{F}{\con{C}{e_1 ... e_n}}{\con{C}{p_1 ... p_n}}{\gamma}$
        & fixes $e$ \\
      and & $\saysMatch{F}{e_i}{p_i}{\gamma_i}$ for each $i$ & (2) \\
      and & $\gamma = \gamma_1 \DisjUnion ... \DisjUnion \gamma_n$ \\
      By e-con: & $\saysE{L}{\con{C}{e_1 ... e_n}}{A}$ \\
      and & $\saysE{L}{e_i}{t_i}$ for each $i$ & (3) \\
    \end{prooftable}
    Applying the I.H. to (1), (2), and (3) yeilds that
    $\saysS{L}{\gamma_i}{\Gamma}$.
    By \cref{thm:union}, $\saysS{L}{\gamma}{\Gamma}$.
  \item[$p = \app{m}{p_1 ... p_n}$] [FILL]
  \item[$p = [\epsilon{]}$] [TODO] By m-empty, $\gamma = \{\}$.
    By p-empty, $\Gamma = \{\}$. The goal follows: $\saysS{L}{\{\}}{\{\}}$.
  \item[$p = [p,ps{]}$] [FILL]
  \item[$p = [\rep{p}{l'}{]}$]
    \begin{prooftable}
      By p-star: & $\saysP{L}{\Gamma}{\rep{p}{l'}}{[t]}$ & fixes $t$ \\
      and & $l' \mapsto [\Gamma'] \in \Gamma$ & (1) \\
      and & $\saysP{L}{\Gamma'}{p}{t}$ & (2) \\
      By m-star: & $\saysMatch{F}{[e_1 ... e_n]}{[\rep{p}{l}]}
        {\{l' \mapsto [\gamma_1 \DisjUnion ... \gamma_n]\}}$
        & fixes $e$, $\gamma$ \\
      and & $\saysMatch{F}{e_i}{p}{\gamma_i}$ for each $i$ & (3) \\
      By e-list: & $\saysE{L}{[e_1 ... e_n]}{[t]}$ \\
      and & $\saysE{L}{e_i}{t_i}$ & (4)
    \end{prooftable}
    By the I.H. together with (2), (3), and (4),
    $\saysS{L}{\gamma_i}{\Gamma'}$ for each $i$.
    By \cref{thm:union},
    $\saysS{L}{\gamma_1 \DisjUnion ... \gamma_n}{\Gamma'}$.
    Finally, by $\gamma$-env,
    $\saysS{L}{\{l' \mapsto [\gamma_1 \DisjUnion ... \gamma_n]\}}
      {\{l' \mapsto [\Gamma']\}}$, which is compatible with the
      specification of $\Gamma$ in (1).
  \end{description}
\end{proof}

\begin{lemma}[Substitition] \label{thm:substitution}
  If $\saysS{L}{\gamma}{\Gamma}$
  and $\saysP{L}{\Gamma}{p}{t}$,
  and $\saysSubs{F}{\gamma}{p}{e}$,
  then $\saysE{L}{e}{t}$.
\end{lemma}
\begin{proof}
  Induction on $p$.
  \begin{description}
  \item[$p = string$]
  \item[$p = string$] By s-str, $\saysSubs{F}{\gamma}{p}{string}$, so $e=string$.
    By p-str, $\saysP{L}{\Gamma}{p}{\String}$, so $t=\String$.
    Finally, by e-str, $\saysE{L}{e}{\String}$ as desired.
  \item[$p = x \not\in F$] (Analogous.)
  \item[$p = x \in F$] By s-fresh, $e = y$ for some fresh name $y$.
    By p-refn or p-decl, $t$ is either {\Refn} or {\Decl}.
    Our goal $\saysE{L}{y}{t}$ follows by either e-refn or e-decl,
    respectively.
  \item[$p = \pvarA$] By rule s-pvar, $\pvarA \mapsto e \in \gamma$.
    By $\gamma$-env, $\alpha \mapsto t \in \Gamma$ and $\saysE{L}{e}{t}$.
    Which is our goal; we are done.
    (Note that by $\gamma$-env, $\Gamma$ may have many \emph{other},
    unnecessary, bindings to pattern variables, but it must \emph{at least}
    contain a correct binding for $\alpha$.)
  \item[$p = \con{C}{p_1 ... p_n}$]
    \begin{prooftable}
      By p-con: & $\saysP{L}{\Gamma}{\con{C}{p_1 ... p_n}}{A}$ & fixes $t$ \\
      and & $A \mapsto \con{C}{t_1 ... t_n} \in L$ & (1) \\
      and & $\saysP{L}{\Gamma}{p_i}{t_i}$ for each $i$ & (2) \\
      By s-con: & $\saysSubs{F}{\gamma}{\con{C}{p_1 ... p_n}}{\con{C}{e_1 ... e_n}}$
        & fixes $e$ \\
      and & $\saysSubs{F}{\gamma}{p_i}{e_i}$ for each $i$ & (3)
    \end{prooftable}
    Using the I.H. with (2) and (3) gives that
    $\saysE{L}{e_i}{t_i}$ for each $i$.
    Using e-con on that fact together with (1) gives that
    $\saysE{L}{\con{C}{e_1 ... e_n}}{A}$, so we are done.
  \item[$p = \app{m}{p_1 ... p_n}$]
    \begin{prooftable}
      By s-sugar: & $\saysSubs{F}{\gamma}{\app{m}{p_1 ... p_n}}{\app{m}{e_1 ... e_n}}$
        & fixes $e$ \\
      and & $\saysSubs{F}{\gamma}{p_i}{e_i}$ for each $i$ & (1) \\
      By p-sugar: & $\saysP{L}{\Gamma}{\app{m}{p_1 ... p_n}}{t}$
        & fixes $t$ \\
      and & $\saysP{L}{\Gamma}{p_i}{t_i}$ for each $i$ & (2) \\
      and & $\saysR{L}{m}{t_1,...,t_n}{t}$ & (3)
    \end{prooftable}
    Using the I.H. with (1) and (2) gives that
    $\saysE{L}{e_i}{t_i}$ for each $i$.
    Finally, using e-sugar on that fact together with (3) gives that
    $\saysE{L}{\app{m}{e_1 ... e_n}}{t}$.
  \item[$p = [\epsilon{]}$]
    By s-empty, $\saysSubs{F}{\gamma}{p}{[]}$, so $e=[]$.
    By p-empty, $\saysP{L}{\emptyEnv}{[\epsilon]}{[t]}$ (for some $t$).
    Finally, by e-list, $\saysE{L}{[]}{[t]}$.
  \item[$p = [p,ps{]}$]
    \begin{prooftable}
      By s-cons: & $\saysSubs{F}{\gamma}{p}{e_1}$ & (1) \\
      and & $\saysSubs{F}{\gamma}{[ps]}{[e_2 ... e_n]}$ & (2) \\
      and & $\saysSubs{F}{\gamma}{[p, ps]}{[e_1 e_2 ... e_n]}$ & fixes $e$ \\
      By p-cons: & $\saysP{L}{\Gamma}{p}{t}$ & (3) \\
      and & $\saysP{L}{\Gamma}{[ps]}{[t]}$ & (4) \\
      and & $\saysP{L}{\Gamma}{[p, ps]}{[t]}$ & fixes $t$
    \end{prooftable}
    We can apply the I.H. using (1) and (3) and the assumption
    $\saysS{L}{\gamma}{\Gamma}$ to get that $\saysE{L}{e_1}{t}$.
    Likewise, the I.H. with (2) and (4) gives
    $\saysE{L}{[e_2 ... e_n]}{[t]}$.
    By e-list (in reverse), $\saysE{L}{e_2}{t} \cdots \saysE{L}{e_n}{t}$.
    Finally, by e-list (forward), $\saysE{L}{[e_1 e_2 ... e_n]}{[t]}$.
  \item[$p = [\rep{p}{l}{]}$]
    \begin{prooftable}
      By s-star: & $\saysSubs{F}{\gamma}{[\rep{p}{l}]}{[e_1 ... e_n]}$ & fixes $e$ \\
      and & $l \mapsto [\gamma_1 ... \gamma_n] \in \gamma$ \\
      and & $\saysSubs{F}{\gamma_i}{p}{e_i}$ for each $i$ & (1) \\
      By $\gamma$-env: & $l \mapsto [\Gamma'] \in \Gamma$ \\
      and & $\saysS{L}{\gamma_i}{\Gamma'}$ for each $i$ & (2) \\
      By p-star: & $\saysP{L}{\Gamma}{[\rep{p}{l}]}{[t]}$ & fixes $t$ \\
      and & $\saysP{L}{\Gamma'}{p}{t}$ & (3)
    \end{prooftable}
    Using the I.H. with (1), (2), and (3) proves that
    $\saysE{L}{e_i}{t}$.
    Then, by e-list, $\saysE{L}{[e_1 ... e_n]}{[t]}$ as desired.
  \end{description}
\end{proof}

\section{Scope Checking}

(See \cref{fig:scope}.)

\begin{figure}
\fbox{$\saysScope{\Sigma}{e}
  {\{\decl{x}\}}
  {\{\refn{x}\}}
  {\{\decl{x}\}}
  {\{\refn{x}\mapsto\decl{x}\}}$}

\Inference[scope-e-decl]{}{
  \saysScope{\Sigma}{\decl{x}}{\{\decl{x}\}}{\{\}}{\{\decl{x}\}}{\{\}}
}

\Inference[scope-e-refn]{}{
  \saysScope{\Sigma}{\refn{x}}{\{\}}{\{\refn{x}\}}{\{\}}{\{\}}
}

\Inference[scope-e-con]{
  \Sigma[C] = \sigma \vspace{0.1em} \IS\\
  \saysScope{\Sigma}{e_i}{P_i}{R_i}{B_i} \text{ for } i \in 1..n \IS\\
  S = \SetSuchThat{
    \decl[a]{x} \mapsto \decl[b]{x}
  }{
    \decl[a]{x} \in P_i,\;
    \decl[b]{x} \in P_j,\;
    \bind{\sigma}{i}{j}
  } \IS\\
  B = \SetSuchThat{
    \refn[a]{x} \mapsto \decl[b]{x}
  }{
    \refn[a]{x} \in R_i,\;
    \decl[b]{x} \in P_j,\;
    \bind{\sigma}{i}{j},\;
    \decl[b]{x} \not\in \domain{S}
  } \IS\\
  R = \SetSuchThat{
    \refn[a]{x}
  }{
    \refn[a]{x} \in R_i,\;
    \NotExists{\decl[b]{x}}{\refn[a]{x} \mapsto \decl[b]{x} \in B\}}
  } \IS\\
  P = \SetSuchThat{
    \decl[a]{x}
  }{
    \decl[a]{x} \in P_i,\;
    \prov{\sigma}{i},\;
    \decl[a]{x} \not\in \domain{S}
  }
}{
  \saysScope{\Sigma}{\con{C}{e_1 ... e_n}}{P}{R}{B}
}


\Inference[scope-p-con]{
  \Sigma[C] = \sigma \vspace{0.1em} \IS\\
  \saysScope{\Sigma}{p_i}{P_i}{R_i}{B_i} \text{ for } i \in 1..n \IS\\
  S = \SetSuchThat{
    a \mapsto b
  }{
    a \in P_i,\;
    b \in P_j,\;
    \bind{\sigma}{i}{j},\;
    \saysCanShadow{F}{a}{b}
  } \IS\\
  B = \SetSuchThat{
    a \mapsto b
  }{
    a \in R_i,\;
    b \in P_j,\;
    \bind{\sigma}{i}{j},\;
    b \not\in \domain{S},\;
    \saysCanBind{F}{a}{b}
  } \IS\\
  R = \SetSuchThat{
    a
  }{
    a \in R_i,\;
    (\NotExists{b}{a \mapsto b \in B\}}
    \text{ or $a$ is a pattern var})
  } \IS\\
  P = \SetSuchThat{
    a
  }{
    a \in P_i,\;
    \prov{\sigma}{i},\;
    a \not\in \domain{S}
  }
}{
  \saysScope{\Sigma}{\con{C}{p_1 ... p_n}}{P}{R}{B}
}

Two checks to make: fresh vars don't bind to non-fresh vars, and named
vars only bind to vars of the same name:
\begin{multicols}{3}
  \Inference{}{
    \saysCanBind{F}{\refn[1]{x}}{\decl[2]{x}}
  }
  \Inference{
    x \not\in F
  }{
    \saysCanBind{F}{\refn[1]{x}}{\pvarA}
  }
  \Inference{
    x \not\in F
  }{
    \saysCanBind{F}{\pvarA}{\decl[1]{x}}
  }
  \Inference{}{
    \saysCanBind{F}{\pvarA}{\pvarB}
  }
  \Inference{}{
    \saysCanShadow{F}{\decl[1]{x}}{\decl[2]{x}}
  }
  \Inference{
    x \not\in F
  }{
    \saysCanShadow{F}{\decl[1]{x}}{\pvarA}
  }
\end{multicols}

\caption{Scope Checking Rules}
\label{fig:scope}
\end{figure}







\chapter{Appendix}

\subsection{Terms}

We will call \textsc{ast} terms \emph{expressions} and write them in
s-expression form. Atomic terms are either variables or literals
(i.e. syntactic constants), and compound terms are built with
\emph{term constructors} $P$:

\begin{jtable}
  $e$
  &$::=$& $\lit{lit}$ &(literal) \\
  &$|$&   $\var{x}$ &(variable) \\
  &$|$&   $\expr{P}{e_1 ... e_n}$ &(\textsc{ast} node)
\end{jtable}

\subsection{Tree Grammars}

A \emph{tree grammar} [CITE] is to trees as a context-free grammar is
to strings. Thus it can be viewed either as a set of instructions for
how to iteratively and nondeterministically rewrite a starting
\emph{nonterminal} to a final tree; \emph{or} it can be viewed as a
specification of a grammar that a tree may or may not follow. We will
take the latter view.

Definitionally, a tree grammar consists of a number of
\emph{productions} that map \emph{nonterminals} $s$ to
\name{patterns}:
\begin{jtable}
  $G$
  &$::=$& $\left\{ \begin{array}{l}
    \production{s_1}{\name{pattern}_1} \\
    \ddd \\
    \production{s_n}{\name{pattern}_n}
    \end{array}\right.$ \\
  \\
  $\name{pattern}$
  &$::=$& $\expr{P}{s_1 \dd s_n}$ &(regular pattern) \\
  &$|$&
  $\exprs{P}{s_1 \dd s_n}{s_{n+1}}$
  &(ellipses pattern) \\
  &$|$&   $\constName{literal}$ &(matches literals) \\
  &$|$&   $\constName{var}$ &(matches variables)
\end{jtable}

The \emph{meaning} of a tree grammar (again, we are viewing the
grammar as a \emph{specification}) is that for each production
``$\production{s}{\name{pattern}}$'', if a term matches the
\name{pattern}, then it also matches the nonterminal $s$. Formally:

\[
\fbox{$\saysG{G}{e}{s}$}
\]

\[
\inference
    [G-literal]
    {}
    {\saysG{G}{\lit{lit}}{\constName{literal}}}
\quad
\inference
    [G-variable]
    {}
    {\saysG{G}{\var{x}}{\constName{variable}}}
\]
    
\[
\inference
    [G-node]
    {\Forall{i \in 1..n} \saysG{G}{e_i}{s_i} \\
      \production{s}{\expr{P}{s_1 \dd s_n}} \in G}
    {\saysG{G}{\expr{P}{e_1 \dd e_n}}{s}}
\quad
\inference
    [G-ellipses]
    {\Forall{i \in 1..n} \saysG{G}{e_i}{s_i} \\
      \Forall{j \in 1..k} \saysG{G}{e_{n+j}}{s} \\
      \production{s}{\exprs{P}{s_1 \dd s_n}{s}} \in G}
    {\saysG{G}{\expr{P}{e_1 \dd e_n \dd e_{n+k}}}{s}}
\]

\end{document}
