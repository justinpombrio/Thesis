\chapter{Desugaring in the Wild}\label{chap:taxonomy}

TODO
\begin{itemize}
\item State version of each system discussed
\item mention macro-defining-macros in expressiveness?
\item More examples
\end{itemize}

There are a bewildering variety of desugaring systems.  In this
chapter, we categorize them into a taxonomy.

We stretch this taxonomy to include systems that aren't quite
desugaring systems, but that are closely associated with
them. Notably, we discuss staged metaprogramming systems, which (i)
operate on code at run-time, rather than compile-time, and (ii) have
to explicitly, rather than implicitly, desugar code.

\section{A Sugar Taxonomy}

There are many dimensions by which desugaring mechanisms vary:
\begin{description}
  \item[Representation] Desugaring is a syntax-to-syntax transformation, but
    how is that syntax represented? There is a big difference between
    transformations on the \emph{text} of the program vs. its
    \emph{concrete surface syntax} vs. its \emph{abstract surface
      syntax}.
  \item[Authorship] Are sugars defined by developers of the language (and thus
    relatively fixed), or by users of the language (and thus flexible)?
  \item[Metalanguage] What is the metalanguage? That is, in what language are sugars
    written? Is it the same language the programs are written in, thus
    allowing sugars and code to be interspersed, or a different
    language?
  \item[Desugaring Order] In what order are constructs desugared? Most
    importantly, are nested sugars desugared from the \emph{innermost
      to outermost} (\Sc{io}), or from \emph{outermost to innermost}
    (\Sc{oi})?
  \item[Phase] \emph{When} does desugaring occur? Is it at compilation time,
    or at evaluation time (as in staged metaprogramming systems)? If
    you recall our definition of syntactic sugar qqq
  \item[Expressiveness] How expressive is it? Can a sugar take an
    arbitrary expression as a parameter, or only a primitive type
    (\Sc{Parameter})? If it can take an expression, can it
    deconstruct it, or only use it parametrically (\Sc{Deconstruction})?
    Do sugars exclusively produce expressions, or can they produce any
    kind of syntax (\Sc{Result})?
  \item[Safety] How \emph{safe} is it? Can desugaring produce
    syntactically invalid code (\Sc{Syntax Safety})? Can it produce an
    unbound variable (\Sc{Scope Safety}), or accidentally capture a
    variable (\Sc{Hygiene})? Can it produce code that contains a type
    error (\Sc{Type Safety})?
\end{description}



There are two big clusters of desugaring systems that should be called
out by name: macros and staged metaprogramming. Racket [CITE] is a
prototypical language with macros: they are user-defined, the
metalanguage is either a \Sc{dsl} or the Racket language itself, its
macros are expanded (primarily) during compilation, and the desugaring
order is outside-in.

Staged metaprogramming also involves user-defined sugars, but instead
of desugaring as a separate phase, code is constructed, composed, and
eventually \Code{eval}'ed at runtime. MetaOCaml is a prototypical
metaprogramming language: the metalanguage is MetaOCaml itself,
the desugaring order is inside-out (reflecting the evaluation
order of OCaml), and its sugars are user-defined and are expanded at runtime.
%This thesis is really about sugars that are expanded
%at compile time---so staged metaprogramming systems are out of scope---but
%they are closely enough related that we include them in the taxonomy.


\subsection{Representation}

There are many ways to represent a program. The most prevalent are as
\emph{text} and as a \emph{tree}. Programs are most commonly
\emph{saved as} and \emph{edited as} text (notable exceptions include
visual and block based language/editors), and they are most commonly
\emph{internally represented as} trees (notable exceptions include
assembly, whose code is linear, and Forth, which does not have a
parsing phase).

Desugaring systems may be based on either representation.
\emph{However, text is a terrible representation for desugaring rules.}
The semantics of a language is almost always defined in terms of its
(abstract syntax) tree representation. Thus, insofar as a programmer
as a programmer is forced to think of their program as text rather
than as a tree, they are being distracted from its semantics. There
are well-known examples of bugs that arise in text-based desugaring
rules unless they are written in a very defensive style: we discuss
these in \ref{sec:cpre}.

There are variations among tree representations as well: desugaring
rules may work over the concrete syntax of the language, or over its
abstract syntax. We discuss this further in [REF].

\subsection{Authorship: Language-defined or User-defined}

Sugars may either be specified and implemented as part of the
language, or they may be defined by users. For example, Haskell list
comprehensions are defined by the Haskell spec [CITE] and implemented
in the compiler(s); thus they are language-defined. Template Haskell
sugars [CITE], on the other hand, can be defined (and used) in any
Haskell program; thus they are user-defined.

Language-defined sugars are a convenient method of simplifying
language design, and they are largely invisible: if done correctly,
users of the language should not be able to tell which syntactic
constructs were implemented as sugar and which were built in.
In contrast, user-defined sugars are much more visible. They give
users the power to extend the language. As a consequence, when
this extended language is shared, other users must contend with it,
which may be a gift or a burden, depending on its quality.

[TODO: Consider moving the rest of this prose into a different section.]

When sugars are user-defined, it raises difficulties with tools such
as editors that need to support the new syntax. For example, if sugars
are language-defined then an editor can just support the full
language. If they are user-defined, however, how can an editor provide
correct indentation, syntax highlighting, etc.?

Different languages work around this problem in different ways.

Lisps partly avoid this problem by having two ``layers'' of syntax.
The lower layer is simply s-expressions, and ensures that parentheses
are well-balanced (and that there are no tokenization errors).
The higher layer checks that the s-expression makes syntactic sense.
For example, \Code{(define x 1} is invalid at both layers,
\Code{(define ((x)) 1)} is valid at the first layer but not at the
second, and \Code{(define x 1)} is valid at both layers.\marginpar{%
The phrase \emph{bicameral syntax} was coined by Shriram Krishnamurthi <CITE plai>.
}
This can be called \emph{bicameral syntax}, by analogy to legislative
houses that are split into a lower and upper level.

This separation of concerns makes it easy for editors to support the
\emph{first} layer of syntax, which cannot be modified by syntactic
sugar, while ignoring the second, which can. Doing so only
\emph{partly} avoids the problem. For example, the DrRacket editor for
Racket [CITE] allows indentation schemes to be set on a per-macro
basis (because different syntactic forms, while purely parenthetical,
still have varying nesting patterns that should be indented
differently), and it displays arrows showing where variables are bound
by being macro-aware and expanding the program (see [REF]).

[FILL: SugarJ, other examples]
[FILL: Spoofax, language workbenches: tie editor to language]


\subsection{Metalanguage}

The \emph{metalanguage} is the language that the sugars are written
in. It could be the same language that programs are written in, a
different (but still general purpose) language, or a \Sc{dsl}
especially for sugars (such as the pattern-based
\Code{define-syntax-rule} \Sc{dsl} we've been using for examples).


\subsection{Desugaring Order}
% https://dl.acm.org/citation.cfm?id=1440085
% GRAMMARS WITH MACRO-LIKE PRODUCTIONS (Fischer)

There are two major desugaring strategies used in desugaring systems.
They loosely correspond to eager and lazy runtime evaluation, but
differ in some important ways, so we will instead refer
to them by their original names [CITE]: Outside-in (\Sc{oi}) and
Inside-out (\Sc{io}) desugaring:\marginpar{
  You can also ask whether desugaring proceeds left-to-right or
  right-to-left. However, this is a comparatively minor detail, so we
  ignore it.
}
\begin{description}
\item[\Sc{oi}] desugaring is superficially similar to lazy
  evaluation, in that desugaring proceeds from the outside in.
  However, there is one crucial difference: in lazy evaluation, if a
  function uses (e.g., does case analysis on) one of its (lazy)
  arguments, that forces the evaluation of that argument. In contrast,
  if an \Sc{oi} sugar does case analysis on one of its
  parameters (which are pieces of syntax), it does not force that
  parameter to be desugared! Rather, the case analysis happens on the
  uninterpreted syntax of that parameter.

  There is an advantage to this: it allows sugars to truly extend the
  syntax of the language because they can treat the syntax of their
  parameters however they want.
  In the most extreme case, it may be impossible
  to even tell where the innermost sugars are, so \Sc{oi} is the only
  possible desugaring order.
\item[\Sc{io}] desugaring is similar to eager evaluation: the
  innermost sugars desugar first. The above paragraph suggested that
  \Sc{io} order may not be possible, but there are a number of common
  settings in which it is: (i) the syntax that sugars can introduce is
  limited enough that you can always tell where nested sugars are;
  (iii) sugars act on the \Sc{ast} rather than on concrete syntax; or
  (ii) sugars cannot deconstruct their parameters, so the desugaring
  order is largely irrelevant anyways; . In these settings, \Sc{io}
  order has the advantage of being more analogous to evaluation (which
  developers are quite familiar with). It is
  also sometimes useful for sugars to use the desugared version of
  their parameters: for example, C++ templates can be used to derive
  specialized implementations for particular types, and it is
  convenient to allow that type to be a template expression.
\end{description}


\subsection{Phase}

The \emph{phase} of a desugaring system is \emph{when} desugaring
happens. This is typically at compile-time, but it doesn't have to be.
For example, staged metaprogramming systems construct and evaluate
programs at run-time [TODO: test], and Racket macros can run at an
arbitrary number of different phases [REF].

\subsection{Expressiveness}

We will examine three measures of expressiveness of each desugaring
system:
\begin{description}
\item[Parameters] What kind of parameters can be passed to a sugar? In
  the most general desugaring systems, any syntactic category---be it
  expression, statement, type, field name, class definition,
  etc.---can be passed to a sugar. Some systems are more limited,
  however: for example, C++ templates can only take types or primitive
  values (e.g., numbers) as parameters.
\item[Result] What kind of syntax can a sugar desugar to? Can it be
  any syntactic category, or is it more limited, e.g., to expressions
  only?
\item[Deconstruction] Sugars are given syntax as parameters. Can they
  deconstruct this syntax (e.g., by case analysis) or must
  they treat it parametrically (i.e., only compose it)?

  There is a strong interaction between the ability to deconstruct
  parameters and desugaring order (OI vs. IO). With OI order, the
  parameters that are deconstructed are in the surface language, since
  they haven't been desugared yet. Under IO order, however, the
  parameters are in the core language, because they \emph{have} been
  desugared.
\end{description}
Of course, three measures isn't enough to fully capture how expressive
a desugaring system is! However, it should give a rough idea of the
situations in which it is appropriate to use.


\subsection{Safety}\marginpar{
  Safety and expressiveness are counterpoints. Safety comes from the
  ability of the compiler to tell whether you're doing something
  wrong, and the less expressive the language, the easier it is to
  tell.
}

Last---but certainly not least---we will consider four different
\emph{safety properties} of desugaring systems:
\begin{description}
  \item[Syntactic Safety] Can a sugar expand to syntactically invalid
    code? For example, in the C Preprocessor (\cref{sec:cpre}), this
    (textual) macro:
\begin{Codes}
#define discriminant(a,b,c) ((b) * (b) - (4 * (a) * (c))
\end{Codes}
    is a completely valid macro that the preprocessor will happily
    expand for you, leading to syntax errors at its use site because
    its parentheses aren't balanced.

    Thus we will call the C Preprocessor \emph{syntactically unsafe}.
    In contrast, a syntactically safe desugaring system would raise an
    error on such a sugar definition, \emph{even if that sugar was
      never used}. This is analogous to type checking: a type checker
    will warn that a function definition contains a type error even, if
    if that function is never used.
  \item[Hygiene] Unlike the other safety properties discussed here,
    hygiene [CITE] is not about the error behavior of the desugaring
    system.  Instead, it is about how desugaring treats variables. As
    an example, consider this simple \Code{or} sugar (using Racket
    syntax):\marginpar{ Why not just write \Code{(if a a b)}? That
      wouldn't work well if \Code{a} had side effects.}
\begin{Codes}
(define-syntax-rule
  (or a b)
  (let ((temp a)) (if temp temp b)))
\end{Codes}
    If a desugaring system did nothing special with variables, then
    this code:
\begin{Codes}
(let ((temp "70 degrees"))
  (or false temp))
\end{Codes}
    would desugar into this code:
\begin{Codes}
(let ((temp "70 degrees"))
  (let ((temp false)) (if temp temp temp)))
\end{Codes}\marginpar{
  Some of the examples here were inspired by Clinger and Rees CITE.
}%[CITE]
    which would then evaluate to \Code{false}, which is wrong. The
    issues is that the user-written variable called \Code{temp} is
    captured by the sugar-introduced variable called \Code{temp}.
    Thus this naive desugaring is \emph{unhygienic}.
    
    At its core, hygiene is lexical scoping for sugars: you should be
    able to tell where a user-written variable is bound by looking at
    its code (and in the example, see that the \Code{temp} in \Code{or
      false temp} should be bound by the surrounding \Code{let}), and
    you should be able to tell where a sugar-introduced variable is
    bound by looking at the sugar definition (and there are analogous
    examples where a sugar-introduced variable gets captured by a
    user-written variable). There is more to say about hygiene, and we
    will discuss it further in [REF], but this is sufficient for our
    present taxometric purposes.
    
  \item[Scope Safety] Can a sugar \emph{introduce} scope errors?  We
    will say that a desugaring system is \emph{scope safe} if sugars
    cannot introduce an unbound identifier or cause a user-defined
    variable to become unbound.

  \item[Type Safety] Can a sugar introduce a type error? We will say
    that a desugaring system is \emph{type safe} if sugars cannot
    introduce type errors. For example, this MetaOCaml program (the
    \Code{.< ... >.} syntax quotes an expression):
\begin{Codes}
let sugar() = .<3 + "four">.;;
\end{Codes}
    does not compile because of the type error present in the
    sugar---even though the sugar is never used.
\end{description}

\section{C Preprocessor} \label{sec:cpre}

\Desc{Desugaring Order} IO

\Desc{Authorship} User-defined

\Desc{Representation} Token stream

\Desc{Safety} [FILL]

% https://gcc.gnu.org/onlinedocs/cpp/
% Not Turing complete: https://gcc.gnu.org/onlinedocs/cpp/Self-Referential-Macros.html
% desugaring order: https://gcc.gnu.org/onlinedocs/cpp/Macro-Arguments.html
%   also section 6.10.3.1: http://www.open-std.org/jtc1/sc22/wg14/www/docs/n1124.pdf
\Desc{Discussion}
The C Preprocessor (hereafter \Sc{cpp}) [CITE] is a \emph{text
  preprocessor}: a source-to-source transformation that operates at
the level of text. (More precisely, it operates on a token stream, in
which the tokens are approximately those of the C language). It is
usually run before compilation for C or C++ programs, but it is not
very language specific, and can be used for other purposes as well.
\Sc{Cpp} is not Turing complete, by a simple mechanism: if a macro
invokes itself (directly or indirectly), the recursive invocation
will not be expanded.

%https://stackoverflow.com/questions/14041453/why-are-preprocessor-macros-evil-and-what-are-the-alternatives
A number of issues arise from the fact that \Sc{cpp} operates on tokens, and
is thus unaware of the higher-level syntax of C [CITE].
As an example, consider this innocent looking
\Sc{cpp} desugaring rule that defines an alias for subtraction:
\begin{Codes}
  #define SUB(a, b) a - b
\end{Codes}
This rule is completely broken. Suppose it is used as follows:
\begin{Codes}
  SUB(0, 2 - 1))
\end{Codes}
This will expand to \Code{0 - 2 - 1} and evaluate to \Code{-3}.
We can revise the rule to fix this:
\begin{Codes}
  #define SUB(a, b) (a) - (b)
\end{Codes}
This will fix the last example, but it is still broken. Consider:
\begin{Codes}
  SUB(5, 3) * 2
\end{Codes}
This will expand to \Code{5 - 3 * 2} and evaluate to \Code{-1}.
The rule can be fully fixed by another set of parentheses:
\begin{Codes}
  #define SUB(a, b) ((a) - (b))
\end{Codes}
In general, both the inside boundary of a rule (the parameters \Code{a}
and \Code{b}), and the outside boundary (the whole \Sc{rhs}) need
to be protected to ensure that the expansion is parsed correctly. If
the sugar is used in expression position, as in the \Code{SUB}
example, this can be done with parentheses. In other positions,
different tricks must be used: e.g., a rule meant to be used in
statement position can be wrapped in \Code{do \{...\} while(0)}.
Software developers should not need to know this.

There are other issues that arise with text-based transformations as
well, such as variable capture. Furthermore, all of these issues are
inherent to text-based transformations, and essentially cannot be
fixed from within the paradigm. \emph{Overall, code transformations
  should never operate at the level of text.}

\section{C++ Templates} \label{sec:cpp}

\Desc{Representation} Concrete syntax
\Desc{Authorship} User-defined
\Desc{Metalanguage} C++ templates (pattern based)
\Desc{Parameters} Types and primitive values
\Desc{Result} Function/method definition, struct/class definition, or type alias
\Desc{Deconstruction} Yes (although the parameters are limited)
\Desc{Desugaring Order} IO
\Desc{Phases} One
\Desc{Syntax safe} Yes
\Desc{Scope safe} NA
\Desc{Type safe} NA

% http://www.open-std.org/jtc1/sc22/wg21/docs/papers/2017/n4659.pdf
C++ templates [CITE] are not general-purpose sugars,
because they cannot take code as an parameter. Thus you cannot, for
example, express a sugar that takes an expression \Code{e} and expands
it to \Code{e + 1}.

Instead, C++ templates are used
primarily to instantiate polymorphic code by replacing type parameters
with concrete types.  Let's use the following template declaration,
taken from [CITE: pg344], as a running example. It declares a function
to compute the area of a circle, that can be instantiated with
different possible (presumably numeric) types \Code{T}:
\begin{Codes}
template<class T>
T circular_area(T r) \{
  return pi<T> * r * r;
\}
\end{Codes}

Besides function definitions, several other kinds of declarations can
be templated, including methods, classes, structs, and type aliases.
The behavior of each is similar. A template may be invoked by passing
parameters in angle brackets. An invoked template acts like the
kind of thing the template declared, and can be used in the same
positions. Thus, e.g., a \Code{struct} template should be invoked in type
position; and our running function template example should be invoked
in expression position to make a function, which can then be called:
\begin{Codes}
  float area = circular_area<float>(1);
\end{Codes}
When a template is invoked like this, a copy of the template
definition is made, with the template parameters replaced with the
concrete parameters.\marginpar{
  If a template is invoked multiple times with the same parameters,
  only one copy of the code will be made, however.
}
In our example, this produces the code:
\begin{Codes}
float circular_area(float r) \{
  return pi<float> * r * r;
\}
\end{Codes}

So far we have only described type parameters, but templates can also
take other kinds of parameters, including primitive values (such as
numbers) and other templates. The ability to manipulate numbers and
invoke other templates at compile time make C++ templates powerful
and, unsurprisingly, Turing complete. However, templates \emph{cannot}
be parameterized over code, and thus are not general-purpose sugars.
For example, most of the examples in this thesis cannot be written as
C++ templates.

Template expansion uses IO desugaring order. This is important because
it is possible
to define both a generic template, that applies most of the time, and
a specialized template, that applies if a parameter has a particular
value. For example, this could be used to make a \Code{HashMap} use a different
implementation if its keys are \Code{int}s. Thus it is important that
a template see the concrete type (e.g. \Code{int}) that is passed to
it, even if this type is the result of another template expansion.


\section{Rust Macros} \label{sec:rust}

\Desc{Representation} Concrete Syntax

\Desc{Authorship} User-defined

\Desc{Desugaring Order} OI

\Desc{Safety} [FILL]

%https://doc.rust-lang.org/1.2.0/book/macros.html
\Desc{Discussion}


\section{Haskell Templates} \label{sec:haskell}

\Desc{Representation} Concrete or abstract syntax
\Desc{Authorship} User-defined
\Desc{Metalanguage} Haskell
\Desc{Parameters} Any
\Desc{Result} Any
\Desc{Deconstruction} Yes
\Desc{Desugaring Order} IO
\Desc{Phases} One
\Desc{Syntax Safe} Yes
\Desc{Scope Safe} No
\Desc{Type Safe} No

%https://hackage.haskell.org/package/template-haskell-2.10.0.0/docs/Language-Haskell-TH-Syntax.html#t:Lit
%https://stackoverflow.com/questions/10857030/whats-so-bad-about-template-haskell
NOTES:
\begin{itemize}
  \item Must be defined in separate file
\end{itemize}


\section{MetaOCaml} \label{sec:metaocaml}

\Desc{Representation} Concrete syntax
\Desc{Authorship} User-defined
\Desc{Metalanguage} OCaml
\Desc{Parameters} Expressions (+ OCaml values)
\Desc{Result} Expressions (+ OCaml values)
\Desc{Deconstruction} No
\Desc{Desugaring Order} IO
\Desc{Phases} Many
\Desc{Syntax Safe} Yes
\Desc{Scope Safe} Yes
\Desc{Type Safe} Yes
