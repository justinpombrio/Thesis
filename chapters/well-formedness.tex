\chapter{Well-Formedness}\label{chap:wf}

In this chapter, we describe how to perform well-formedness checking
on \Sc{ast}s, and how to check that desugaring rules preserve
well-formedness.


\section{AST Signatures} % Or Syntype Checking

Language's \textsc{ast}s typically have well-formedness criteria that
restrict how they can be put together. For example, perhaps
expressions can only occur within statements; statements occur only
within blocks; or data definitions can only occur at the top
level. These restrictions are \emph{usually} enforced by parsing,
although this becomes more difficult in the presence of sugar
(e.g. Racket [CITE] [FILL]).

% TODO: Say 'signature', 'arity', 'sort' instead?
These well-formedness conditions can (largely) be captured by giving
each syntactic construct a \emph{syntax type}:

\begin{Table}
\Sc{ast} defn. $G$ &$::=$& $A \mapsto \{t_1, ... t_n\}$
  & (where each $t$ is a syn. con.) \\
syntactic category $A$ \hspace{-1em} &$::=$& \textit{name} \\
syntax type $t$ &$::=$& $A$ & syntactic category \\
  &$|$& $\Core{C}{t_1 ... t_n}$ & core syntactic construct \\
  &$|$& $\Aux{C}{t_1 ... t_n}$  & auxilliary syntactic construct \\
  &$|$& $\Surf{C}{t_1 ... t_n}$ & surface syntactic construct \\
  &$|$& $[t]$ & list \\
  &$|$& $\ValueT$ & primitive value \\
  &$|$& $\DeclT$ & variable declaration \\
  &$|$& $\RefnT$ & variable reference
\end{Table}

For example, if a language has only ``define'' statements, numeric
expressions, and addition expressions, that could be captured by the
\Sc{ast} definition:
\begin{Table}
  $\textit{Stmt}$ &$\mapsto$& $\{\Core{Define}{\DeclT\;\textit{Expr}}\}$ \\
  $\textit{Expr}$ &$\mapsto$& $\{\Core{Number}{\ValueT},$ \\
  && $\Core{Plus}{\textit{Expr}\;\textit{Expr}}\}$
\end{Table}

If parsing the surface language will respect this \Sc{ast} signature
\emph{anyways}, why bother writing it down? First, note that  qqq

Lemma: If rules wf check, then e obeys Surf implies ds(e) obeys
Core.

\paragraph{Exhaustion Checking}
We perform exhaustion checking to make sure that sugars cover all
possible cases of their arguments, but do not give the algorithm here.
It works by looking at \emph{shapes}: a shape is a pattern that
contains types in place of pattern variables. It is straightforward to
check whether an expression matches a shape, and to convert a pattern
into a shape. Exhaustion checking uses the fact that the expressions
that do \emph{not} match a shape can be expressed as a union of shapes.
[TODO: prove]

\begin{figure}

\TypeLabel{\SaysExpr{L}{e}{t}}

\begin{multicols}{2}
  
  \Inference[e-core]{
    A \mapsto \Core{C}{t_1 ... t_n} \in L \\
    \SaysExpr{L}{e_1}{t_1} \;\cdots\; \SaysExpr{L}{e_n}{t_n}
  }{
    \SaysExpr{L}{\Core{C}{e_1 ... e_n}}{A}
  }

  \Inference[e-refn]{}{
    \SaysExpr{L}{\Refn{x}}{\TRefn}
  }

  \Inference[e-decl]{}{
    \SaysExpr{L}{\Decl{x}}{\TDecl}
  }

  \Inference[e-str]{}{
    \SaysExpr{L}{\textit{string}}{\TString}
  }

  \Inference[e-list]{
    \SaysExpr{L}{e_1}{t} \;\cdots\; \SaysExpr{L}{e_n}{t}
  }{
    \SaysExpr{L}{[e_1 ... e_n]}{[t]}
  }

  \Inference[e-sugar]{
    \SaysRule{L}{m}{t_1,...,t_n}{t} \\
    \SaysExpr{L}{e_1}{t_1} \;\cdots\; \SaysExpr{L}{e_n}{t_n}
  }{
    \SaysExpr{L}{\Surf{m}{e_1 ... e_n}}{t}
  }
\end{multicols}
\caption{Well-Formedness Checking: Terms}
\end{figure}

\begin{figure}
\TypeLabel{\SaysPatt{L}{\Gamma}{p}{t}}

\begin{multicols}{2}

  \Inference[p-pvar]{
    \PVarA : t \in \Gamma
  }{
    \SaysPatt{L}{\Gamma}{\PVarA}{t}
  }

  \Inference[p-refn]{}{
    \SaysPatt{L}{\Gamma}{x}{\TRefn}
  }

  \Inference[p-decl]{}{
    \SaysPatt{L}{\Gamma}{x}{\TDecl}
  }

  \Inference[p-str]{}{
    \SaysPatt{L}{\Gamma}{\textit{string}}{\TString}
  }

  \Inference[p-core]{
    A \mapsto \Core{C}{t_1 ... t_n} \in L \\
    \SaysPatt{L}{\Gamma}{p_1}{t_1} \;\cdots\; \SaysPatt{L}{\Gamma}{p_n}{t_n}
  }{
    \SaysPatt{L}{\Gamma}{\Core{C}{p_1 ... p_n}}{A}
  }

  \Inference[p-sugar]{
    \SaysRule{L}{m}{t_1,...,t_n}{t} \\
    \SaysPatt{L}{\Gamma}{p_1}{t_1} \;\cdots\; \SaysPatt{L}{\Gamma}{p_n}{t_n}
  }{
    \SaysPatt{L}{\Gamma}{\Surf{m}{p_1 ... p_n}}{t}
  }

  \Inference[p-empty]{}{
    \SaysPatt{L}{\Gamma}{[\epsilon]}[t]
  }

  \Inference[p-cons]{
    \SaysPatt{L}{\Gamma}{p}{t} \\
    \SaysPatt{L}{\Gamma}{[ps]}{[t]}
  }{
    \SaysPatt{L}{\Gamma}{[\Cons{p}{ps}]}{[t]}
  }

  \Inference[p-ellipsis]{
    l \mapsto \EList{\Gamma'} \in \Gamma & \SaysPatt{L}{\Gamma'}{p}{t}
  }{
    \SaysPatt{L}{\Gamma}{[\Rep{p}{l}]}{[t]}
  }
\end{multicols}

\TypeLabel{\SaysRule{L}{m}{t,...,t}{t}}
\Inference[g-sugar]{
  \DsRuleFancy
      {m}
      {(p_{11},...,p_{1n});\Gamma_1;F_1 \To p_1'}
      {(p_{k1},...,p_{kn});\Gamma_k;F_k \To p_k'}
      \in L \\
  \text{The cases are exhaustive over $t_1,...,t_n$ in $G$} \\
  \SaysPatt{L}{\Gamma_{i}}{p_{ij}}{t_j}
    \text{ for each $i \in 1..k, j \in 1..n$} \\
  \SaysPatt{L}{\Gamma_{i}}{p_i'}{t}
    \text{ for each $i \in 1..k$}
}{
  \SaysRule{L}{m}{t_1,...,t_n}{t}
}

\TypeLabel{\SaysEnv{L}{\gamma}{\Gamma}}
\Inference[$\gamma$-env]{
  \SaysExpr{L}{e_1}{t_1} \;\cdots\; \SaysExpr{e_n}{t_n} \\
  \SaysEnv{L}{\gamma_{11}}{\Gamma_1} \;\cdots\; \SaysEnv{L}{\gamma_{1j}}{\Gamma_1} \\
  ... \\
  \SaysEnv{L}{\gamma_{m1}}{\Gamma_m} \;\cdots\; \SaysEnv{L}{\gamma_{mk}}{\Gamma_m} \\
}{
  \SaysEnv{L}{
    \begin{cases}
      \PVarA_1 \mapsto e_1,\,...,\,\PVarA_n \mapsto e_n \\
      i_1 \mapsto [\gamma_{11},\,...,\,\gamma_{1j}] \\
      ... \\
      i_m \mapsto [\gamma_{m1},\,...,\,\gamma_{mk}]
    \end{cases}
  }{
    \Gamma',
    \begin{cases}
      \PVarA_1: t_1,\,...,\,\PVarA_n: t_n, \\
      %\PVarA_1': t_1',\,...,\,\PVarA_{n'}': t_{n'}' \\
      i_1 \mapsto [\Gamma_1],\,...,\,i_m \mapsto [\Gamma_m], \\
      %i_1' \mapsto [\Gamma_1'],\,...,\,i_{m'}' \mapsto [\Gamma_{m'}']
    \end{cases}
  }
}

\caption{Well-Formedness Checking: Patterns}
\end{figure}
