\chapter{Desugaring, Formally}\label{chap:formalism}

TODO
\begin{itemize}
\item ``We make no assumption that the core language is a subset of the
  surface language, so our results can be applied to, e.g., EDSLs.''
\item Move surf,core,aux and ellipses to impl section
\end{itemize}

In this chapter, we give formal definitions for \Sc{ast}s and
desugaring that will be used throughout the rest of the thesis.
Most relevantly, we assume that desugaring is given (externally to the
language) as a set of pattern-based rewrite rules (a la Racket's \Code{syntax-rules}
[CITE]). This is important because it will form an expressive limit on
resugaring. Our techniques for resugaring evaluation sequences [REF] and
resugaring scope rules [REF] work on the kind of desugaring rules
given here, but no more. The chapter on resugaring type systems [REF]
will further restrict the allowed desuarging rules.

\section{ASTs}\label{sec:formal-term}

For the purposes of resugaring, we require \Sc{ast}s to make three
distinctions explicit.  First, the \Sc{ast} must distinguish between
variable \emph{declarations} $\Decl{x}$ (i.e., binding sites), and
variable \emph{references} $\Refn{x}$ (i.e., uses).  Second, each
variable and node in the \Sc{ast} must have a unique identity $i$ that
distinguishes it from other occurrences of the same variable or node.
Finally, the \Sc{ast} should distinguish between \emph{core} terms,
\emph{surface} terms, and \emph{auxiliary} terms (which we will
describe in [REF]). Exactly why these requirements are necessary will
become clear in [REF] and [REF].

We will refer to \Sc{ast}s (and parts of \Sc{ast}s)
as \emph{terms}, and write them $e$.\marginpar{%
  Referring to terms as $e$ is unfortunate, but we need to
  reserve $t$ to refer to types in REF.
}
With that said, terms can be inductively defined as:
\begin{Table}
constructor $C$ &$::=$& \textit{name} & syntactic construct name \\
term $e$ &$::=$& $\Value$ & primitive value \\
  &$|$& $\Core{C}[i]{e_1 \dd e_n}$ & core \Sc{ast} node \\
  &$|$& $\Surf{C}[i]{e_1 \dd e_n}$ & surface \Sc{ast} node \\
  &$|$& $\Aux{C}[i]{e_1 \dd e_n}$ & auxilliary \Sc{ast} node \\
  &$|$& $[e_1 \dd e_n]$ & list \\
  &$|$& $\Refn[i]{x}$  & variable reference \\
  &$|$& $\Decl[i]{x}$  & variable declaration \\
%  &$|$& $\Tag{p}{p}{e}$ & (described in \cref{chap:resugar-eval})
\end{Table}
Nodes must have fixed arity, so lists are used when a node needs to
contain an arbitrary number of subterms. All of the work in this
thesis will be agnostic to the representation of values, and never
inspect or modify one.


\section{Patterns}

Our desugaring will be based on \emph{patterns}: desugaring will
proceed by \emph{matching} a \Sc{lhs} pattern against a term, and then
\emph{substituting} into the \Sc{rhs} pattern.

Our definition of patterns determines both the expressiveness of the
resulting transformation system and the ability to formally reason about
it. There is a natural trade-off between the two. We pick a definition similar to
that of Scheme \Code{syntax-rules}-style macros, though without guard
expressions.

Formally, our patterns are \emph{regular tree
  expressions}~\cite{regular-tree-expressions}. Regular tree
expressions $\mathit{trx}$ are the natural extension of regular
expressions to handle trees: they add a primitive $\Node{C}{\mathit{trx}_1
\,...\,\mathit{trx}_n}$ for matching a tree node labeled $C$ with
branches matching the regular tree expressions
$\mathit{trx}_1\,...\,\mathit{trx}_n$. Whereas regular tree
expressions conventionally allow choice, we represent choice by
allowing desugaring rules to have multiple cases instead.

Patterns $p$ are defined inductively as:
\begin{Table}
ellipsis label $l$ &$::=$& \textit{name} & ellipsis label \\
pattern $p$ &$::=$& $\PVarA$ & pattern variable \\
  &$|$& $\Core{C}{p_1 ... p_n}$ & core \Sc{ast} node \\
  &$|$& $\Surf{C}{p_1 ... p_n}$ & surface \Sc{ast} node \\
  &$|$& $\Aux{C}{p_1 ... p_n}$ & auxilliary \Sc{ast} node \\
  &$|$& $[ps]$ & list \\
  &$|$& $value$ & primitive value \\
  &$|$& $\Refn[i]{x}$ $|$ $\Decl[i]{x}$  & variable \\
%  &$|$& $\Tag{p}{p}{e}$ & (described in \cref{chap:resugar-eval}) \\
  &$|$& $\Fresh{x}{p}$ & fresh variable [TODO: pick one] \\
seq. pattern $ps$ &$::=$& $\epsilon$ & empty sequence \\
  &$|$& $\Cons{p}{ps}$ & cons \\
  &$|$& $\Rep{p}{l}$ & ellipsis with label $l$ \\
  &$|$& $\EList{p_1 ... p_n}_l$ & ellipsis list (used internally) \\
\end{Table}
In essence, a pattern is a term that can contain pattern variables
$\PVarA$. However, it is very convenient to also allow
\emph{ellipses}, which denote zero or more repetitions of the pattern
they follow. (We formally write ellipses patterns as $\Rep{p}{l}$ to
distinguish them from metasyntactic ellipses).\marginpar{%
  Technically, ellipses do not add any power to pattern-based
  desugaring rules, because they are already Turing complete CITE.
  However, they are very convenient.
}
To simplify the formalism, we assume that every ellipsis is given a
\emph{label} $l$, such that ellipses that share pattern variables have
the same label (and thus must, e.g., be repeated the same number of
times). Our implementation does not require these labels, however.
Instead, it automatically derives them, and raises an error in the
cases where they cannot be sensibly derived.

\section{Matching and Substitution}

[TODO: discuss ellipses lists]
[TODO: make sure not to use $\bowtie$ for match2, as it's used in
  scope resugaring for another purpose.]

Our desugaring transformations are implemented with simpler operations
on patterns: matching and substitution.

\newcommand{\blist}[1]{\lceil #1 \rceil}

\emph{Matching} a term against a pattern induces an \emph{environment}
that binds the pattern's variables. This environment may be
\emph{substituted} into a pattern to produce another term. Formally,
an environment is a map with three kinds of bindings: a pattern
variable $\PVarA$ may be bound to a term $e$, an ellipsis label $l$
may be bound to a sequence of environments (one for each repetition),
and a variable name may be bound to a different variable name
(this will only occur when resugaring evaluation sequences, in [REF]).
\begin{Table}
substitution $\gamma$ &$::=$&
$\begin{cases}
  \PVarA \mapsto e, ... \\
  l \mapsto \EList{\gamma ... \gamma}, ... \\
  x \mapsto x, ...
\end{cases}$
& \begin{tabular}{@{} l @{}}
    regular binding \\ ellipsis binding \\ fresh var binding
  \end{tabular}
\end{Table}


We will write $\SimpleMatch{e}{p}$ to denote matching a term $e$
against a pattern $p$, and write $\SimpleSubs{\gamma}{p}$ to denote
substituting the bindings of an environment $\gamma$ into a pattern
$p$. We will write $\gamma_1 \cup \gamma_2$ for the \emph{disjoint}
union of $\gamma_1$ and $\gamma_2$. The matching and substitution
algorithms are given in \cref{fig:formal-match,fig:formal-substitute}.
[TODO: eliminate $F$?]

\begin{figure}
\TypeLabel{\SaysMatch{F}{e}{p}{\gamma}{p}}
\begin{multicols}{2}
  \Inference[m-pvar]{
    e \text{ is not a \Code{Tag}}
  }{
    \SaysMatch{F}{e}{\PVarA}{\{\PVarA \mapsto e\}}{\PVarA}
  }

  \Inference[m-str]{}{
    \SaysMatch{F}{string}{string}{\{\}}{string}
  }

  \Inference[m-empty]{}{
    \SaysMatch{F}{[\phantom{.}]}{[\epsilon]}{\{\}}{[\epsilon]}
  }

  \Inference[m-cons]{
    \SaysMatch{F}{e_1}{p}{\gamma_1}{p'} \\
    \SaysMatch{F}{[e_2,...,e_n]}{[ps]}{\gamma_s}{[ps']} \\
    \gamma_1 \cup \gamma_2 = \gamma
  }{
    \SaysMatch{F}{[e_1 ... e_n]}{[p,ps]}{\gamma}{[p',ps']}
  }

  \Inference[m-var-capture]{
    x \not\in F
  }{
    \SaysMatch{F}{x}{x}{\{\}}{x}
  }

  \Inference[m-var-fresh]{
    x \in F
  }{
    \SaysMatch{F}{y}{x}{\{x \mapsto y\}}{y}
  }

  \Inference[m-fresh]{
    \SaysMatch{F,x}{e}{p}{\gamma}{p'}
  }{
    \SaysMatch{F}{e}{\Fresh{x}{p}}{\gamma}{\Fresh{x}{p'}}
  }
\end{multicols}
\vspace{1em}

\Inference[m-ellipsis]{
  \SaysMatch{F}{e_1}{p}{\gamma_1}{p_1'} \;\cdots\; \SaysMatch{F}{e_n}{p}{\gamma_n}{p_n'}
}{
  \SaysMatch{F}{[e_1 ... e_n]}{[\Rep{p}{l}]}
            {\{l \mapsto \EList{\gamma_1 ... \gamma_n}\}}
            {\EList{p_1' ... p_n'}_l}
}

\Inference[m-core]{
  \SaysMatch{F}{e_1}{p_1}{\gamma_1}{p_1'} \;\cdots\; \SaysMatch{F}{e_n}{p_n}{\gamma_n}{p_n'} &
}{
  \SaysMatch{F}{\Core{C}{e_1 ... e_n}}{\Core{C}{p_1 ... p_n}}
            {\gamma_1 \cup ... \cup \gamma_n}
            {\Core{C}{p_1' ... p_n'}}
}
  
\Inference[m-tag]{
  \SaysMatch{F}{e}{p}{\gamma}{p'}
}{
  \SaysMatch{F}{\Tag{p_{lhs}}{p_{rhs}}{e}}{p}{\gamma}{\Tag{p_{lhs}}{p_{rhs}}{p'}}
}

\caption{Matching}
\label{fig:formal-match}
\end{figure}


\begin{figure}
\TypeLabel{\SaysSubs{F}{\gamma}{p}{e}}
\begin{multicols}{2}
  \Inference[s-pvar]{
    \PVarA \mapsto e \in \gamma
  }{
    \SaysSubs{F}{\gamma}{\PVarA}{e}
  }

  \Inference[s-str]{}{
    \SaysSubs{F}{\gamma}{string}{string}
  }

  \Inference[s-empty]{}{
    \SaysSubs{F}{\gamma}{[\epsilon]}{[\phantom{.}]}
  }

  \Inference[s-cons]{
    \SaysSubs{F}{\gamma}{p}{e_1} \\
    \SaysSubs{F}{\gamma}{[ps]}{[e_2 ... e_n]}
  }{
    \SaysSubs{F}{\gamma}{[p,ps]}{[e_1 e_2 ... e_n]}
  }

  \Inference[s-var-capture]{
    x \not\in \gamma
  }{
    \SaysSubs{F}{\gamma}{x}{x}
  }

  \Inference[s-var-fresh]{
    x \mapsto y \in \gamma
  }{
    \SaysSubs{F}{\gamma}{x}{y}
  }

  \Inference[s-fresh]{
    \SaysSubs{F}{\gamma,(x \mapsto y)}{p}{e} \quad\text{for fresh $y$}
  }{
    \SaysSubs{F}{\gamma}{\Fresh{x}{p}}{e}
  }
\end{multicols}
\vspace{1em}

\Inference[s-ellipsis]{
  l \mapsto \EList{\gamma_1 ... \gamma_n} \in \gamma \\
  \SaysSubs{F}{\gamma_1}{p}{e_1} \;\cdots\; \SaysSubs{F}{\gamma_n}{p}{e_n}
}{
  \SaysSubs{F}{\gamma}{[\Rep{p}{l}]}{[e_1 ... e_n]}
}

\Inference[s-ellipsis-list]{
  l \mapsto \EList{\gamma_1 ... \gamma_n} \in \gamma \\
  \SaysSubs{F}{\gamma_1}{p_1}{e_1} \;\cdots\; \SaysSubs{F}{\gamma_n}{p_n}{e_n}
}{
  \SaysSubs{F}{\gamma}{\EList{p_1 ... p_n}}{[e_1 ... e_n]}
}

\Inference[s-core]{
  \SaysSubs{F}{\gamma}{p_1}{e_1} \;\cdots\; \SaysSubs{F}{\gamma}{p_n}{e_n}
}{
  \SaysSubs{F}{\gamma}{\Core{C}{p_1 ... p_n}}{\Core{C}{e_1 ... e_n}}
}

\Inference[s-sugar]{
  \SaysSubs{F}{\gamma}{p_1}{e_1} \;\cdots\; \SaysSubs{F}{\gamma}{p_n}{e_n}
}{
  \SaysSubs{F}{\gamma}{\Surf{m}{p_1 ... p_n}}{\Surf{m}{e_1 ... e_n}}
}

\Inference[s-tag]{
  \SaysSubs{F}{\gamma}{p}{e}
}{
  \SaysSubs{F}{\gamma}{\Tag{p_{lhs}}{p_{rhs}}{p}}{\Tag{p_{lhs}}{p_{rhs}}{e}}
}

\caption{Substitution}
\label{fig:formal-substitute}
\end{figure}

It will prove important later that matching and substitution are
(approximately) inverses:
\begin{lemma}[matching and substitution]
  [TODO: Matching and substitution are approximately inverses]
%  $\SaysSubs{F}{\gamma}{p}{e}$, then $\SaysMatch{F}{e}{p}{\gamma}$.
\end{lemma}
\begin{proof}
  Induct on $p$.
  [FILL]
\end{proof}
%However, the reverse is not true. Matching does not undo substitution,
%because substitution in non-deterministic (because it generates fresh
%variables).

\section{Desugaring}\label{sec:formal-desugar}

See \cref{fig:formal-expansion}.
[TODO: Replace step with something that looks like desugaring.]
[TODO: Replace $v$ with something that looks like core terms.]

\begin{Table}
fresh vars $F$ &$::=$& $\{x,...\}$ & fresh variable set [TODO: pick one] \\
rewrite case $c$ &$::=$&
  $\DsRuleCase{(p_1,\,...,\,p_k)}{\Gamma}{p'}$ \\
desugaring rule $r$ &$::=$&
  $\DsRule{C}{k_1,...,k_n}$ \\
desugaring rules $rs$ &$::=$& $\{r_1, ..., r_n\}$
\end{Table}
[TODO: note that gamma described later.]

\subsection{Restrictions on Desugaring}\label{sec:formal-reqs}

Not every syntactically valid desugaring rule is semantically
sensible, and not every semantically sensible desugaring rule is
feasible to resugar. Here we give (i) a set of well-formedness
criteria on desugaring rules, without which they don't make semantic
sense, and (ii) a set of restrictions on desugaring rules that we need
to effectively resugar them. The resugaring techniques described in
[REF] will rely on these restrictions.

\paragraph{Well-formedness Criteria for Desugaring Rules}
\begin{enumerate}
\item \emph{Each pattern variable in the \Sc{rhs} also appears in the
  \Sc{lhs}.} Otherwise the pattern variable would be unbound during
  expansion.
\item \emph{An ellipsis of depth $n$ must contain at least one pattern
  variable that appears at depth $n$ on the other side of the rule.}
  Otherwise it is impossible to know how many times to repeat its
  pattern during substitution. (The \emph{depth} of an ellipsis
  measures how deeply nested it is within other ellipses; a top-level
  ellipsis has depth 1, an ellipsis within an ellipsis depth 2, and so
  forth.)
\item $p$ contains no references or declarations. Rather, these
  should be contained in its pattern variables during expansion.
\end{enumerate}

\paragraph{Restrictions on Desugaring Rules}
\begin{enumerate}
\item \emph{Each patern variable appears exactly once in the \Sc{lhs}
  and exactly once in the \Sc{rhs}.}
  Allowing duplicate pattern variables complicates matching,
  unification, and proofs of correctness. It also copies code and, in
  the worst case, can exponentially blow up programs.  We therefore
  disallow duplication, with the sole exception of pattern variables
  bound to atomic terms. [TODO: impl]
  
  Likewise, dropping pattern variables complicates running desugaring
  rules in reverse, which we will need to do in [REF]. (Naively, when
  running a desugaring rule in reverse, a pattern variable that
  appeared on the \Sc{lhs} but not on the \Sc{rhs} would be unbound.)
\item \emph{Each transformation's \Sc{lhs} must have the form
  $\Surf{C}{e_1,\,...,\,e_n}$.} We will rely on this fact when showing that
  unexpansion is an inverse of expansion in \cref{sec:reval-inverses}.
\item References and declarations in the \Sc{rhs} are given fresh names during
  expansion to ensure hygiene.
\end{enumerate}

In addition, we have one extra restriction for scope resugaring, and
one for type resugaring.
When resugaring \emph{scope rules} [REF], desugaring rules may not
contain ellipses. [TODO: lift this restriction!]
When resugaring \emph{type rules} [REF], we will also
require that for each desugaring rule, its cases be disjoint.

[FILL: desugaring]\marginpar{
  Formally, this desugaring definition is a term rewriting system (\Sc{trs})~\cite{trs}. We call
  the \Sc{trs}'s variables \emph{pattern variables} to avoid confusing them
  with references and declarations, which are constants from the
  perspective of the TRS.}

\begin{figure}
  NOTES: Core terms are terms without Surf or Aux nodes. Desugaring
  takes a term and produces a core term. Each sugar lhs and rhs is a
  surface pattern. Pattern variables match against core terms.
  \begin{Table}
    $c$ &$::=$& $\Value$ \\
    &$|$& $\Refn{x}$ \\
    &$|$& $\Decl{x}$ \\
    &$|$& $\Core{C}{c \dd}$ \\
    &$|$& $\Tag{p}{p}{c}$ \\
    &$|$& $[c \dd]$ \\
    \\
    $E_{ds}$ &$::=$& $\square$ \\
    &$|$& $[c \dd E_{ds}\,e \dd]$ \\
    &$|$& $\Core{C}{c \dd E_{ds}\,e \dd}$ \\
    &$|$& $\Surf{C}{c \dd E_{ds}\,e \dd}$ \\
    &$|$& $\Aux{C}{c \dd E_{ds}\,e \dd}$ \\
    &$|$& $\Tag{p}{p}{E_{ds}}$
  \end{Table}

  \begin{multicols}{2}
  \TypeLabel{\SaysDesugar{L}{s}{c}}

  \Inference[desugar]{
    \SaysDss{L}{s}{c}
  }{
    \SaysDesugar{L}{s}{c}
  }

  \TypeLabel{\SaysDs{L}{e}{e}} 

  \Inference[ds-ctx]{
    \SaysExp{L}{e}{e'}
  }{
    \SaysDs{L}{E_{ds}[e]}{E_{ds}[e']}
  }
  \end{multicols}

  \TypeLabel{\SaysExp{L}{e}{e}}
  
  \Inference[exp-sugar]{
    L = G,rs &
    \DsRule{C}{k_1 ... k_n} \in rs \IS\\
    k_i = \DsRuleCase{(p_1,...,p_n)}{\Gamma}{F}{p_{rhs}} \IS\\
    \SaysMatch{L}{c_i}{p_i}{\gamma_i}{p_i'} \text{ for each $i$} \IS\\
    \SaysMatch{L}{\Surf{C}{c_1 ... c_n}}{\Surf{C}{p_1 ... p_n}}{\gamma}{p_{lhs}} \IS\\
    \gamma' \text{ gives fresh names to the variables in $F$} \IS\\
%    \gamma_1 \cup ... \cup \gamma_n \cup \gamma' = \gamma \IS\\
    \SaysSubs{F}{(\gamma \cup \gamma')}{p_{rhs}}{e'}
  }{
    \SaysExp{L}{\Surf{C}{c_1 ... c_n}}{\Tag{p_{lhs}}{p_{rhs}}{e'}}
  }
  
  \Inference[exp-core]{
    L = G,rs &
    C \not\in rs &
    p = \Surf{C}{\PVarA_1 ... \PVarA_n} &
    p' = \Core{C}{\PVarA_1 ... \PVarA_n}
  }{
    \SaysExp{L}{\Surf{C}{c_1 ... c_n}}{\Tag{p}{p'}{\Core{C}{c_1 ... c_n}}}
  }
  \caption{Desugaring}
  \label{fig:formal-expansion}
\end{figure}



\newpage
\section{Example}

\subsection{Define-struct}

\paragraph{Core AST}
\begin{Codes}
Stmts:
| [\{splicing-begin stmts:Stmts\} @rest:Stmts]
   binding stmts in rest
   providing stmts, rest

| [\{let x:Var v:Expr\} @rest:Stmts]
   binding x in rest

| [\{fun f:Var args:Args body:Expr\} @rest:Stmts]
   binding args in body, rest in body
   providing f, rest
\end{Codes}

%% ALTERNATIVELY:

%% Stmts:
%% | \{splicing-begin stmts:Stmts rest:Stmts\}
%%    binding stmts in rest
%%    providing stmts, rest

%% | \{let x:Var v:Expr rest:Stmts\}
%%    binding x in rest

%% | \{fun f:Var args:Args body:Expr rest:Stmts\}
%%    binding args in body, rest in body
%%    providing f, rest

%% | \{end\}

%% Params:
%% | \{param x:Var rest:Params\}
%% | \{end\}

\paragraph{Auxiliary AST}
\begin{Codes}
IStructFields:
| [field:IStructField ...fields:IStructFields]
  providing field, fields

IStructField:
| \{i-struct-field field:Str get:Var set:Var\}
  providing get, set
\end{Codes}

\paragraph{Surface AST}
\begin{Codes}
SurfStmts:
  .....
| [(define-struct name:Var fields:StructFields) @rest:SurfStmts]
  binding name in rest, fields in rest
  providing name, fields, rest

StructFields:
| [field:StructField ...fields:StructFields]
  providing field, fields

StructField:
| (struct-field field:Str get:Var set:Var)
  providing get, set
\end{Codes}

\paragraph{Desugaring Rules}
\begin{Codes}
   [(struct-field field:Str get:Var set:Var) @rest:IStructFields]
=> [\{i-struct-field field get set\} @rest]
  
   [(define-struct name:Var
      [(struct-field field:Str get:Var set:Var) ...]) @rest:SurfStmts]
=> [\{fun name [x ...] \{record [\{record-field field x\} ...]\}\}
    \{splicing-begin [\{fun get [rec] \{record-get rec field\}\} ...]\}
    \{splicing-begin [\{fun set [rec val] \{record-set rec field val\}\} ...]\}
    @rest]
\end{Codes}

\subsection{Pyret For Expressions}

To handle Pyret for-expressions, we need to add desugaring rules for
the \Code{for} and for its bindings. 
First, when a for-expression binding (e.g. \Code{n from 0}) desugars,
it will simply return its binding (\Code{n}) and its value (\Code{0}).
It can do so with the desugaring rule:
\begin{Codes}
   (s-for-bind l:Loc b:Bind v:Expr)
=> \{for-bind b v\}
\end{Codes}
where \Code{ForBind} is a new auxiliary type:
\begin{Codes}
ForBind += \{for-bind Bind Expr\}

with list scope:
  [\{for-bind b v\} bs ...]
  export b
  export bs
\end{Codes}

Then the surface grammar can be extended for \Code{for} expressions:
\begin{Codes}
Expr += (s-for Loc Expr [ForBind] Ann Expr Bool)
with scope:
  bind binds in body
\end{Codes}

Then for-expressions can be implemented with the desugaring rule:
\begin{Codes}
   (s-for l:Loc
          iter:Expr
          [\{for-bind bind:Bind value:Expr\} ...]@binds
          ann:Ann
          body:Expr
          blocky:Bool)
=> (Apply l iter
     [(Lambda l (CONCAT "for-body<" (FORMAT l false) ">")
        [] [bind ...] ann "" body None None blocky)
      value ...])
\end{Codes}

Notice that \Code{s-for} is pattern matching against the results of
desugaring the \Code{s-for-bind}s. The \Code{(CONCAT ...)} stuff is to
compute at compile time a name for this lambda, which is what Pyret
currently does.

\newcommand{\C}{\(\sb{c}\)}
\subsubsection{Desugaring Pyret For Expressions}
\begin{Codes}
   (s-for map
     [(s-for-bind x (list [(+ 1 2)]))]
     (* x x))
=> (s-for map [(s-for-bind x (list
     [(Tag<(+ a b),(+\C a b)> (+\C 1 2))]))] (* x x))
=> (s-for map [(s-for-bind x (Tag<(list a),(list\C a)> (list\C
     [(Tag<(+ a b),(+\C a b)> (+\C 1 2))])))] (* x x))
=> (s-for map
     [(Tag<(s-for-bind b v),\{for-bind b v\}> \{for-bind x
       (Tag<(list a),(list\C a)> (list\C
         [(Tag<(+ a b),(+\C a b)> (+\C 1 2))]))\})] (* x x))
=> (s-for map
     [(Tag<(s-for-bind b v),\{for-bind b v\}> \{for-bind x
       (Tag<(list a),(list\C a)> (list\C
         [(Tag<(+ a b),(+\C a b)> (+\C 1 2))]))\})] (* x x))
=> (s-for map
     [(Tag<(s-for-bind b v),\{for-bind b v\}> \{for-bind x
       (Tag<(list a),(list\C a)> (list\C
         [(Tag<(+ a b),(+\C a b)> (+\C 1 2))]))\})]
     (Tag<(* a b),(*\C a b)>
       (*\C x x)))
=> (Tag<(s-for iter
         [(Tag<(s-for-bind b v),\{for-bind b v\}>
             \{for-bind bind value\}) ...]
           body),
         (apply iter [(lambda [bind ...] body) value ...])>
      (apply map [(lambda [x] (Tag<(* a b),(*\C a b)> (*\C x x)))
        (Tag<(list a),(list\C a)> (list\C
          [(Tag<(+ a b),(+\C a b)> (+\C 1 2))]))]))
=> (Tag<(s-for iter
         [(Tag<(s-for-bind b v),\{for-bind b v\}>
             \{for-bind bind value\}) ...]
           body),
         (apply iter [(lambda [bind ...] body) value ...])>
      (apply map [(Tag<..,..> (lambda\C [x]
          (Tag<(* a b),(*\C a b)> (*\C x x))))
        (Tag<(list a),(list\C a)> (list\C
          [(Tag<(+ a b),(+\C a b)> (+\C 1 2))]))]))
=> (Tag<(s-for iter
         [(Tag<(s-for-bind b v),\{for-bind b v\}>
             \{for-bind bind value\}) ...]
           body),
         (apply iter [(lambda [bind ...] body) value ...])>
      (Tag<..,..> (apply\C map [(Tag<..,..>(lambda\C [x]
          (Tag<(* a b),(*\C a b)> (*\C x x))))
        (Tag<(list a),(list\C a)> (list\C
          [(Tag<(+ a b),(+\C a b)> (+\C 1 2))]))])))
-> (Tag<(s-for iter
         [(Tag<(s-for-bind b v),\{for-bind b v\}>
             \{for-bind bind value\}) ...]
           body),
         (apply iter [(lambda [bind ...] body) value ...])>
      (Tag<..,..> (apply\C map [(Tag<..,..> (lambda\C [x]
           (Tag<(* a b),(*\C a b)> (*\C x x))))
        (Tag<(list a),(list\C a)> (list\C [3]))])))
<= (Tag<(s-for iter
         [(Tag<(s-for-bind b v),\{for-bind b v\}>
             \{for-bind bind value\}) ...]
           body),
         (apply iter [(lambda [bind ...] body) value ...])>
      (Tag<..,..> (apply\C map [(Tag<..,..> (lambda\C [x] (* x x)))
        (Tag<(list a),(list\C a)> (list\C [3]))])))
<= (Tag<(s-for iter
         [(Tag<(s-for-bind b v),\{for-bind b v\}>
             \{for-bind bind value\}) ...]
           body),
         (apply iter [(lambda [bind ...] body) value ...])>
      (Tag<..,..> (apply\C map [(lambda [x] (* x x))
        (list [3])])))
<= (Tag<(s-for iter
         [(Tag<(s-for-bind b v),\{for-bind b v\}>
             \{for-bind bind value\}) ...]
           body),
         (apply iter [(lambda [bind ...] body) value ...])>
      (apply map [(lambda [x] (* x x)) (list [3])]))
<= (Tag<(s-for iter
         [(Tag<(s-for-bind b v),\{for-bind b v\}>
             \{for-bind bind value\}) ...]
           body),
         (apply iter [(lambda [bind ...] body) value ...])>
      (apply map [(lambda [x] (* x x)) (list [3])]))
  // iter=map, bind=[x], body=(* x x), value=[(list [3])]
<= (s-for map
     [(Tag<(s-for-bind b v),\{for-bind b v\}>
        \{for-bind x (list [3])\})]
     (* x x))
<= (s-for map
     [(s-for-bind x (list [3]))]
     (* x x))
\end{Codes}

