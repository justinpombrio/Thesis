\chapter{Desugaring, Formally}\label{chap:formalism}

In this chapter, we give formal definitions for \Sc{ast}s and
desugaring that will be used throughout the rest of the thesis.  Most
relevantly, we assume that desugaring is given as a set of
pattern-based rewrite rules (a la Racket's \Code{syntax-rules}
[CITE]). This is important because it will form an expressive limit on
resugaring. Our techniques for resugaring evaluation sequences [REF] and
resugaring scope rules [REF] work on the kind of desugaring rules
given here, but no more. The chapter on resugaring type systems [REF]
will further restrict the allowed desuarging rules.

\section{ASTs}

For the purposes of resugaring, we place two requirements on
\textsc{ast}s. First, the \Sc{ast} must distinguish between variable
\emph{declarations} $\Decl{x}$ (i.e., binding sites), and variable
\emph{references} $\Refn{x}$ (i.e., uses). Second, each variable and
node in the \Sc{ast} must have a unique identity $u$ that
distinguishes it from other occurrences of the same variable or node.
(Exactly why these requirements are necessary will become clear in
[REF] and [REF].)
[TODO: distinguish core, aux, surf]
With this taken into account, the definition of \textsc{ast}s is
straightforward:
\marginpar{
  ~\\\vspace{1em}
  Why are lists necessary---can't they just be represented by
  \Sc{ast} nodes? We will shortly (REF) require \Sc{ast} nodes to have
  fixed arity.
}
\begin{Table}
constructor $C$ &$::=$& \textit{name} & syntactic construct name \\
ast $e$ &$::=$& $\Value$ & primitive value \\
  &$|$& $\Core{C}[u]{e_1 \dd e_n}$ & core \Sc{ast} node \\
  &$|$& $\Surf{C}[u]{e_1 \dd e_n}$ & surface \Sc{ast} node \\
  &$|$& $\Aux{C}[u]{e_1 \dd e_n}$ & auxilliary \Sc{ast} node \\
%  &$|$& $\Bitter{C}{e_1 \dd e_n}$ & bitter \Sc{ast} node \\
  &$|$& $[e_1 \dd e_n]$ & list \\
  &$|$& $\Refn[u]{x}$  & variable reference \\
  &$|$& $\Decl[u]{x}$  & variable declaration \\
%  &$|$& $\Tag{p}{p}{e}$ & (described in \cref{chap:resugar-eval})
%value $v$ &$::=$& $e$ & with no sugar invocations \\
\end{Table}


\section{Patterns}

We represent the \Sc{lhs}s and \Sc{rhs}s of desugaring rules uniformly
as \emph{patterns}. Patterns $p$ are defined inductively as:
\begin{Table}
ellipsis label $l$ &$::=$& \textit{name} & ellipsis label \\
%shape $\dot{e}$ &$::=$& ...e... & (same cases as $e$) \\
%  &$|$& $\bullet$ & hole \\
pattern $p$ &$::=$& $\PVarA$ & pattern variable \\
  &$|$& $\Core{C}{p_1 ... p_n}$ & core \Sc{ast} node \\
  &$|$& $\Surf{m}{p_1 ... p_n}$ & surface \Sc{ast} node \\
  &$|$& $\Aux{m}{p_1 ... p_n}$ & auxilliary \Sc{ast} node \\
  &$|$& $[ps]$ & list \\
  &$|$& $value$ & primitive value \\
  &$|$& $\Refn[i]{x}$ $|$ $\Decl[i]{x}$  & variable \\
  &$|$& $\Tag{p}{p}{e}$ & (used internally) \\
  &$|$& $\Fresh{x}{p}$ & fresh variable \\
seq. pattern $ps$ &$::=$& $\epsilon$ & empty sequence \\
  &$|$& $\Cons{p}{ps}$ & cons \\
  &$|$& $\Rep{p}{l}$ & ellipsis with label $l$ \\
  &$|$& $\EList{p_1 ... p_n}_l$ & ellipsis list (used internally) \\
fresh vars $F$ &$::=$& $\{x,...\}$ & fresh variable set \\
type env. $\Gamma$ &$::=$&
$\begin{cases}
  \PVarA:t, ... \\
  i \mapsto [\Gamma], ...
\end{cases}$ \\
substitution $\gamma$ &$::=$&
$\begin{cases}
  \PVarA \mapsto e, ... \\
  l \mapsto \EList{\gamma ... \gamma}, ... \\
  x \mapsto x, ...
\end{cases}$
\end{Table}

[TODO: The rest of the prose in this section is out of date]
Variables are denoted by a lowercase
identifier, labeled nodes are denoted by an uppercase identifier followed
by a parenthesized list of subpatterns, and lists are denoted by a
parenthesized list of subpatterns. Nodes must have fixed arity, so lists are
used when a node needs to contain an arbitrary number of subterms.
Ellipses (which we write formally as $\ddd{\bullet}$ to distinguish them
from metasyntactic ellipses) in a list pattern denote zero or more
repetitions of the pattern
they follow. A \emph{term} $T$ is simply a pattern without variables or
ellipses. Tags and origins ($O$) are described in \cref{sec:ds-tagging}.
We do \emph{not} address hygiene in our system. We believe it is largely
orthogonal to the problem at hand, and that our transformation system
could be made hygienic without significant alterations.

Our definition of patterns determines both the expressiveness of the
resulting transformation system and the ability to formally reason about
it. There is a natural trade-off between the two. We pick a definition similar to
that of Scheme \Code{syntax-rules}-style macros, though without guard
expressions.

Formally, our patterns are
\emph{regular tree expressions}~\cite{regular-tree-expressions}.  Regular
tree expressions $\mathit{trx}$ are the
natural extension of regular expressions to handle trees: they add a
primitive $l;\mathit{trx}_1 \,...\,\mathit{trx}_n$ for matching a tree node
labeled $l$ with branches matching the regular tree expressions
$\mathit{trx}_1\,...\,\mathit{trx}_n$. Whereas regular tree
expressions conventionally allow choice, we encode it using multiple
rules, making the pattern language simpler.



\begin{Table}
rewrite case $c$ &$::=$&
  $\DsRuleCase{(p_1,\,...,\,p_k)}{\Gamma}{p'}$ \\
desugaring rule $r$ &$::=$&
  $\DsRule{C}{k_1,...,k_n}$ \\
desugaring rules $rs$ &$::=$& $\{r_1, ..., r_n\}$
\end{Table}



\subsection{Matching and Substitution}

Our desugaring transformations are implemented with simpler operations on patterns:
matching and substitution.

\newcommand{\blist}[1]{\lceil #1 \rceil}

\emph{Matching} a term against a pattern induces an \emph{environment}
that binds the pattern's variables. This environment may be
\emph{substituted} into a pattern to produce another term. Formally, an
environment is a mapping from pattern variables $\PVarA$ to bindings $b$[TODO],
where each \emph{binding} is either a term $e$, a \emph{list binding}
$\blist{b_1...b_n}$, or an \emph{ellipsis binding}
$\blist{b_1...b_n\ \ddd{b_e}}$. A pattern variable within ellipses is
bound to a list binding $\blist{b_1...b_n}$ instead of a list term
$[b_1...b_n]$; they behave slightly differently under
substitution. Ellipsis bindings are similar, but needed only during
unification when a variable within an ellipsis is itself bound to an
ellipsis pattern.

We will write $\SimpleMatch{e}{p}$ to denote matching a term $e$ against a pattern
$p$, and write $\SimpleSubs{\gamma}{p}$ to denote substituting the bindings of an
environment $\gamma$ into a pattern $p$.
We will write $\gamma_1 \cup \gamma_2$ for the
\emph{right-biased} union of $\gamma_1$ and $\gamma_2$. The matching and
substitution
algorithms are given in \cref{fig:subs}, while bindings are defined in
\cref{fig:bind}.


\begin{figure}
\TypeLabel{\SaysMatch{F}{e}{p}{\gamma}{p}}
\begin{multicols}{2}
  \Inference[m-pvar]{
    e \text{ is not a \Code{Tag}}
  }{
    \SaysMatch{F}{e}{\PVarA}{\{\PVarA \mapsto e\}}{\PVarA}
  }

  \Inference[m-str]{}{
    \SaysMatch{F}{string}{string}{\{\}}{string}
  }

  \Inference[m-empty]{}{
    \SaysMatch{F}{[\phantom{.}]}{[\epsilon]}{\{\}}{[\epsilon]}
  }

  \Inference[m-cons]{
    \SaysMatch{F}{e_1}{p}{\gamma_1}{p'} \\
    \SaysMatch{F}{[e_2,...,e_n]}{[ps]}{\gamma_s}{[ps']} \\
    \gamma_1 \cup \gamma_2 = \gamma
  }{
    \SaysMatch{F}{[e_1 ... e_n]}{[p,ps]}{\gamma}{[p',ps']}
  }

  \Inference[m-var-capture]{
    x \not\in F
  }{
    \SaysMatch{F}{x}{x}{\{\}}{x}
  }

  \Inference[m-var-fresh]{
    x \in F
  }{
    \SaysMatch{F}{y}{x}{\{x \mapsto y\}}{y}
  }

  \Inference[m-fresh]{
    \SaysMatch{F,x}{e}{p}{\gamma}{p'}
  }{
    \SaysMatch{F}{e}{\Fresh{x}{p}}{\gamma}{\Fresh{x}{p'}}
  }
\end{multicols}
\vspace{1em}

\Inference[m-ellipsis]{
  \SaysMatch{F}{e_1}{p}{\gamma_1}{p_1'} \;\cdots\; \SaysMatch{F}{e_n}{p}{\gamma_n}{p_n'}
}{
  \SaysMatch{F}{[e_1 ... e_n]}{[\Rep{p}{l}]}
            {\{l \mapsto \EList{\gamma_1 ... \gamma_n}\}}
            {\EList{p_1' ... p_n'}_l}
}

\Inference[m-core]{
  \SaysMatch{F}{e_1}{p_1}{\gamma_1}{p_1'} \;\cdots\; \SaysMatch{F}{e_n}{p_n}{\gamma_n}{p_n'} &
}{
  \SaysMatch{F}{\Core{C}{e_1 ... e_n}}{\Core{C}{p_1 ... p_n}}
            {\gamma_1 \cup ... \cup \gamma_n}
            {\Core{C}{p_1' ... p_n'}}
}
  
\Inference[m-tag]{
  \SaysMatch{F}{e}{p}{\gamma}{p'}
}{
  \SaysMatch{F}{\Tag{p_{lhs}}{p_{rhs}}{e}}{p}{\gamma}{\Tag{p_{lhs}}{p_{rhs}}{p'}}
}

\caption{Matching}
\end{figure}


\begin{figure}
\TypeLabel{\SaysSubs{F}{\gamma}{p}{e}}
\begin{multicols}{2}
  \Inference[s-pvar]{
    \PVarA \mapsto e \in \gamma
  }{
    \SaysSubs{F}{\gamma}{\PVarA}{e}
  }

  \Inference[s-str]{}{
    \SaysSubs{F}{\gamma}{string}{string}
  }

  \Inference[s-empty]{}{
    \SaysSubs{F}{\gamma}{[\epsilon]}{[\phantom{.}]}
  }

  \Inference[s-cons]{
    \SaysSubs{F}{\gamma}{p}{e_1} \\
    \SaysSubs{F}{\gamma}{[ps]}{[e_2 ... e_n]}
  }{
    \SaysSubs{F}{\gamma}{[p,ps]}{[e_1 e_2 ... e_n]}
  }

  \Inference[s-var-capture]{
    x \not\in \gamma
  }{
    \SaysSubs{F}{\gamma}{x}{x}
  }

  \Inference[s-var-fresh]{
    x \mapsto y \in \gamma
  }{
    \SaysSubs{F}{\gamma}{x}{y}
  }

  \Inference[s-fresh]{
    \SaysSubs{F}{\gamma,x \mapsto y}{p}{e} \text{ for fresh $y$}
  }{
    \SaysSubs{F}{\gamma}{\Fresh{x}{p}}{e}
  }
\end{multicols}
\vspace{1em}

\Inference[s-ellipsis]{
  l \mapsto \EList{\gamma_1 ... \gamma_n} \in \gamma \\
  \SaysSubs{F}{\gamma_1}{p}{e_1} \;\cdots\; \SaysSubs{F}{\gamma_n}{p}{e_n}
}{
  \SaysSubs{F}{\gamma}{[\Rep{p}{l}]}{[e_1 ... e_n]}
}

\Inference[s-ellipsis-list]{
  l \mapsto \EList{\gamma_1 ... \gamma_n} \in \gamma \\
  \SaysSubs{F}{\gamma_1}{p_1}{e_1} \;\cdots\; \SaysSubs{F}{\gamma_n}{p_n}{e_n}
}{
  \SaysSubs{F}{\gamma}{\EList{p_1 ... p_n}}{[e_1 ... e_n]}
}

\Inference[s-core]{
  \SaysSubs{F}{\gamma}{p_1}{e_1} \;\cdots\; \SaysSubs{F}{\gamma}{p_n}{e_n}
}{
  \SaysSubs{F}{\gamma}{\Core{C}{p_1 ... p_n}}{\Core{C}{e_1 ... e_n}}
}

\Inference[s-sugar]{
  \SaysSubs{F}{\gamma}{p_1}{e_1} \;\cdots\; \SaysSubs{F}{\gamma}{p_n}{e_n}
}{
  \SaysSubs{F}{\gamma}{\Surf{m}{p_1 ... p_n}}{\Surf{m}{e_1 ... e_n}}
}

\Inference[s-tag]{
  \SaysSubs{F}{\gamma}{p}{e}
}{
  \SaysSubs{F}{\gamma}{\Tag{p_{lhs}}{p_{rhs}}{p}}{\Tag{p_{lhs}}{p_{rhs}}{e}}
}

\caption{Substitution}
\end{figure}

\begin{lemma}[matching and substitution]
  Matching and substitution are inverses:
%  $\SaysSubs{F}{\gamma}{p}{e}$, then $\SaysMatch{F}{e}{p}{\gamma}$.
\end{lemma}
\begin{proof}
  Induct on $p$.
  [FILL]
\end{proof}
%However, the reverse is not true. Matching does not undo substitution,
%because substitution in non-deterministic (because it generates fresh
%variables).

\subsection{Expansion}

See \cref{fig:expansion}.
[TODO: Replace step with something that looks like desugaring.]
[TODO: Replace $v$ with something that looks like core terms.]

\subsection{Restrictions on Desugaring}

Not every syntactically valid desugaring rule is semantically
sensible, and not every semantically sensible desugaring rule is
feasible to resugar. Here we give (i) a set of well-formedness
criteria on desugaring rules, without which they don't make semantic
sense, and (ii) a set of restrictions on desugaring rules that we need
to effectively resugar them. The resugaring techniques described in
[REF] will rely on these restrictions.

\paragraph{Well-formedness Criteria for Desugaring Rules}
\begin{enumerate}
\item \emph{Each pattern variable in the \Sc{rhs} also appears in the
  \Sc{lhs}.} Otherwise the pattern variable would be unbound during
  expansion.
\item \emph{An ellipsis of depth $n$ must contain at least one pattern
  variable that appears at depth $n$ on the other side of the rule.}
  Otherwise it is impossible to know how many times to repeat its
  pattern during substitution. (The \emph{depth} of an ellipsis
  measures how deeply nested it is within other ellipses; a top-level
  ellipsis has depth 1, an ellipsis within an ellipsis depth 2, and so
  forth.)
\end{enumerate}

\paragraph{Restrictions on Desugaring Rules}
\begin{enumerate}
\item \emph{Each patern variable appears exactly once in the \Sc{lhs}
  and exactly once in the \Sc{rhs}.}
  Allowing duplicate pattern variables complicates matching,
  unification, and proofs of correctness. It also copies code and, in
  the worst case, can exponentially blow up programs.  We therefore
  disallow duplication, with the sole exception of pattern variables
  bound to atomic terms. [TODO: impl]
  
  Likewise, dropping pattern variables complicates running desugaring
  rules in reverse, which we will need to do in [REF]. (Naively, when
  running a desugaring rule in reverse, a pattern variable that
  appeared on the \Sc{lhs} but not on the \Sc{rhs} would be unbound.)
\item \emph{Each transformation's \Sc{lhs} must have the form
  $\Surf{C}{e_1,\,...,\,e_n}$.} We will rely on this fact when showing that
  unexpansion is an inverse of expansion in \cref{sec:reval-inverses}.
\end{enumerate}

In addition, we have one extra restriction for scope resugaring, and
one for type resugaring.
When resugaring \emph{scope rules} [REF], desugaring rules may not
contain ellipses. [TODO: lift this restriction!]
When resugaring \emph{type rules} [REF], we will also
require that for each desugaring rule, its cases be disjoint.


\begin{figure}
  NOTES: Core terms are terms without Surf or Aux nodes. Desugaring
  takes a term and produces a core term. Each sugar lhs and rhs is a
  surface pattern. Pattern variables match against core terms.
  \begin{Table}
    $c$ &$::=$& $\Value$ \\
    &$|$& $\Refn{x}$ \\
    &$|$& $\Decl{x}$ \\
    &$|$& $\Core{C}{c \dd}$ \\
    &$|$& $\Tag{p}{p}{c}$ \\
    &$|$& $[c \dd]$ \\
    \\
    $E_{ds}$ &$::=$& $\square$ \\
    &$|$& $[c \dd E_{ds}\,e \dd]$ \\
    &$|$& $\Core{C}{c \dd E_{ds}\,e \dd}$ \\
    &$|$& $\Surf{C}{c \dd E_{ds}\,e \dd}$ \\
    &$|$& $\Aux{C}{c \dd E_{ds}\,e \dd}$ \\
    &$|$& $\Tag{p}{p}{E_{ds}}$
  \end{Table}

  \begin{multicols}{2}
  \TypeLabel{\SaysDesugar{L}{s}{c}}

  \Inference[desugar]{
    \SaysDss{L}{s}{c}
  }{
    \SaysDesugar{L}{s}{c}
  }

  \TypeLabel{\SaysDs{L}{e}{e}} 

  \Inference[ds-ctx]{
    \SaysExp{L}{e}{e'}
  }{
    \SaysDs{L}{E_{ds}[e]}{E_{ds}[e']}
  }
  \end{multicols}

  \TypeLabel{\SaysExp{L}{e}{e}}
  
  \Inference[exp-sugar]{
    L = G,rs &
    \DsRule{C}{k_1 ... k_n} \in rs \IS\\
    k_i = \DsRuleCase{(p_1,...,p_n)}{\Gamma}{F}{p_{rhs}} \IS\\
    \SaysMatch{L}{c_i}{p_i}{\gamma_i}{p_i'} \text{ for each $i$} \IS\\
    \SaysMatch{L}{\Surf{C}{c_1 ... c_n}}{\Surf{C}{p_1 ... p_n}}{\gamma}{p_{lhs}} \IS\\
    \gamma' \text{ gives fresh names to the variables in $F$} \IS\\
%    \gamma_1 \cup ... \cup \gamma_n \cup \gamma' = \gamma \IS\\
    \SaysSubs{F}{(\gamma \cup \gamma')}{p_{rhs}}{e'}
  }{
    \SaysExp{L}{\Surf{C}{c_1 ... c_n}}{\Tag{p_{lhs}}{p_{rhs}}{e'}}
  }
  
  \Inference[exp-core]{
    L = G,rs &
    C \not\in rs &
    p = \Surf{C}{\PVarA_1 ... \PVarA_n} &
    p' = \Core{C}{\PVarA_1 ... \PVarA_n}
  }{
    \SaysExp{L}{\Surf{C}{c_1 ... c_n}}{\Tag{p}{p'}{\Core{C}{c_1 ... c_n}}}
  }
  \caption{Expansion}
  \label{fig:expansion}
\end{figure}




\subsection{Type Soundness}

We prove soundness by way of progress + preservation:
\begin{theorem}[Soundness]
  If $\SaysExpr{L}{e}{t}$, then
  $\SaysDss{L}{e}{v}$ where $\SaysExpr{L}{v}{t}$, or $e$ runs forever.
\end{theorem}
\begin{proof}
\Cref{thm:progress} (progress) and \cref{thm:preservation}
(preservation) together imply that either:
(i) $e$ is a value, or (ii) $\SaysDs{L}{e}{e'}$ and $\SaysExpr{L}{e'}{t}$.
Apply this repeatedly. Either $e$ eventually steps to a value $v$, and
has remained the same type $t$ throughout the evaluation, or $e$ never
halts.
\end{proof}

\begin{lemma}[Progress] \label{thm:progress}
  If $\SaysExpr{L}{e}{t}$, then
  $\SaysDs{L}{e}{e'}$, or $e$ is a value.
\end{lemma}
\begin{proof}
  First, verify that our evaluation contexts include every case that
  isn't a value. Thus either $e$ is a value and we are done, or $e$
  contains a redex: $e=E[\Surf{m}{e_1 ... e_n}]$.
  In the latter case, we will show that $e$ can take a step because
  the ds-sugar rule applies. There are two premises that need to be
  satisfied:
  \begin{itemize}
    \item First, we must show that $m$ is bound in $L$. Since $e$
      type-checked, it must be: the only rule which can type-check a
      sugar invocation is p-sugar; this in turn must use rule
      g-sugar; finally g-sugar requires that $m \in L$.
    \item Second, we must show that the pattern match of $e_1,...,e_n$
      succeeds on any case ${\DsRuleCase{(p_1,...,p_n)}{\Gamma}{F}{p'}}$
      of the desugaring
      rule. By \cref{thm:exhaustion}, it does.
  \end{itemize}
\end{proof}

\begin{assumption}[Exhaustion] \label{thm:exhaustion}
  If the set of cases in a desugaring rule are exhaustive over
  $t_1,...,t_n$ according to our exhaustion checking algorithm, then
  for every possible argument list $e_1,...,e_n$ that matches the
  given types (i.e., $\SaysExpr{L}{e_1}{t_1},...,\SaysExpr{L}{e_n}{t_n}$),
  there is a case $k_i$ such that $e_1,...,e_n$ successfully matches
  against $k_i$. [TODO: prove]
\end{assumption}
\begin{proof}
  \emph{Not given}. We have not stated our exhaustion checking
  algorithm here, and so cannot prove it correct. We believe it is
  straightforward (if tedious).
\end{proof}

\begin{lemma}[Preservation] \label{thm:preservation}
  If $\SaysExpr{L}{e}{t}$ and $\SaysDs{L}{e}{e'}$, then $\SaysExpr{L}{e'}{t}$.
\end{lemma}
\begin{proof}
  Since $e$ can take an expansion step, it must have a redex (via
  ds-ctx): $e = E[\Surf{m}{e_1 ... e_n}]$. And furthermore (by ds-sugar) $m$
  must be bound in $L$, and there must be a first case of $m$ that
  matches $e$.  Call it $k_i = (p_1,...,p_n);\Gamma \To p'$. Then:
  \begin{ProofTable}
  By ds-case: & $\SaysMatch{L}{e_i}{p_i}{\gamma_i}$
    for some $\gamma_i$ for each $i$ & (1) \\
  and & $\SaysSubs{F}{\gamma_i \cup ...
    \cup \gamma_n}{p'}{e'}$ & (2) \\
  and & $\SaysDs{L}{E[e]}{E[e']}$ \\
  By e-sugar: & $\SaysExpr{L}{\Surf{m}{e_1 ... e_n}}{t}$ \\
  and & $\SaysRule{L}{m}{t_1 ... t_n}{t}$ \\
  and & $\SaysExpr{L}{e_i}{t_i}$ for each $i$ & (3) \\
  By g-sugar: & $\SaysPatt{L}{\Gamma}{p_i}{t_i}$ for each $i$ & (4) \\
  and & $\SaysPatt{L}{\Gamma}{p'}{t}$ & (5)
  \end{ProofTable}
  By \cref{thm:matching} with (1), (3), and (4),
  $\SaysEnv{\gamma_i}{\Gamma}$ for each $i$. By \cref{thm:union},
  $\SaysEnv{\gamma_1 \cup ... \gamma_n}{\Gamma}$.
  Finally, by \cref{thm:substitution} with that last fact together
  with (2) and (5), $\SaysExpr{L}{e'}{t}$.
\end{proof}

\begin{lemma}[Union of Substitutions] \label{thm:union}
  If $\SaysEnv{L}{\gamma_1}{\Gamma}$ and $\SaysEnv{L}{\gamma_2}{\Gamma}$,
  then $\SaysEnv{L}{\gamma_1 \cup \gamma_2}{\Gamma}$.
\end{lemma}
\begin{proof}
  [TODO]
\end{proof}

\begin{lemma}[Matching] \label{thm:matching}
  If $\SaysPatt{L}{\Gamma}{p}{t}$
  and $\SaysExpr{L}{e}{t}$
  and $\SaysMatch{F}{e}{p}{\gamma}$,
  then $\SaysEnv{L}{\gamma}{\Gamma}$
\end{lemma}
\begin{proof}
  Induction on $p$.
  \begin{description}
  \item[$p = string$]
    \begin{ProofTable}
      By p-str: & $\SaysPatt{L}{\Gamma}{string}{\TString}$ & fixes $t$ \\
      By m-str: & $\SaysMatch{F}{string}{\TString}{\EmptySubs}$
        & fixes $\gamma$
    \end{ProofTable}
    Finally, by $\gamma$-env, $\SaysEnv{F}{\EmptySubs}{\Gamma}$
    (this applies for any $\Gamma$).
  \item[$p = x \not\in F$] (Analogous.)
  \item[$p = x \in F$] By p-refn or p-decl, 
    $\Gamma = \{\}$ and $t$ is {\TRefn} or {\TDecl}.
    By m-fresh, $e = y$ for some fresh name $y$, and $\gamma = \{\}$.
    And the conclusion follows: $\SaysEnv{L}{\{\}}{\{\}}$. [TODO]
  \item[$p = \PVarA$]
    \begin{ProofTable}
      By p-pvar: & $\SaysPatt{L}{\Gamma}{\PVarA}{t}$ & fixes $t$ \\
      and & $\PVarA: t \in \Gamma$ & (1) \\
      By m-pvar: & $\SaysMatch{F}{e}{\alpha}{\{\PVarA \mapsto e\}}$
        & fixes $\gamma$
    \end{ProofTable}
    Finally, using $\gamma$-env on the premise $\SaysExpr{L}{e}{t}$
    gives that $\SaysEnv{L}{\gamma}{\{\PVarA: t\}},\Gamma'$ for any
    $\Gamma'$. By (1), this is the form of $\Gamma$, so we can set
    $\Gamma'$ such that $\Gamma = {\{\PVarA: t\}},\Gamma'$, and we are done.
  \item[$p = \Core{C}{p_1 ... p_n}$]
    \begin{ProofTable}
      By p-core: & $\SaysPatt{L}{\Gamma}{\Core{C}{p_1 ... p_n}}{A}$ & fixes $t$ \\
      and & $A \mapsto \Core{C}{t_1 ... t_n} \in L$ \\
      and & $\SaysPatt{L}{\Gamma}{p_i}{t_i}$ for each $i$ & (1) \\
      By m-core: &
        $\SaysMatch{F}{\Core{C}{e_1 ... e_n}}{\Core{C}{p_1 ... p_n}}{\gamma}$
        & fixes $e$ \\
      and & $\SaysMatch{F}{e_i}{p_i}{\gamma_i}$ for each $i$ & (2) \\
      and & $\gamma = \gamma_1 \cup ... \cup \gamma_n$ \\
      By e-core: & $\SaysExpr{L}{\Core{C}{e_1 ... e_n}}{A}$ \\
      and & $\SaysExpr{L}{e_i}{t_i}$ for each $i$ & (3) \\
    \end{ProofTable}
    Applying the I.H. to (1), (2), and (3) yeilds that
    $\SaysEnv{L}{\gamma_i}{\Gamma}$.
    By \cref{thm:union}, $\SaysEnv{L}{\gamma}{\Gamma}$.
  \item[$p = \Surf{m}{p_1 ... p_n}$] [FILL]
  \item[$p = [\epsilon{]}$] [TODO] By m-empty, $\gamma = \{\}$.
    By p-empty, $\Gamma = \{\}$. The goal follows: $\SaysEnv{L}{\{\}}{\{\}}$.
  \item[$p = [p,ps{]}$] [FILL]
  \item[$p = [\Rep{p}{l'}{]}$]
    \begin{ProofTable}
      By p-ellipsis: & $\SaysPatt{L}{\Gamma}{\Rep{p}{l'}}{[t]}$ & fixes $t$ \\
      and & $l' \mapsto [\Gamma'] \in \Gamma$ & (1) \\
      and & $\SaysPatt{L}{\Gamma'}{p}{t}$ & (2) \\
      By m-ellipsis: & $\SaysMatch{F}{[e_1 ... e_n]}{[\Rep{p}{l}]}
        {\{l' \mapsto [\gamma_1 \cup ... \gamma_n]\}}$
        & fixes $e$, $\gamma$ \\
      and & $\SaysMatch{F}{e_i}{p}{\gamma_i}$ for each $i$ & (3) \\
      By e-list: & $\SaysExpr{L}{[e_1 ... e_n]}{[t]}$ \\
      and & $\SaysExpr{L}{e_i}{t_i}$ & (4)
    \end{ProofTable}
    By the I.H. together with (2), (3), and (4),
    $\SaysEnv{L}{\gamma_i}{\Gamma'}$ for each $i$.
    By \cref{thm:union},
    $\SaysEnv{L}{\gamma_1 \cup ... \gamma_n}{\Gamma'}$.
    Finally, by $\gamma$-env,
    $\SaysEnv{L}{\{l' \mapsto [\gamma_1 \cup ... \gamma_n]\}}
      {\{l' \mapsto [\Gamma']\}}$, which is compatible with the
      specification of $\Gamma$ in (1).
  \end{description}
\end{proof}

\begin{lemma}[Substitition] \label{thm:substitution}
  If $\SaysEnv{L}{\gamma}{\Gamma}$
  and $\SaysPatt{L}{\Gamma}{p}{t}$,
  and $\SaysSubs{F}{\gamma}{p}{e}$,
  then $\SaysExpr{L}{e}{t}$.
\end{lemma}
\begin{proof}
  .[TODO]: update proof (freshness)
  Induction on $p$.
  \begin{description}
  \item[$p = string$]
  \item[$p = string$] By s-str, $\SaysSubs{F}{\gamma}{p}{string}$, so $e=string$.
    By p-str, $\SaysPatt{L}{\Gamma}{p}{\TString}$, so $t=\TString$.
    Finally, by e-str, $\SaysExpr{L}{e}{\TString}$ as desired.
  \item[$p = x \not\in F$] (Analogous.)
  \item[$p = x \in F$] By s-fresh, $e = y$ for some fresh name $y$.
    By p-refn or p-decl, $t$ is either {\TRefn} or {\TDecl}.
    Our goal $\SaysExpr{L}{y}{t}$ follows by either e-refn or e-decl,
    respectively.
  \item[$p = \PVarA$] By rule s-pvar, $\PVarA \mapsto e \in \gamma$.
    By $\gamma$-env, $\alpha \mapsto t \in \Gamma$ and $\SaysExpr{L}{e}{t}$.
    Which is our goal; we are done.
    (Note that by $\gamma$-env, $\Gamma$ may have many \emph{other},
    unnecessary, bindings to pattern variables, but it must \emph{at least}
    contain a correct binding for $\alpha$.)
  \item[$p = \Core{C}{p_1 ... p_n}$]
    \begin{ProofTable}
      By p-core: & $\SaysPatt{L}{\Gamma}{\Core{C}{p_1 ... p_n}}{A}$ & fixes $t$ \\
      and & $A \mapsto \Core{C}{t_1 ... t_n} \in L$ & (1) \\
      and & $\SaysPatt{L}{\Gamma}{p_i}{t_i}$ for each $i$ & (2) \\
      By s-core: & $\SaysSubs{F}{\gamma}{\Core{C}{p_1 ... p_n}}{\Core{C}{e_1 ... e_n}}$
        & fixes $e$ \\
      and & $\SaysSubs{F}{\gamma}{p_i}{e_i}$ for each $i$ & (3)
    \end{ProofTable}
    Using the I.H. with (2) and (3) gives that
    $\SaysExpr{L}{e_i}{t_i}$ for each $i$.
    Using e-core on that fact together with (1) gives that
    $\SaysExpr{L}{\Core{C}{e_1 ... e_n}}{A}$, so we are done.
  \item[$p = \Surf{m}{p_1 ... p_n}$]
    \begin{ProofTable}
      By s-sugar: & $\SaysSubs{F}{\gamma}{\Surf{m}{p_1 ... p_n}}{\Surf{m}{e_1 ... e_n}}$
        & fixes $e$ \\
      and & $\SaysSubs{F}{\gamma}{p_i}{e_i}$ for each $i$ & (1) \\
      By p-sugar: & $\SaysPatt{L}{\Gamma}{\Surf{m}{p_1 ... p_n}}{t}$
        & fixes $t$ \\
      and & $\SaysPatt{L}{\Gamma}{p_i}{t_i}$ for each $i$ & (2) \\
      and & $\SaysRule{L}{m}{t_1,...,t_n}{t}$ & (3)
    \end{ProofTable}
    Using the I.H. with (1) and (2) gives that
    $\SaysExpr{L}{e_i}{t_i}$ for each $i$.
    Finally, using e-sugar on that fact together with (3) gives that
    $\SaysExpr{L}{\Surf{m}{e_1 ... e_n}}{t}$.
  \item[$p = [\epsilon{]}$]
    By s-empty, $\SaysSubs{F}{\gamma}{p}{[]}$, so $e=[]$.
    By p-empty, $\SaysPatt{L}{\EmptyEnv}{[\epsilon]}{[t]}$ (for some $t$).
    Finally, by e-list, $\SaysExpr{L}{[]}{[t]}$.
  \item[$p = [p,ps{]}$]
    \begin{ProofTable}
      By s-cons: & $\SaysSubs{F}{\gamma}{p}{e_1}$ & (1) \\
      and & $\SaysSubs{F}{\gamma}{[ps]}{[e_2 ... e_n]}$ & (2) \\
      and & $\SaysSubs{F}{\gamma}{[p, ps]}{[e_1 e_2 ... e_n]}$ & fixes $e$ \\
      By p-cons: & $\SaysPatt{L}{\Gamma}{p}{t}$ & (3) \\
      and & $\SaysPatt{L}{\Gamma}{[ps]}{[t]}$ & (4) \\
      and & $\SaysPatt{L}{\Gamma}{[p, ps]}{[t]}$ & fixes $t$
    \end{ProofTable}
    We can apply the I.H. using (1) and (3) and the assumption
    $\SaysEnv{L}{\gamma}{\Gamma}$ to get that $\SaysExpr{L}{e_1}{t}$.
    Likewise, the I.H. with (2) and (4) gives
    $\SaysExpr{L}{[e_2 ... e_n]}{[t]}$.
    By e-list (in reverse), $\SaysExpr{L}{e_2}{t} \cdots \SaysExpr{L}{e_n}{t}$.
    Finally, by e-list (forward), $\SaysExpr{L}{[e_1 e_2 ... e_n]}{[t]}$.
  \item[$p = [\Rep{p}{l}{]}$]
    \begin{ProofTable}
      By s-ellipsis: & $\SaysSubs{F}{\gamma}{[\Rep{p}{l}]}{[e_1 ... e_n]}$ & fixes $e$ \\
      and & $l \mapsto \EList{\gamma_1 ... \gamma_n} \in \gamma$ \\
      and & $\SaysSubs{F}{\gamma_i}{p}{e_i}$ for each $i$ & (1) \\
      By $\gamma$-env: & $l \mapsto \EList{\Gamma'} \in \Gamma$ \\
      and & $\SaysEnv{L}{\gamma_i}{\Gamma'}$ for each $i$ & (2) \\
      By p-ellipsis: & $\SaysPatt{L}{\Gamma}{[\Rep{p}{l}]}{[t]}$ & fixes $t$ \\
      and & $\SaysPatt{L}{\Gamma'}{p}{t}$ & (3)
    \end{ProofTable}
    Using the I.H. with (1), (2), and (3) proves that
    $\SaysExpr{L}{e_i}{t}$.
    Then, by e-list, $\SaysExpr{L}{[e_1 ... e_n]}{[t]}$ as desired.
  \end{description}
\end{proof}


\newpage
\section{Example}

\subsection{Define-struct}

\paragraph{Core AST}
\begin{Codes}
Stmts:
| [\{splicing-begin stmts:Stmts\} @rest:Stmts]
   binding stmts in rest
   providing stmts, rest

| [\{let x:Var v:Expr\} @rest:Stmts]
   binding x in rest

| [\{fun f:Var args:Args body:Expr\} @rest:Stmts]
   binding args in body, rest in body
   providing f, rest
\end{Codes}

%% ALTERNATIVELY:

%% Stmts:
%% | \{splicing-begin stmts:Stmts rest:Stmts\}
%%    binding stmts in rest
%%    providing stmts, rest

%% | \{let x:Var v:Expr rest:Stmts\}
%%    binding x in rest

%% | \{fun f:Var args:Args body:Expr rest:Stmts\}
%%    binding args in body, rest in body
%%    providing f, rest

%% | \{end\}

%% Params:
%% | \{param x:Var rest:Params\}
%% | \{end\}

\paragraph{Auxiliary AST}
\begin{Codes}
IStructFields:
| [field:IStructField ...fields:IStructFields]
  providing field, fields

IStructField:
| \{i-struct-field field:Str get:Var set:Var\}
  providing get, set
\end{Codes}

\paragraph{Surface AST}
\begin{Codes}
SurfStmts:
  .....
| [(define-struct name:Var fields:StructFields) @rest:SurfStmts]
  binding name in rest, fields in rest
  providing name, fields, rest

StructFields:
| [field:StructField ...fields:StructFields]
  providing field, fields

StructField:
| (struct-field field:Str get:Var set:Var)
  providing get, set
\end{Codes}

\paragraph{Desugaring Rules}
\begin{Codes}
   [(struct-field field:Str get:Var set:Var) @rest:IStructFields]
=> [\{i-struct-field field get set\} @rest]
  
   [(define-struct name:Var
      [(struct-field field:Str get:Var set:Var) ...]) @rest:SurfStmts]
=> [\{fun name [x ...] \{record [\{record-field field x\} ...]\}\}
    \{splicing-begin [\{fun get [rec] \{record-get rec field\}\} ...]\}
    \{splicing-begin [\{fun set [rec val] \{record-set rec field val\}\} ...]\}
    @rest]
\end{Codes}

\subsection{Pyret For Expressions}

To handle Pyret for-expressions, we need to add desugaring rules for
the \Code{for} and for its bindings. 
First, when a for-expression binding (e.g. \Code{n from 0}) desugars,
it will simply return its binding (\Code{n}) and its value (\Code{0}).
It can do so with the desugaring rule:
\begin{Codes}
   (s-for-bind l:Loc b:Bind v:Expr)
=> \{for-bind b v\}
\end{Codes}
where \Code{ForBind} is a new auxiliary type:
\begin{Codes}
ForBind += \{for-bind Bind Expr\}

with list scope:
  [\{for-bind b v\} bs ...]
  export b
  export bs
\end{Codes}

Then the surface grammar can be extended for \Code{for} expressions:
\begin{Codes}
Expr += (s-for Loc Expr [ForBind] Ann Expr Bool)
with scope:
  bind binds in body
\end{Codes}

Then for-expressions can be implemented with the desugaring rule:
\begin{Codes}
   (s-for l:Loc
          iter:Expr
          [\{for-bind bind:Bind value:Expr\} ...]@binds
          ann:Ann
          body:Expr
          blocky:Bool)
=> (Apply l iter
     [(Lambda l (CONCAT "for-body<" (FORMAT l false) ">")
        [] [bind ...] ann "" body None None blocky)
      value ...])
\end{Codes}

Notice that \Code{s-for} is pattern matching against the results of
desugaring the \Code{s-for-bind}s. The \Code{(CONCAT ...)} stuff is to
compute at compile time a name for this lambda, which is what Pyret
currently does.

\newcommand{\C}{\(\sb{c}\)}
\subsubsection{Desugaring Pyret For Expressions}
\begin{Codes}
   (s-for map
     [(s-for-bind x (list [(+ 1 2)]))]
     (* x x))
=> (s-for map [(s-for-bind x (list
     [(Tag<(+ a b),(+\C a b)> (+\C 1 2))]))] (* x x))
=> (s-for map [(s-for-bind x (Tag<(list a),(list\C a)> (list\C
     [(Tag<(+ a b),(+\C a b)> (+\C 1 2))])))] (* x x))
=> (s-for map
     [(Tag<(s-for-bind b v),\{for-bind b v\}> \{for-bind x
       (Tag<(list a),(list\C a)> (list\C
         [(Tag<(+ a b),(+\C a b)> (+\C 1 2))]))\})] (* x x))
=> (s-for map
     [(Tag<(s-for-bind b v),\{for-bind b v\}> \{for-bind x
       (Tag<(list a),(list\C a)> (list\C
         [(Tag<(+ a b),(+\C a b)> (+\C 1 2))]))\})] (* x x))
=> (s-for map
     [(Tag<(s-for-bind b v),\{for-bind b v\}> \{for-bind x
       (Tag<(list a),(list\C a)> (list\C
         [(Tag<(+ a b),(+\C a b)> (+\C 1 2))]))\})]
     (Tag<(* a b),(*\C a b)>
       (*\C x x)))
=> (Tag<(s-for iter
         [(Tag<(s-for-bind b v),\{for-bind b v\}>
             \{for-bind bind value\}) ...]
           body),
         (apply iter [(lambda [bind ...] body) value ...])>
      (apply map [(lambda [x] (Tag<(* a b),(*\C a b)> (*\C x x)))
        (Tag<(list a),(list\C a)> (list\C
          [(Tag<(+ a b),(+\C a b)> (+\C 1 2))]))]))
=> (Tag<(s-for iter
         [(Tag<(s-for-bind b v),\{for-bind b v\}>
             \{for-bind bind value\}) ...]
           body),
         (apply iter [(lambda [bind ...] body) value ...])>
      (apply map [(Tag<..,..> (lambda\C [x]
          (Tag<(* a b),(*\C a b)> (*\C x x))))
        (Tag<(list a),(list\C a)> (list\C
          [(Tag<(+ a b),(+\C a b)> (+\C 1 2))]))]))
=> (Tag<(s-for iter
         [(Tag<(s-for-bind b v),\{for-bind b v\}>
             \{for-bind bind value\}) ...]
           body),
         (apply iter [(lambda [bind ...] body) value ...])>
      (Tag<..,..> (apply\C map [(Tag<..,..>(lambda\C [x]
          (Tag<(* a b),(*\C a b)> (*\C x x))))
        (Tag<(list a),(list\C a)> (list\C
          [(Tag<(+ a b),(+\C a b)> (+\C 1 2))]))])))
-> (Tag<(s-for iter
         [(Tag<(s-for-bind b v),\{for-bind b v\}>
             \{for-bind bind value\}) ...]
           body),
         (apply iter [(lambda [bind ...] body) value ...])>
      (Tag<..,..> (apply\C map [(Tag<..,..> (lambda\C [x]
           (Tag<(* a b),(*\C a b)> (*\C x x))))
        (Tag<(list a),(list\C a)> (list\C [3]))])))
<= (Tag<(s-for iter
         [(Tag<(s-for-bind b v),\{for-bind b v\}>
             \{for-bind bind value\}) ...]
           body),
         (apply iter [(lambda [bind ...] body) value ...])>
      (Tag<..,..> (apply\C map [(Tag<..,..> (lambda\C [x] (* x x)))
        (Tag<(list a),(list\C a)> (list\C [3]))])))
<= (Tag<(s-for iter
         [(Tag<(s-for-bind b v),\{for-bind b v\}>
             \{for-bind bind value\}) ...]
           body),
         (apply iter [(lambda [bind ...] body) value ...])>
      (Tag<..,..> (apply\C map [(lambda [x] (* x x))
        (list [3])])))
<= (Tag<(s-for iter
         [(Tag<(s-for-bind b v),\{for-bind b v\}>
             \{for-bind bind value\}) ...]
           body),
         (apply iter [(lambda [bind ...] body) value ...])>
      (apply map [(lambda [x] (* x x)) (list [3])]))
<= (Tag<(s-for iter
         [(Tag<(s-for-bind b v),\{for-bind b v\}>
             \{for-bind bind value\}) ...]
           body),
         (apply iter [(lambda [bind ...] body) value ...])>
      (apply map [(lambda [x] (* x x)) (list [3])]))
  // iter=map, bind=[x], body=(* x x), value=[(list [3])]
<= (s-for map
     [(Tag<(s-for-bind b v),\{for-bind b v\}>
        \{for-bind x (list [3])\})]
     (* x x))
<= (s-for map
     [(s-for-bind x (list [3]))]
     (* x x))
\end{Codes}

