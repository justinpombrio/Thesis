\chapter{Notation}\label{chap:notation}

In this chapter, we define notation and set assumptions
that will be used throughout the rest of the thesis.

Most importantly, we assume that desugaring is given (externally to the
language) as a set of pattern-based rewrite rules (a la Scheme's \Code{syntax-rules}~\cite{scheme5}).
This is important because it will form an expressive limit on
our techniques for resugaring evaluation sequences (\cref{chap:resugar-eval}),
scope rules (\cref{chap:resugar-scope}), and type systems (\cref{chap:resugar-types}).


\section{ASTs}\label{sec:formal-term}

For the purposes of resugaring scope (\cref{chap:resugar-scope}), we require \Sc{ast}s
(\emph{abstract syntax trees}) to explicitly distinguish between
variable \emph{declarations} $\Decl{x}$ (i.e., binding sites), and
variable \emph{references} $\Refn{x}$ (i.e., uses).
We also require variables to have an \emph{\textsc{ast} position} $i$
that uniquely distinguishes them.
These distinctions will be important in \cref{chap:resugar-scope}, but
can otherwise be ignored.

We will refer to \Sc{ast}s (and parts of \Sc{ast}s)
as \emph{terms}, and write them as $e$.\marginpar{%
  While $t$ would be a more obvious letter to use for terms, we use it
  to refer to types in \cref{chap:resugar-types}.
}
Terms can be inductively defined as:
\begin{TableForMitch}
constructor $C$ &$::=$& \textit{name} & syntactic construct name \\
term $e$ &$::=$& $\Value$ & primitive value \\
  &$|$& $\Node{C}{e_1 \dd e_n}$ & \Sc{ast} node \\
  &$|$& $\Refn[i]{x}$  & variable reference at position $i$ \\
  &$|$& $\Decl[i]{x}$  & variable declaration at position $i$ \\
\end{TableForMitch}
By ``constructor'', we mean the name of a syntactic construct, such as
\Code{Plus} or \Code{If}. Thus $\NodeRm{Plus}{\Refn[i]{x}\ 1}$ is a term.

\section{Desugaring}\label{sec:formal-desugar}

We require that desugaring be based on \emph{patterns}: it proceeds by
\emph{matching} a \Sc{lhs} (left-hand-side) pattern against a term,
and then \emph{substituting} into the \Sc{rhs} (right-hand-side)
pattern.  We will describe how this works, piece by piece.

First, \emph{patterns} $p$ are terms that can contain pattern variables $\PVarA$:
\begin{Table}
pattern $p$ &$::=$& $\PVarA$ & pattern variable \\
  &$|$& $\Value$ & primitive value \\
  &$|$& $\Node{C}{p_1 ... p_n}$ & \Sc{ast} node \\
  &$|$& $\Refn[i]{x}$  & variable reference at position $i$ \\
  &$|$& $\Decl[i]{x}$  & variable declaration at position $i$ \\
\end{Table}
A term can then be \emph{matched} against a pattern---written
$\Match{e}{p}$---to produce an \emph{environment} $\gamma$. An
environment is a mapping from pattern variables to terms:
\begin{Table}
  environment $\gamma$ &$::=$& $\{\PVarA \mapsto e, ...\}$
\end{Table}
Once an environment is obtained, it can be \emph{substituted} into a
pattern---written $\Subs{\gamma}{p}$---to produce a term.

We can now say how to \emph{expand} a single desugaring rule. Each
rule has the form:
\begin{Table}
  desugaring rule $r$ &$::=$& $p \To p'$
\end{Table}
where $p$ is the rule's \Sc{lhs}, and $p'$ is the rule's \Sc{rhs}.
To expand a rule $p \To p'$ in a term $e$, match against the \Sc{lhs}
and substitute into the \Sc{rhs}: $\Subs{(\Match{e}{p})}{p'}$.

Finally, \emph{desugaring} is the recursive expansion of all of the
sugars in a term. We will typically assume that desugaring happens in
\Sc{oi} order (see \cref{sec:taxonomy-order}), although
\cref{chap:resugar-scope} will be agnostic to desugaring order.

\paragraph{An Example}
Putting this all together, suppose we have the
following sugar, which encodes Let using Lambda:
\begin{Table}
  $\NodeRm{Let}{\PVarA\;\PVarB\;\PVarC}$
  &$\To$&
  $\NodeRm{Apply}{\NodeRm{Lambda}{\PVarA\;\PVarC}\;\PVarB}$
  & ``Let $\PVarA$ equal $\PVarB$ in $\PVarC$''
\end{Table}
and we want to \emph{expand} the term
$\NodeRm{Let}{\DeclX\;1\;\RefnX}$. To do so, we match against the
\Sc{lhs}:
\[
\gamma = \Match{\NodeRm{Let}{\DeclX\;1\;\RefnX}}{\NodeRm{Let}{\PVarA\;\PVarB\;\PVarC}}
= \{\PVarA \mapsto \DeclX,\, \PVarB \mapsto 1,\, \PVarC \mapsto \RefnX\}
\]
and substitute into the \Sc{rhs}:
\[
\Subs{\{\PVarA \mapsto \DeclX,\, \PVarB \mapsto 1,\, \PVarC \mapsto \RefnX\}}
     {\NodeRm{Apply}{\NodeRm{Lambda}{\PVarA\;\PVarC}\;\PVarB}}
     = \NodeRm{Apply}{\NodeRm{Lambda}{\DeclX\;\RefnX}\;1}
\]
Since there are no more sugars in the term, we are done desugaring.


\subsection{Restrictions on Desugaring}\label{sec:formal-reqs}

Not every syntactically valid desugaring rule is semantically
sensible, and not every semantically sensible desugaring rule is
feasible to resugar. Here we give (i) a set of well-formedness
criteria on desugaring rules, without which they don't make semantic
sense, and (ii) a set of restrictions on desugaring rules that we need
to effectively resugar them. These restrictions are sufficient for all
of the resugaring methods presented in this thesis.

\paragraph{Well-formedness Criteria for Desugaring Rules}\WhitePhantom{.}\\ \noindent
For every desugaring rule:
\begin{enumerate}
\item \emph{Each pattern variable in the \Sc{rhs} also appears in the
  \Sc{lhs}.} Otherwise the pattern variable would be unbound during
  expansion.
\item \emph{The \Sc{lhs} pattern contains no references or declarations.} Rather, these
  should be bound to its pattern variables during expansion.
  For example, the Let sugar above \emph{could not} have been written as:
\begin{Table}
  $\NodeRm{Let}{\DeclX\;\PVarB\;\PVarC}$
  &$\To$&
  $\NodeRm{Apply}{\NodeRm{Lambda}{\DeclX\;\PVarC}\;\PVarB}$
  & ``Let $\DeclX$ equal $\PVarB$ in $\PVarC$''
\end{Table}
\item \emph{Each desugaring rule's \Sc{lhs} must have the form
  $\Node{C}{p_1,\,...,\,p_n}$.} I.e., it cannot be just a variable or primitive.
  We will need this fact in \cref{chap:resugar-eval}.
\end{enumerate}

\paragraph{Restrictions on Desugaring Rules}\WhitePhantom{.}\\ \noindent
For every desugaring rule:
\begin{enumerate}
\item \emph{Each pattern variable appears at most once in the \Sc{lhs} and at
  most once in the \Sc{rhs}.}
  Allowing duplicate pattern variables complicates matching, unification,
  and proofs of correctness. It also copies code
  and, in the worst case, can exponentially blow up programs.
  We therefore disallow duplication,
  with some exceptions for pattern variables bound to atomic terms.
%\item References and declarations in the \Sc{rhs} are given fresh names during
%   expansion to ensure hygiene.
% (Discussed, but not actually assumed in Chapter 5.)
\item \emph{The rules' \Sc{lhs}s must be disjoint.} If they
  are not, it presents issues both for resugaring evaluation sequences
  (\cref{sec:reval-disjoint}), and resugaring type systems
  (\cref{sec:rtype-req-desugar}).
\end{enumerate}

Some of the resugaring chapters relax these assumptions. For instance, the
evaluation resugaring chapter (\cref{chap:resugar-eval}) formally
supports ellipses (which allow matching zero or more repetitions of a
pattern), and the scope resugaring chapter is agnostic to the order of
desugaring (which is assumed to be \Sc{oi} here). However, all
chapters handle at least these kinds of sugars.

\subsection{Relationship to Term Rewriting Systems}

A set of pattern-based desugaring rules can be viewed as a term
rewriting system~\cite{trs}. Two of the main questions studied about
term rewriting systems is how to tell whether they are
\emph{confluent}, and whether they \emph{terminate}. Confluence asks:
if rewrites are applied in an arbitrary order, are they guaranteed to
eventually produce the same result? The main mechanism for studying
this is \emph{critical pairs}~\cite{critical-pairs},
which capture the situations in which rewrite rules overlap. In
practice, however, desugaring systems tend to fix a desugaring order
(e.g. \Sc{oi}), making the quesiton of confluence moot.

Likewise, there is literature on determining whether a set of rewrite
rules always terminates (which is undecidable in
general)~\cite{trs-termination-undecidable}. This could be used to
check whether desugaring is guaranteed to halt (with potential false
positives). Termination checking is not necessary for
\emph{resugaring}, however: term resugaring, scope rule resugaring,
and type rule resugaring all halt, even if desugaring itself may not.
