\chapter{Notation}\label{chap:notation}

In this chapter, we set the groundwork of notation and basic assumptions
that will be used throughout the rest of the thesis.

Most importantly, we assume that desugaring is given (externally to the
language) as a set of pattern-based rewrite rules (a la Scheme's \Code{syntax-rules}~\cite{scheme5}.)
This is important because it will form an expressive limit on
our techniques for resugaring evaluation sequences \cref{chap:resugar-eval},
scope rules \cref{chap:resugar-scope}, and type systems \cref{chap:resugar-types}.


\section{ASTs}\label{sec:formal-term}

For the purposes of resugaring scope, we require \Sc{ast}s
(\emph{abstract syntax trees}) to explicitly distinguish between
variable \emph{declarations} $\Decl{x}$ (i.e., binding sites), and
variable \emph{references} $\Refn{x}$ (i.e., uses).
This distinction will be important in \cref{chap:resugar-scope}, but
can otherwise be ignored.

We will refer to \Sc{ast}s (and parts of \Sc{ast}s)
as \emph{terms}, and write them as $e$.\marginpar{%
  While $t$ would be a more obvious letter to use for terms, we use it
  to refer to types in \cref{chap:resugar-types}.
}
Terms can be inductively defined as:
\begin{Table}
constructor $C$ &$::=$& \textit{name} & syntactic construct name \\
term $e$ &$::=$& $\Value$ & primitive value \\
  &$|$& $\Node{C}{e_1 \dd e_n}$ & \Sc{ast} node \\
  &$|$& $\Refn[i]{x}$  & variable reference at position $i$ \\
  &$|$& $\Decl[i]{x}$  & variable declaration at position $i$ \\
\end{Table}
A ``constructor'' is the name of a syntactic construct, e.g. \Code{+}
or \Code{if}. Thus $(\NodeRm{+}{\Refn[i]{x}\ 1}$ is a term.

\section{Patterns}

We require that desugaring be based on \emph{patterns}: it
proceeds by \emph{matching} a \Sc{lhs} pattern against a term,
and then \emph{substituting} into the \Sc{rhs} pattern.

Patterns $p$ are just terms that can contain pattern variables $\PVarA$:
\begin{Table}
pattern $p$ &$::=$& $\PVarA$ & pattern variable \\
  &$|$& $\Value$ & primitive value \\
  &$|$& $\Node{C}{p_1 ... p_n}$ & \Sc{ast} node \\
  &$|$& $\Refn[i]{x}$  & variable reference at position $i$ \\
  &$|$& $\Decl[i]{x}$  & variable declaration at position $i$ \\
\end{Table}
Matching a term against a pattern produces an \emph{environment}, or a
mapping from pattern variable to term:
\begin{Table}
  environment $\gamma$ &$::=$& $\{\PVarA \mapsto e, ...\}$
\end{Table}
We will write desugaring rules with a double arrow:
\begin{Table}
  desugaring rule $r$ &$::=$& $p \To p'$
\end{Table}

\section{Desugaring}\label{sec:formal-desugar}

Putting this all together, suppose we have the
following sugar, which encodes Let using Lambda:
\begin{Table}
  $\NodeRm{Let}{\PVarA\;\PVarB\;\PVarC}$
  &$\To$&
  $\NodeRm{Apply}{\NodeRm{Lambda}{\PVarA\;\PVarC}\;\PVarB}$
  & ``Let $\PVarA$ equal $\PVarB$ in $\PVarC$''
\end{Table}
and we want to \emph{expand} the term
$\NodeRm{Let}{\DeclX\;1\;\RefnX}$. To do so, we match against the
\Sc{lhs}:
\[
\gamma = \Match{\NodeRm{Let}{\DeclX\;1\;\RefnX}}{\NodeRm{Let}{\PVarA\;\PVarB\;\PVarC}}
= \{\PVarA \mapsto \DeclX,\, \PVarB \mapsto 1,\, \PVarC \mapsto \RefnX\}
\]
and substitute into the \Sc{rhs}:
\[
\Subs{\{\PVarA \mapsto \DeclX,\, \PVarB \mapsto 1,\, \PVarC \mapsto \RefnX\}}
     {\NodeRm{Apply}{\NodeRm{Lambda}{\PVarA\;\PVarC}\;\PVarB}}
     = \NodeRm{Apply}{\NodeRm{Lambda}{\DeclX\;\RefnX}\;1}
\]

That was the \emph{expansion} of a single rule, a.k.a. a single
\emph{rewrite}. \emph{Desugaring} is
the expansion of all sugars in a term. We write it as $\Desugar{e}$,
and assume that it is in \Sc{oi} order (see \cref{sec:taxonomy-order}).


\subsection{Restrictions on Desugaring}\label{sec:formal-reqs}

Not every syntactically valid desugaring rule is semantically
sensible, and not every semantically sensible desugaring rule is
feasible to resugar. Here we give (i) a set of well-formedness
criteria on desugaring rules, without which they don't make semantic
sense, and (ii) a set of restrictions on desugaring rules that we need
to effectively resugar them.

\paragraph{Well-formedness Criteria for Desugaring Rules}
\begin{enumerate}
\item \emph{Each pattern variable in the \Sc{rhs} also appears in the
  \Sc{lhs}.} Otherwise the pattern variable would be unbound during
  expansion.
\item The \Sc{lhs} pattern contains no references or declarations. Rather, these
  should be contained in its pattern variables during expansion.
\end{enumerate}

\paragraph{Restrictions on Desugaring Rules}
\begin{enumerate}
\item \emph{Each pattern variable appears at most once in the \Sc{lhs} and at
  most once in the \Sc{rhs}.}
  Allowing duplicate pattern variables complicates matching, unification,
  and proofs of correctness. It also copies code
  and, in the worst case, can exponentially blow up programs.
  We therefore disallow duplication,
  with some exceptions for pattern variables bound to atomic terms.
\item \emph{Each desugaring rule's \Sc{lhs} must have the form
  $\Node{C}{e_1,\,...,\,e_n}$.} We will rely on this fact when showing that
  unexpansion is an inverse of expansion in \cref{sec:reval-inverses}.
\item References and declarations in the \Sc{rhs} are given fresh names during
  expansion to ensure hygiene.
\item \emph{The rules' \Sc{lhs}s must be disjoint.} If they
  are not, it presents issues both for resugaring evaluation sequences,
  and resugaring type systems.
\end{enumerate}

Some of the resugaring chapters go beyond the assumptions shown here: for instance, the
evaluation resugaring chapter (\cref{chap:resugar-eval}) formally
supports ellipses (which allow matching zero or more repetitions of a
pattern), and the scope resugaring chapter is agnostic to the order of
desugaring (which is assumed to be \Sc{oi} here). However, all
chapters handle at least these kinds of sugars.
