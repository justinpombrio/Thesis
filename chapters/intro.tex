\chapter{Syntactic Sugar}

\section{What is it?}

% http://www.cs.cmu.edu/~crary/819-f09/Landin64.pdf
The term \emph{syntactic sugar} was introduced by Peter Landin in
1964[CITE]. It refers to surface syntactic forms that are provided for
convenience, but could instead be written using the syntax of the rest
of the language. This captures the spirit and purpose of syntactic
sugar, but is not discriminating enough to be a useful definition.
I will define syntactic sugar as follows:
%% [TODO: definition must (i) rule out functions, (ii) rule out
%%   metaprogramming, (iii) include Racket macros, which are non-local
%%   and non-phase-specific.]
\begin{quote}
  A syntactic construct in an implementation of a programming language
  is \emph{syntactic sugar} if it is translated at compile time into
  the syntax of the rest of the language.
\end{quote}
%% \begin{quote}
%%   'A construct in a language is called ``syntactic sugar'' if it can be
%%   removed from the language without any effect on what the language
%%   can do: functionality and expressive power will remain the same.' (Wikipedia)
%% \end{quote}

The name suggests that syntactic sugar is inessential: it ``sweetens''
the language to make it more palatable, but does not otherwise change
its substance. The name also naturally leads to related terminology:
\emph{desugaring} is the removal of syntactic sugar by expanding it;
and \emph{resugaring} is a term I am introducing for the restoration
of information that was lost during desugaring.

\section{What is it Good for?}

Syntactic sugar is used to define abstractions. But languages have
other ways to define abstractions already: functions, classes, data
definitions, etc. If an abstraction can be implemented using these
features, it's almost always better to do so. Thus:
\begin{quote}
  Syntactic sugar should only be used to implement an abstraction if
  it cannot be implemented in the core language directly.
\end{quote}

Therefore, to find places where it is a \emph{good} idea to use sugar,
we can just look for things that most programming languages
\emph{cannot} abstract over:
\begin{enumerate}
  \item In most languages, variable names are first order and cannot
    be manipulated at run-time (e.g., a variable cannot be passed as
    an argument to a function: if you attempt to do so, the thing the
    variable is bound to will be passed instead). Therefore, creating
    new binding constructs is a good use for syntactic sugar in most
    languages. However, in R[CITE] variable names can be abstracted
    over (e.g. \Code{assign("x", 3); x} prints 3), and so sugar isn't
    necessary for this purpose.
  \item Most languages force the arguments to a function to be
    evaluated when it is called, rather than allowing the function to
    choose whether to
    evaluate them or not. Thus if an abstraction requires delaying
    evaluation, it is a good candidate to be a sugar. However, Haskell
    has lazy evaluation, and thus does not need sugars for this purpose.
  \item Most languages cannot manipulate data definitions at run-time:
    e.g., field names cannot be dynamically constructed. Thus creating
    new data definition constructs (e.g., a way to define state
    machines) is a good use case for sugars. However, in Python field
    names are first class (e.g., they can be added or assigned using
    \Code{setattr}), so sugars are not needed to abstract over fields
    in data definitions.
    % e.g.:
    %\Code{class C: pass; c = C(); setattr(c, ``x'', 3); print(c.x)}
    \footnote{
    It may sound like I am suggesting that everything should be
    manipulatable at run-time. I am not. The more
    things which are fixed at compile time (variable names, field
    names, etc.), the more (i) programmers can reason about their
    programs; (ii) tools can reason about programs; (iii) compilers
    can optimize programs (without herculean effort). It is
    \emph{good} that sugar is sometimes necessary.
  }
\end{enumerate}

Overall, syntactic sugar is a way to \emph{extend a language}.
In a limited sense, this is what functions are for as well.
Functions, however, are limited: they cannot take a variable as an
argument, delay evaluation, introduce new syntax, etc. This is where
sugar shines.


\subsection{A Thousand Grains of Sugar}

There are many axes on which desugaring mechanisms vary:
\begin{enumerate}
  \item Desugaring is a syntax-to-syntax transformation, but what is
    the representation of syntax? There is a big difference between
    transformations on the \emph{text} of the program vs. its
    \emph{concrete surface syntax} vs. its \emph{abstract surface
      syntax}.
  \item Are sugars defined by developers of the language (and thus
    relatively fixed), or by users of the language (and thus flexible)? % macros?
  \item What is the metalanguage? That is, in what language are sugars
    written? Is it the same language the programs are written in, thus
    allowing sugars and code to be interspersed, or a different
    language?
  \item In what order are constructs desugared? Most importantly, are
    nested sugars evaluated from the \emph{innermost to outermost}, or
    from \emph{outermost to innermost}?
  \item How many \emph{phases} of desugaring are there? Can there be
    more than one?
  \item How \emph{safe} is it? Can desugaring produce syntactically
    invalid code? Can it produce an unbound variable, or accidentally
    capture a variable? Can it produce code that contains a type error?
  \item How powerful is it? Can a sugar take an arbitrary expression
    as a parameter, or only a primitive type? If it can take an
    expression, can it deconstruct it, or only use it parametrically?
    Are the sugars written in a general purpose language, or in a
    restricted one, e.g. as declarative pattern-based rules?
\end{enumerate}

There is one big cluster of desugaring systems that should be called
out by name: macro systems. They can be loosely defined as:
\begin{quote}
  \emph{Macros} are user-defined sugars.
\end{quote}
In practice, macro systems tend to share several features: (i) by
definition, they are user-defined; (ii) the metalanguage is the
programming language itself [CHECK]; and (iii) the evaluation order is
usually outside-in [CHECK]. [CHECK: others?]

\subsection{Syntax Representation}

There are many ways to represent a program. The most prevalent are as
\emph{text} and as a \emph{tree}. Programs are most commonly
\emph{saved as} and \emph{edited as} text (notable exceptions include
visual and block based language/editors), and they are most commonly
\emph{internally represented as} trees (notable exceptions include
assembly, whose code is linear, and Forth, which does not have a
parsing phase).

Desugaring systems may be based on either representation.
\emph{However, text is a terrible representation for desugaring rules.}
The semantics of a language is almost always defined in terms of its
(abstract syntax) tree representation. Thus, insofar as a programmer
as a programmer is forced to think of their program as text rather
than as a tree, they are being distracted from its semantics. There
are well-known examples of bugs that arise in text-based desugaring
rules unless they are written in a very defensive style: I discuss
these in \ref{sec:cpre}.

There are variations among tree representations as well: desugaring
rules may work over the concrete syntax of the language, or over its
abstract syntax. I discuss this further in [REF].

\subsection{Language-defined or User-defined}

Sugars may either be specified and implemented as part of the
language, or they may be defined by users. For example, Haskell list
comprehensions are defined by the Haskell spec [CITE] and implemented
in the compiler(s); thus they are language-defined. Template Haskell
sugars [CITE], on the other hand, can be defined (and used) in any
Haskell program; thus they are user-defined. The difference between the two
is whether sugar is only a convenient method of simplifying language
design, or whether users are given the power to extend the language
themselves.

When sugars are user-defined, it raises difficulties with tools such
as editors that need to support the new syntax. For example, if sugars
are language-defined then an editor can just support the full
language. If they are user-defined, however, how can an editor provide
correct indentation, syntax highlighting, etc.?

Different languages work around this problem in different ways. Lisps
mostly avoid the problem by mostly not having syntax. They don't
\emph{entirely} avoid the problem, though. For example, the DrRacket editor for
Racket [CITE] allows indentation schemes to be set on a per-macro
basis (because different syntactic forms, while purely parenthetical,
still have varying nesting patterns that should be indented
differently), and it displays arrows showing where variables are bound
by being macro-aware and expanding the program (see [REF]).
[FILL: SugarJ, other examples]

[FILL: resugaring in general]

[TODO: macro-defining-macros]


\subsection{Metalanguage}


\subsection{Evaluation Strategy}
% https://dl.acm.org/citation.cfm?id=1440085
% GRAMMARS WITH MACRO-LIKE PRODUCTIONS (Fischer)

There are two major evaluation strategies used in desugaring systems.
They loosely correspond to eager and lazy runtime evaluation, but
differ in some important ways described below, so I will instead refer
to them by their original names [CITE]: Outside-in (\textsc{oi}) and
Inside-out (\textsc{io}) evaluation:
\begin{description}
\item[\textsc{oi}] evaluation is similar to lazy evaluation.
  However, it has an unusual property: [FILL]
\item[\textsc{io}] evaluation is similar to eager evaluation. It has [FILL]
\end{description}

\subsection{Number of Phases}

\subsection{Safety} [TODO]: AST safety, scope-safety, type-safety
