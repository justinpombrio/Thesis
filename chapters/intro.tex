\chapter{Syntactic Sugar}

\section{What is it?}

% http://www.cs.cmu.edu/~crary/819-f09/Landin64.pdf
The term \emph{syntactic sugar} was introduced by Peter Landin in
1964~\cite{syntactic-sugar}. It refers to surface syntactic forms that are provided for
convenience, but could instead be written using the syntax of the rest
of the language. This captures the spirit and purpose of syntactic
sugar.

The name suggests that syntactic sugar is inessential: it ``sweetens''
the language to make it more palatable, but does not otherwise change
its substance. The name also naturally leads to related terminology:
\emph{desugaring} is the removal of syntactic sugar by expanding it;
and \emph{resugaring} is a term that we will introduce in this thesis
for recovering various pieces of information that were lost during
desugaring.

% https://docs.oracle.com/javase/specs/jls/se8/html/jls-14.html#jls-14.14.2
% (Section 14.14.2)
% Also Rust: http://xion.io/post/code/rust-for-loop.html
As an example of a sugar, consider Java's ``enhanced for statement''
(a.k.a., for-each loop), which can be used to print out the best
numbers:
\begin{Codes}
for (int n : best_numbers) \{
  System.out.println(n);
\}
\end{Codes}
This enhanced \Code{for} is quite convenient, but it isn't really
\emph{necessary}, because you can always do the same thing with a
regular for loop and an iterator:
\begin{Codes}
for (Iterator i = best_numbers.iterator(); i.hasNext(); ) \{
  int n = (int) i.next();
  System.out.println(n);
\}
\end{Codes}

And in fact the Java spec formally recognizes this equivalence, and
writes, ``The meaning of the enhanced for statement is given by
translation into a basic for statement.''~\cite[section 14.14.2]{java8} Ignoring some
irrelevant details, it states that this \emph{sugar}:
\begin{Codes}
for (<type> <var> : <expr>) \{ <statement> \}
\end{Codes}
\emph{desugars} into:
\begin{Codes}
for (Iterator i = <expr>.iterator(); i.hasNext(); ) \{
  <type> <var> = (<type>) i.next();
  <statement>
\}
\end{Codes}
Here, the things we have written in angle brackets are
\emph{parameters} to the sugar: they are pieces of code that it
takes as arguments.


\paragraph{But what \emph{is} it?}
An astute reader may have noticed that we still haven't actually
defined syntactic sugar. Here is a reasonable definition:\marginpar{%
  Words can be thought of as pointing to clusters in concept-space.
  An extensional definition like this is an attempt to draw a
  neat box around such a cluster, which is always a little dubious
  because clusters typically have fuzzy boundaries and aren't
  box-shaped.
}
\begin{quote}
  A syntactic construct in an implementation of a programming language
  is \emph{syntactic sugar} if it is translated at compile-time into
  the syntax of the rest of the language.
\end{quote}
%% Note: definition must (i) rule out functions, (ii) rule out
%%   metaprogramming, (iii) include Racket macros, which are non-local
%%   and non-phase-specific.]
%% \begin{quote}
%%   'A construct in a language is called ``syntactic sugar'' if it can be
%%   removed from the language without any effect on what the language
%%   can do: functionality and expressive power will remain the same.' (Wikipedia)
%% \end{quote}
Thus syntactic sugar splits a language into two parts: a (small) core
language and a rich set of usable syntax atop that core. In this
thesis, we use the term \emph{surface} to refer to the language the
programmer sees, and \emph{core} for the target of desugaring.
This makes desugaring a subset of \emph{compilation}: it is
compilation from a language to a subset of that language. It also
makes clear that \emph{macros} are a special case of syntactic sugar:
they are a way of allowing users of a language to
define syntactic sugar within the language itself.
% Also: macros are a DSL for writing a compiler
%   https://www.cs.utah.edu/plt/publications/macromod.pdf


\section{What is it good for?}

Syntactic sugar is an essential component of programming languages and
systems, and it is now actively used in many practical settings:
\begin{itemize}
\item In the definition of language constructs in many languages
  ranging from Java to Haskell.
\item To allow users to extend the language, in languages ranging from the Lisp
  family to C++ to Julia to Rust.
\item To shrink the semantics of large scripting languages with many
  special-case behaviors, such as JavaScript and Python, to small core
  languages that tools can more easily process~\cite{lambda-js,politz:s5,politz:python}.
\end{itemize}
Desugaring allows a language to expose a rich surface syntax, while
compiling down to a small core. Having a smaller core reduces the
cognitive burden of learning the essence of the language. It also
reduces the effort needed to write tools for the language or do proofs
decomposed by program structure (such as type soundness proofs). Thus,
heaping sugar atop a core is a smart engineering trade-off that ought
to satisfy both creators and users of a language. Observe that this
trade-off does not depend in any way on desugaring being offered as a
surface linguistic feature (such as macros).


\section{When should you use it?}

Syntactic sugar is used to define abstractions. But languages have
other ways to define abstractions already: functions, classes, data
definitions, etc. If an abstraction can be implemented using these
features, it's almost always better to do so, because developers are
already deeply familiar with them. Thus:
\begin{quote}
  Syntactic sugar should only be used to implement an abstraction if
  it cannot be implemented in the core language directly.
\end{quote}

Therefore, to find places where it is a \emph{good} idea to use sugar,
we should look for things that most programming languages
\emph{cannot} abstract over:
\begin{enumerate}
  \item In most languages, variable names are first order and cannot
    be manipulated at run-time (e.g., a variable cannot be passed as
    an argument to a function: if you attempt to do so, the value the
    variable is bound to will be passed instead). Therefore, creating
    new binding constructs is a good use for syntactic sugar in most
    languages. However, in R~\cite{rlang}, variable names can be abstracted
    over (e.g. ``\Code{assign("x", 3); x}'' prints 3), and so sugar isn't
    necessary for this purpose.
  \item Most languages use eager evaluation,
    which forces the arguments to a function to be
    evaluated when it is called, rather than allowing the function to
    choose whether to
    evaluate them or not. Thus if an abstraction requires delaying
    evaluation, it is a good candidate to be a sugar. However, Haskell
    has lazy evaluation, and thus does not need sugars for this purpose.
  \item Most languages cannot manipulate data definitions at run-time:
    e.g., field names cannot be dynamically constructed. Thus creating
    new data definition constructs (e.g., a way to define state
    machines) is a good use case for sugars. However, in Python, field
    names are first class (e.g., they can be added or assigned using
    \Code{setattr}), so it does not need sugars to abstract over fields
    in data definitions.%e.g. \Code{class C: pass; c = C(); setattr(c, ``x'', 3); print(c.x)}
    \footnote{
    It may sound like we are suggesting that everything should be
    manipulatable at run-time. We are not. The more
    things which are fixed at compile time (variable names, field
    names, etc.), the more (i) programmers can reason about their
    programs; (ii) tools can reason about programs; (iii) compilers
    can optimize programs (without herculean effort). It is
    \emph{good} that sugar is sometimes necessary.
  }
\end{enumerate}

Overall, syntactic sugar is a way to \emph{extend a language}.
In a limited sense, this is what functions are for as well.
Functions, however, are limited: they cannot take a variable as an
argument, delay evaluation, introduce new syntax, etc. This is where
sugar shines.


\section{What are its downsides?}

There are two sets of downsides to syntactic sugar. The first
set of downsides applies to languages that allow user-defined
syntactic sugar (i.e., macros), and arises from the powerful nature of
syntactic sugar. Sugars can manipulate many things
such as control flow and variable binding that are otherwise fixed in
a language. As a result, badly written sugars can lead to code that is
convoluted in ways not otherwise possible.  Furthermore, syntactic
sugar (by design) \emph{hides} details: programmers only ever see the
sugar, and not the desugared code (except perhaps in error messages),
and thus may not be fully aware of what sort of code they are implicitly
writing. In short, it can be dangerous to allow users to extend a
language, and this is exactly what macros allow. However, this is a
language design issue and thus outside the scope of this thesis.

Instead, we address situtions in which the abstraction provided by
sugar leaks~\cite{leaky-abstractions}. The code
generated by desugaring can be large and complicated, creating an
onerous comprehension burden; it may even use features of the core
language that the user does not know. Therefore, programmers using
sugar must not be forced to confront the details of the sugar; they should
only confront the core language when they use it directly. This is a
concern irrespective of whether sugars are defined as part of the
language or whether users can define their own sugars.

We call out three particular ways that syntactic sugar breaks abstractions:

\begin{description}
\item[Evaluation Steps] Syntactic sugar obscures the evaluation steps
  the program takes when it runs, since these evaluation steps happen
  in the core language rather than the surface language the program
  was written in.
\item[Scope Rules] In a similar manner, syntactic sugar obscures the
  (implicitly defined) scope rules for a surface language.
\item[Type Rules] Finally, in typed languages with syntactic sugar,
  type error messages frequently reveal the desugared code (which the
  programmer didn't write), thus breaking the sugar's abstraction.
\end{description}

\section{How can these (particular) downsides be fixed?}

We address these three problems with a general approach called
\emph{resugaring}. Thus my \textbf{thesis statement} is that:\marginpar{%
While I use ``we'' throughout the rest of document to acknowledge
all the others who helped with this work, my thesis statement is for
me to defend. Hence ``my''.
}
\begin{quote}
Many aspects of programming languages---in particular evaluation
steps, scope rules, and type rules---can be non-trivially
\emph{resugared} from core to surface language, restoring the
abstraction provided by syntactic sugar.
\end{quote}

We hope that this work will give desugaring its rightful
place in the programming language space, freeing future implementers
and researchers to take full advantage of sugar in their semantics
and systems work. Only when using syntactic sugar no longer
inadvertently breaks abstractions can the adage
``oh, that's \emph{just} syntactic sugar'' finally become true.

\section{Roadmap}

The rest of this thesis is organized as follows:\marginpar{%
  Related work is dicussed individually in chapters
  \cref{chap:resugar-eval,chap:resugar-scope,chap:resugar-types}.
}
\paragraph{Part II: Desugaring}
\begin{description}
\item[\Cref{chap:taxonomy}] discusses and taxonomizes existing
  desugaring systems (of which there are a wide variety).
\item[\Cref{chap:notation}] provides notation that
  will serve as groundwork for the rest of the chapters.
% [TODO] \item[\Cref{chap:well-formedness}] provides an algorithm for
% checking that desugaring rules preserve syntactic well-formedness.
\end{description}
\paragraph{Part III: Resugaring}
\begin{description}
\item[\Cref{chap:resugar-eval}] shows how to resugar \emph{evaluation sequences}.
\item[\Cref{chap:resugar-scope}] shows how to resugar \emph{scope rules}.
\item[\Cref{chap:resugar-types}] shows how to resugar \emph{type rules}.
% [TODO] \item[\Cref{chap:implementation}] discusses an implementation of
%  evaluation sequence resugaring and scope resugaring for the Pyret
%  (pyret.org) language.
% \item[\Cref{chap:conclusion}] concludes and discusses future work.
\end{description}

Related work is discussed in each resugaring chapter, wherever it is most relevant.
