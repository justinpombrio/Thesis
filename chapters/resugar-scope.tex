\chapter{Resugaring Scope Rules}\label{chap:resugar-scope}

In this chapter, we show how to lift scoping rules defined on a core
language to rules on the surface, a process of \emph{scope inference}.
In the process we introduce a new representation of binding
structure---scope as a preorder---and present a theoretical advance:
proving that a desugaring system preserves $\alpha$-equivalence even
though scoping rules have been provided \emph{only} for the core
language.

This chapter comes from work published in ICFP 2017 under the title
\emph{Inferring Scope through Syntactic Sugar} (co-authored with
Shriram Krishnamurthi and Mitchell Wand)~\cite{pombrio-scope}.

\section{Introduction}

Traditionally,
scoping rules are defined on the core language, not on the
surface. However, many tools depend on source representations. For
instance, editors need to know the surface language's scoping in order
to perform auto-complete, distinguish free from bound variables, or draw
arrows to show bound and binding instances. Likewise, refactorers need
to know binding structure to perform correct transformations.
These tools are hard to construct if scoping is only known
for the core language.

Many tools that exploit binding information for the source
do so by desugaring the program and obtaining its binding in the core
language (this, for instance, is the approach used by
DrRacket~\cite{drscheme} for overlaying binding arrows on the
source).
However, this approach is far from ideal. It requires tools to be able
to desugar programs and to resolve binding in the core
language. This is an intimate level of knowledge of a language, though:
syntactic sugar is supposed to be an abstraction, so external tools
should ideally be unaware that a language even \emph{has} syntactic
sugar. Additionally, this approach fails completely if the source
program cannot be desugared because it is incomplete or syntactically
invalid (as programs are most of the time while editing). It is
therefore better to disentangle the editor from the language,
providing the editor precisely what it needs: scoping rules for the
surface language.

We therefore present a static inference process that, given a
specification of syntactic sugar and scoping rules on a core language,
\emph{automatically constructs} scoping rules for the surface
language. The inferred rules are guaranteed to give the same binding
structure to a surface program as that program would have in the core language
after desugaring (\cref{thm:rscope-resugar}).
Essentially, scope inference ``pushes scope back through the sugar''. We
can think of this as statically lifting a ``lightweight semantics'' of
the language. Thus it is a precursor to lifting other notions of semantics
(such as lifting type-checking rules, as we do in the next chapter)
though of course the mechanics of doing so will depend heavily on the
semantics itself.

The intended application of this work is as follows:
\begin{enumerate}
\item Begin with a core language with known scoping rules, and a set
  of pattern-based desugaring rules. (We give a formal description of
  scope in \cref{sec:rscope-sap}, and a language for specifying scope in
  \cref{sec:rscope-rules}.)
\item Infer surface language scoping rules from the core scoping
  rules. (We give a scope inference algorithm in
  \cref{sec:rscope-resugar}, and show how to make it hygienic in
  \cref{sec:rscope-hygiene}.)
\item Add these inferred scoping rules to various tools that can exploit
  them (Sublime, Atom, CodeMirror, etc.).
\end{enumerate}

An alternative approach would be to specify scoping rules for the
surface language, and verify that they are consistent with the core
language. This approach has been advocated for
scoping~\cite{herman-hygiene,stansifer-romeo},
type systems~\cite{typechecking-exts},
and formal semantics~\cite{ziggurat}.
However, this assumes that language developers are always programming
language experts who are knowledgeable about binding, able to verify consistency, and willing
to do this extra work. These are particularly unsafe assumptions for
domain-specific languages, which we believe are a strong use case for
our technique.

\subsection*{Contributions and Outline}
\begin{description}
\item[Modelling Scope] In \cref{sec:rscope-sap}, we give a formal
  description of scope as a \emph{preorder} 
  (which we motivate through examples in
  \cref{sec:rscope-example}).
  This preorder then defines the name
  binding structure of a program, such as where variable
  references are bound, and which variable declarations shadow others.
\item[Binding Specification Language] In \cref{sec:rscope-rules}, we
  present a \emph{binding specification language}, i.e., a language
  for specifying the name binding structure of a programming
    language. This specification makes it possible to compute the
    scope structure (a preorder) of concrete programs in that language.
  \item[Scope Inference] In \cref{sec:rscope-resugar}, we show how to
  \emph{infer} these scoping rules through syntactic sugar. This is our main contribution. We describe our
  implementation and provide case studies in
  \cref{sec:rscope-impl}, and prove that---given reasonable
  assumptions---desugaring after scope inference will be hygienic in
  \cref{sec:rscope-hygiene}.
\end{description}


\section{Two Worked Examples}\label{sec:rscope-example}

We will begin by building up to our scope inference technique via two
worked examples.
\textbf{They are slightly simplified for expository purposes}. 
\Cref{sec:rscope-rules} describes the generalization, and
\cref{sec:rscope-for-example,sec:rscope-named-let-example}
provide examples.
(While the generalization is sometimes important, it has no effect on the
examples of this section.)

\subsection{Example: Single-arm Let}
For the first example, consider a simple Let construct that allows
only a single binding:
\begin{Table}
  $e$ &$::=$& $\NodeRm{Let}{\DeclX\;e_1\;e_2}$
  & ``Let $\DeclX$ equal $e_1$ in $e_2$'' \\
  &$|$& $\ldots$
\end{Table}
(Here the superscript \textsc{d} indicates that this
occurrence of the variable $\VarX$ is a declaration of $\VarX$.
In general, we
will distinguish \emph{declarations}, i.e., binding sites, from
\emph{references}, i.e., use sites.)

This Let may be desugared to Apply and Lambda by the following desugaring rule, which
we will write using s-expressions:
\begin{LongTable}
  $\NodeRm{Let}{\PVarA\;\PVarB\;\PVarC}$
  &$\To$&
  $\NodeRm{Apply}{\NodeRm{Lambda}{\PVarA\;\PVarC}\;\PVarB}$
\end{LongTable}

Now suppose that we know the scoping rules of Apply and Lambda, and
wish to derive what the scoping rules for Let must be, given
the desugaring rule and assuming the language is statically and lexically
scoped. More precisely, we wish to find a scoping rule
for Let such that the desugaring rules \emph{preserve binding structure}
(and thus neither cause variable capture nor cause
variables to become unbound).

The first step will be to write down what we know about the
scope on the \textsc{rhs} (right hand side) of the rule.
Pictorially, we might draw:
\begin{center}
\sideLabel{\textsc{rhs}}{
\begin{tikzScopeDiagram}[simple]
  \tikzRoot
    {A}{\tikzParentTwo{Apply}
      {B}{\tikzParentTwo{Lambda}
        {C}{\tikzChild{$\PVarA$}}
        {D}{\tikzChild{$\PVarC$}}}
      {G}{\tikzChild{$\PVarB$}}}
  \begin{tikzEdges}
    \tikzEdgeL{C}{D};
  \end{tikzEdges}
\end{tikzScopeDiagram}
}
\end{center}
where the dotted lines show the tree structure of the \textsc{ast}, and
where the teal/solid arrow means that the Lambda's parameter ($\PVarA$) can be
used in its body ($\PVarC$).
Similarly, there are no arrows among the children
of Apply because function application does not introduce any binding.

We also know from lexical scope that any declarations in scope at a
node in an \textsc{ast} should also be in scope at its children. This
can be denoted with upward arrows:
\begin{center}
\sideLabel{\textsc{rhs}}{
\begin{tikzScopeDiagram}[simple]
  \tikzRoot
    {A}{\tikzParentTwo{Apply}
      {B}{\tikzParentTwo{Lambda}
        {C}{\tikzChild{$\PVarA$}}
        {D}{\tikzChild{$\PVarC$}}}
      {E}{\tikzChild{$\PVarB$}}}
  \begin{tikzEdges}
    \tikzEdgeL{C}{D};
    \tikzEdge{A}{B};
    \tikzEdge{A}{E};
    \tikzEdge{B}{C};
    \tikzEdge{B}{D};
  \end{tikzEdges}
\end{tikzScopeDiagram}
}
\end{center}
In general, the meaning of the arrows is that a variable declaration
is in scope at every part of the program which has a (directed) path to it.
(In the case of variable shadowing, the outer declaration is in scope
at the inner declaration, which in turn is in scope at some region;
references in this region will be bound to the dominating inner declaration.)

Now we can begin to infer what the scope must look like on the
\textsc{lhs} (left hand side) of the desugaring rule. We want the rule to
\emph{preserve} binding, therefore there should be a path from one
pattern variable to another in the \textsc{lhs} iff there is a similar path in the
\textsc{rhs}. If there was a path from $\PVarA$ to $\PVarB$ in the
\textsc{lhs} but not in the \textsc{rhs}, that would mean that a
variable (in $\PVarA$) that used to be bound (by $\PVarB$) could
become unbound. Likewise, if there was a path between two pattern variables in the
\textsc{rhs} but not in the \textsc{lhs}, that could result in
unwanted variable capture.

Thus, since there is a path from $\PVarC$ to $\PVarA$ in the rule's
\textsc{rhs}, there must also be a path from $\PVarC$ to $\PVarA$ in the
\textsc{lhs}. This gives:
\begin{center}
  \sideLabel{\textsc{lhs}}{
  \begin{tikzScopeDiagram}[simple]
    \tikzRoot
      {A}{\tikzParentThree{Let}
        {B}{\tikzChild{$\PVarA$}}
        {C}{\tikzChild{$\PVarB$}}
        {D}{\tikzChild{$\PVarC$}}}
    \begin{tikzEdges}
      \tikzEdgeL{B}{D};
    \end{tikzEdges}
  \end{tikzScopeDiagram}
  }
\end{center}
In English, this arrow says that the variable declared at $\PVarA$ is
in scope at the Let's body $\PVarC$, as expected.

There are still some missing arrows, however: there should be down
arrows to indicate that any declaration in scope at the Let should
also be in scope at its children. These can be inferred in a similar
way: whenever there is a path from the root to a pattern variable on the
\textsc{rhs}, there should be a similar path on the \textsc{lhs}.
Since on the \textsc{rhs} there are paths to each pattern variable from the root,
the same should hold true on the \textsc{lhs}:
\begin{center}
\sideLabel{\textsc{lhs}}{
\begin{tikzScopeDiagram}[simple]
  \tikzRoot
    {A}{\tikzParentThree{Let}
      {B}{\tikzChild{$\PVarA$}}
      {C}{\tikzChild{$\PVarB$}}
      {D}{\tikzChild{$\PVarC$}}}
  \begin{tikzEdges}
    \tikzEdgeL{B}{D};
    \tikzEdge{A}{B};
    \tikzEdge{A}{C};
    \tikzEdge{A}{D};
  \end{tikzEdges}
\end{tikzScopeDiagram}
}
\end{center}
This gives a complete scoping rule for this Let construct.


\subsection{Example: Multi-arm Let*}
\label{sec:rscope-example2}

Next, take a more involved example: a multi-armed Let* construct (in
the style of Lisp/Scheme/Racket).
It will have the following grammar:
\begin{Table}
  $e$ &$::=$& $\NodeRm{Let*}{b\;e}$
  & ``Let-bind $b$ in $e$'' \\
  &$|$& $\ldots$ \\
  $b$ &$::=$& $\NodeRm{Bind}{\DeclX\;e\;b}$
  & ``Bind $\DeclX$ to $e$, and bind $b$'' \\
  &$|$&   $\ConstRm{EndBinds}$
  & ``No more bindings''
\end{Table}
This grammar separates out the Let's bindings into nested
subterms.\marginpar{%
We call the bindings just ``Bind'', even though they are specific
  to Let. If a language has other forms of binding as well, ``Bind'' may need a more
  specific name such as ``LetBind''.
}
It is
necessary to do this if more complex binding patterns are allowed,
such as arbitrarily deep pattern-matching.

Let* can then be implemented with two desugaring rules:
\begin{LongTable}
$\NodeRm{Let*}{\NodeRm{Bind}{\PVarA\;\PVarB\;\PVarC}\;\PVarD}$
$\To$
$\NodeRm{Apply}{\NodeRm{Lambda}{\PVarA\;\NodeRm{Let*}{\PVarC\;\PVarD}}\;\PVarB}$
\\ \\
$\NodeRm{Let*}{\ConstRm{EndBinds}\;\PVarA}$
$\To$
$\NodeRm{Begin}{\PVarA}$
\end{LongTable}
These rules would, for example, make the following transformation:
\begin{Table}
$\NodeRm{Let*}{\NodeRm{Bind}{\DeclX\;1\;\NodeRm{Bind}{\DeclY\;2\;\ConstRm{EndBinds}}}\;\NodeRm{Plus}{\RefnX\;\RefnY}}$ \\
\quad $\To$
$\NodeRm{Apply}{\NodeRm{Lambda}{\DeclX\;\NodeRm{Apply}{\NodeRm{Lambda}{\DeclY\;\NodeRm{Plus}{\RefnX\;\RefnY}}\;2}}\;1}$
\end{Table}

Given that we know the scoping rules of Apply, Lambda, and Begin,
we can use them to derive the scoping rules for Let* and Bind. The scoping for
the second rule is trivial, so we will concentrate just on the first rule.

As before, the first step is to write down what we know about the
scope on the \textsc{rhs}:
\begin{center}
\sideLabel{\textsc{rhs}}{
\begin{tikzScopeDiagram}[simple]
  \tikzRoot
    {A}{\tikzParentTwo{Apply}
      {B}{\tikzParentTwo{Lambda}
        {C}{\tikzChild{$\PVarA$}}
        {D}{\tikzParentTwo{Let*}
          {E}{\tikzChild{$\PVarC$}}
          {F}{\tikzChild{$\PVarD$}}}}
      {G}{\tikzChild{$\PVarB$}}}
  \begin{tikzEdges}
    \tikzEdgeL{C}{D};
    \tikzEdge{A}{B};
    \tikzEdge{A}{G};
    \tikzEdge{B}{C};
    \tikzEdge{B}{D};
    \tikzEdge{D}{E};
    \tikzEdge{D}{F};
  \end{tikzEdges}
\end{tikzScopeDiagram}
}
\end{center}
Unlike in the previous example, this diagram is not (necessarily)
complete, since we don't yet know the scoping rule for
Let*. (This will happen when desugaring rules use recursion.)
We have drawn two upward arrows on Let*, despite the fact that we
don't yet know its scoping rule: technically, these arrows should
(and can) be inferred, but we start with them to simplify this example.

Now we can begin to infer what the scope must look like on the
\textsc{lhs}. As before, we want the rule to preserve
binding. Thus, since the \textsc{rhs} has a path from $\PVarC$ to
$\PVarA$ and from $\PVarD$ to $\PVarA$, the same must be true in the
\textsc{lhs} (labeling the arrows for reference):
\begin{center}
\sideLabel{\textsc{lhs}}{
\begin{tikzScopeDiagram}[simple]
  \tikzRoot
    {A}{\tikzParentTwo{Let*}
      {B}{\tikzParentThree{Bind}
        {C}{\tikzChild{$\PVarA$}}
        {D}{\tikzChild{$\PVarB$}}
        {E}{\tikzChild{$\PVarC$}}}
      {F}{\tikzChild{$\PVarD$}}}
  \begin{tikzEdges}
    \tikzEdgeRR{C}{E}[][c];
    \tikzEdge{C}{B}[][b];
    \tikzEdgeL{B}{F}[][a];
  \end{tikzEdges}
\end{tikzScopeDiagram}
}
\end{center}
Notice that we drew the path from $\PVarD$ to $\PVarA$ with \emph{two}
arrows. This is because we will assume that scoping rules are local,
relating only terms and their immediate children.

We have now learned something about the scoping rules for Let* and
Bind! When read in English, these three arrows say that:
\begin{enumerate}
\item[a.] Declarations from a Let*'s binding list are visible in its body.
\item[b.] A Bind's variable declaration is provided by the Bind (so
  that it can be used by the Let*).
\item[c.] A Bind's variable declaration is visible to later Binds in the
  binding list.
\end{enumerate}

This information can now be applied to fill in the previously
incomplete \textsc{rhs} picture. Arrow (a) represents a
fact about the scoping of \emph{every} Let*, so it must also apply in the
\textsc{rhs} (highlighting it orange/dashed for exposition):
\begin{center}
\sideLabel{\textsc{rhs}}{
\begin{tikzScopeDiagram}[simple]
  \tikzRoot
    {A}{\tikzParentTwo{Apply}
      {B}{\tikzParentTwo{Lambda}
        {C}{\tikzChild{$\PVarA$}}
        {D}{\tikzParentTwo{Let*}
          {E}{\tikzChild{$\PVarC$}}
          {F}{\tikzChild{$\PVarD$}}}}
      {G}{\tikzChild{$\PVarB$}}}
  \begin{tikzEdges}
    \tikzEdgeL{C}{D};
    \tikzEdge{A}{B};
    \tikzEdge{A}{G};
    \tikzEdge{B}{C};
    \tikzEdge{B}{D};
    \tikzEdge{D}{E};
    \tikzEdge{D}{F};
    \tikzEdgeL{E}{F}[color=arrowColorHL,dashed];
  \end{tikzEdges}
\end{tikzScopeDiagram}
}
\end{center}
Adding this arrow introduces a path from $\PVarD$ to $\PVarC$,
however, that needs to be reflected back at the \textsc{lhs}!
\begin{center}
\sideLabel{\textsc{lhs}}{
\begin{tikzScopeDiagram}[simple]
  \tikzRoot
    {A}{\tikzParentTwo{Let*}
      {B}{\tikzParentThree{Bind}
        {C}{\tikzChild{$\PVarA$}}
        {D}{\tikzChild{$\PVarB$}}
        {E}{\tikzChild{$\PVarC$}}}
      {F}{\tikzChild{$\PVarD$}}}
  \begin{tikzEdges}
    \tikzEdgeRR{C}{E}[][c];
    \tikzEdge{C}{B}[][b];
    \tikzEdgeL{B}{F}[][a];
    \tikzEdge{E}{B}[color=arrowColorHL,dashed][d];
  \end{tikzEdges}
\end{tikzScopeDiagram}
}
\end{center}
In general, the algorithm is to monotonically add arrows until
reaching the least fixpoint. In this particular case, arrow $d$ is the last
fact to be inferred:
\begin{enumerate}
\item[d.] A Bind also provides any declarations provided by later Binds
  in the binding list.
\end{enumerate}
This concludes the interesting facts to be inferred about the scoping
rules for Let* and Bind. We have ignored the upward arrows that reflect
lexical scope from parent to child for simplicity, but these can
be inferred by the same process.


\subsection{Scope as a Preorder}

In the two preceding examples, we have expressed the scope of a program
diagrammatically with arrows. When reasoning about scope, it will be
helpful to be able to \emph{transcribe} these diagrams into a
textual form.

To do so, recall the (approximate) meaning of the arrows:\marginpar{%
We make the meaning of arrows precise in \cref{sec:rscope-sap-defs}.
}
a declaration is in scope at every part of the program which has a
(directed) path to it, and is shadowed by declarations of the same name
that have a path to it. Thus the arrows are only meaningful insofar as
they produce paths. Furthermore, paths have two important properties:
\begin{enumerate}
  \item They are closed under reflexivity: there is always an (empty) path
    from $a$ to $a$.
  \item They are closed under transitivity: if there is a path from
    $a$ to $b$ and a path from $b$ to $c$, then there is a path from
    $a$ to $c$.
\end{enumerate}
These are also precisely the properties that define a \emph{preorder}.
Thus, we will transcribe scope diagrams as preorders, writing $a \< b$
when there is a path from $a$ to $b$.
For example, in the (incomplete) diagram we inferred for the \textsc{lhs} of the
multi-arm Let* sugar:
\begin{center}
\begin{tikzScopeDiagram}[simple]
  \tikzRoot
    {A}{\tikzParentTwo{Let*}
      {B}{\tikzParentThree{Bind}
        {C}{\tikzChild{$\PVarA$}}
        {D}{\tikzChild{$\PVarB$}}
        {E}{\tikzChild{$\PVarC$}}}
      {F}{\tikzChild{$\PVarD$}}}
  \begin{tikzEdges}
    \tikzEdgeRR{C}{E};
    \tikzEdge{C}{B};
    \tikzEdgeL{B}{F};
    \tikzEdge{E}{B};
  \end{tikzEdges}
\end{tikzScopeDiagram}
\end{center}
The corresponding preorder is:
$\hspace{2em} \PVarD \< \ConstRm{Bind} \< \PVarC \< \PVarA$



\section{Describing Scope as a Preorder}
\label{sec:rscope-sap}

We have informally described the notion of scope as a preorder,
primarily using diagrams. In this section, we will describe it
formally. First, however, we need to lay down some
starting assumptions and basic definitions.

\subsection{Basic Assumptions}
\label{sec:rscope-prelim}

We will make a number of assumptions to make reasoning about scope
more tractable:
  \begin{itemize}
  \item We only deal with scoping that is \emph{static} and \emph{lexical}.
  \item Scoping rules will be as local as possible, only relating a
    term to its \emph{immediate} children. Longer relationships will
    be achieved by transitivity.\marginpar{%
      If we allowed non-local arrows, then in the previous example,
      inference would produce a single arrow from $\PVarD$
      to $\PVarA$ instead of arrows ``a'' and ``b''. Then the orange/dashed arrow
      could \emph{not} be inferred, since it relied on the existence
      of arrow ``a'', and the inference process would fail at its task.
      }
  \item We work on an \textsc{ast}, instead of directly on the surface
    syntax. As mentioned in \cref{sec:formal-term},
    variable references (use sites) and declarations
    (binding sites) must be syntactically distinguished in this
    \textsc{ast}.
  \item Each kind of term has a fixed arity. (Indefinite arity is
    possible using a list encoding, as in Let* above.)
\end{itemize}

The last two of these assumptions are already reflected in our
definition of (\textsc{ast}) terms in \cref{sec:formal-term}.
For convenience, we repeat that definition here:

\begin{Table}
constructor $C$ &$::=$& \textit{name} & syntactic construct name \\
term $e$ &$::=$& $\Value$ & primitive value \\
  &$|$& $\Node{C}{e_1 \dd e_n}$ & \Sc{ast} node \\
  &$|$& $\Refn[i]{x}$  & variable reference at position $i$ \\
  &$|$& $\Decl[i]{x}$  & variable declaration at position $i$ \\
\end{Table}
To reiterate, references and declarations have both a name \textsc{x}
(as written in the source), and an \textsc{ast} position $i$
(that uniquely distinguishes them).
Occasionally it will be useful to refer to a variable which could be
either a declaration or a reference: in this case
we omit the superscript, e.g. $\VarXi$.
Likewise, we will omit the position subscript $i$
when there is no ambiguity. We will also sometimes write $C$ in place
of $\Node{C}{e_1\,...\,e_n}$ when there is no danger of ambiguity.

By the last assumption above, we do not support lists
$[e_1 \dd e_n]$ (which have indefinite arity). Instead we require them
to be encoded into fixed-arity constructs.

\subsection{Basic Definitions}
\label{sec:rscope-sap-defs}

  We define scope in terms of a perorder.
  A \emph{preorder} ($\<$) is a reflexive and transitive relation. It
  need not be anti-symmetric, however, so it is possible that $a \< b$
  and $b \< a$ for distinct $a$ and $b$.
We capture scope as a preorder on a term $e$ as follows:

\begin{definition}[Scope]\label{def:rscope-scope}
  A \emph{scope preorder} on a term $e$ is a preorder ($\<$) on the references
  and declarations in $e$ such that references are least in this
  preorder (i.e., nothing is ever smaller than a reference).
\end{definition}
\begin{definition}
  A reference $\RefnXi$ is \emph{in scope of} a declaration
  $\DeclXj$ when $\RefnXi \< \DeclXj$.
\end{definition}
\begin{definition}
  A declaration $\DeclXi$ is \emph{more specific than}
    another $\DeclXj$ when $\DeclXi \< \DeclXj$.
\end{definition}
Note that these definitions rely on the \emph{existence} of a preorder
($\<$), but don't say how to determine it for a given term. We will present
\emph{scoping rules} to do so in \cref{sec:rscope-rules}.
These definitions therefore provide very little on their own, but they can be built upon to define
some common concepts:

\begin{definition}[Bound]
  \label{def:rscope-bound}
  A reference is \emph{bound} by the most specific declaration(s) that
  it has the same name as and is in scope of. More formally, we write:
  \[ \RefnX \Bound \DeclX \Defeq
  \DeclX \in \min \St{\DeclXi}{\RefnX \< \DeclXi} \]
  where $\min S$ finds the (zero or more) least elements of $S$:
\[ \min S \Defeq \St{a \in S}{
  \NotExists{b \in S}
    b \< a \text{ and } a \not\< b}
\]
\end{definition}

\begin{definition}[Unbound]
A reference is \emph{unbound} (or \emph{free}) when it is not bound by
any declaration.
\end{definition}

\begin{definition}[Ambiguously Bound]\label{def:rscope-ambig}
  A variable reference is \emph{ambiguously bound} when it is bound by
  more than one declaration.
\end{definition}

Ambiguous binding may occur, for instance, if two declarations
have the same name and are both parameters to the same function.
In this case, a reference in the body of the function would be
ambiguously bound to both of them.
We also capture the idea of \emph{shadowing}, where a more specific
declaration hides a less specific declaration:

\begin{definition}[Shadowing]
  \label{def:rscope-shadow}
  \marginpar{%
We use the same notation $\square \Bound \square$ for both binding
and shadowing because the definitions are analogous.}
  A declaration \emph{shadows} the most specific declarations that it
  has the same name as and is more specific than. Formally, $\DeclXi$
  shadows $\DeclXj$ when:
  \[ \DeclXi \Bound \DeclXj \Defeq
  \DeclXj \in \min \St{\DeclXk}{i \neq k \text{ and } \DeclXi \< \DeclXk}
  \]
\end{definition}

\subsection{Validating the Definitions}
Since this description of scope is new, readers might wonder
whether our definitions of concepts match their vernacular meaning.
We provide evidence that they do in three forms.

First, we prove a simple lemma below showing that shadowing
behaves as one would expect.
Second, we show (\cref{sec:rscope-sos}) that the notion of scope used in
``Binding as Sets of Scopes''~\cite{flatt:scope} obeys our {\sap}
definitions, for an appropriate choice of preorder $(\<)$.
Finally, we introduce a second, very intuitive definition of
scope called {\sas}, and show that it is equivalent to {\sap} up to
a normalization.

\begin{lemma}[Shadowing]
  If one declaration shadows another, then a reference in scope of the
  shadowing declaration cannot be unambiguously bound by the shadowed declaration.
\end{lemma}
\begin{proof}
  Suppose that $\DeclXj$ shadows $\DeclXi$ ($\DeclXj$ is the shadowing
  declaration and $\DeclXi$ is the shadowed declaration), and
  $\RefnXk$ is in scope of $\DeclXj$.
  By definition, $\RefnXk$ will be bound by
  $\min \St{\DeclXl}{\RefnXk \< \DeclXl}$.  But $\RefnXk \<
  \DeclXj \< \DeclXi$, so $\DeclXi$ cannot be in this set (assuming
  that $\RefnXk$ is not ambiguously bound), and $\RefnXk$
  cannot be bound by $\DeclXi$.
\end{proof}

\subsection{Relationship to ``Binding as Sets of Scopes''}
\label{sec:rscope-sos}
{\Sap} aligns with the notion of scope expressed by Flatt~\cite{flatt:scope}.
In his formulation,
each subterm in the program is labeled with a \emph{set of scopes},
called its \emph{scope set}. A reference's binding (i.e., declaration)
is then found ``as one whose set of scopes is a subset of the
reference's own scopes (in addition to having the same symbolic
name)''. If there is more than one such declaration, a reference is
bound by the one with the largest (superset-wise) scope set. If there
is no unique such element, then the reference is
``ambiguous''~\cite[pp. 3]{flatt:scope}.

This can be expressed in terms of {\sap}. Take the preorder
$(\<)$ to be (the least relation such that):
\begin{Table}
$\RefnXi \< \RefnXi$ \\
$\RefnX \< \DeclY$   iff $\Scopeset{\RefnX} \supseteq \Scopeset{\DeclY}$ \\
$\DeclXi \< \DeclYj$ iff $\Scopeset{\DeclXi} \supseteq \Scopeset{\DeclYj}$
\end{Table}
Then our definition of a reference's binding agrees with Flatt's, and
our definition of an ``ambiguously bound'' reference agrees with his
definition of an ``ambiguous'' reference.

\subsection{Axiomatizing Scope as Sets}
\label{sec:rscope-sas}

In this section, we describe an alternative axiomatization of scope,
called \emph{\sas}.
In {\sas}, instead of there being a preorder over variables, each
declaration has a \emph{scope}, which is the part of the program in
which it can be referenced. For example, the scope of a function's
parameter is that function's body. The axioms for a term $e$ are then:

\begin{description}
\item[Scope]
  Each declaration in $e$ has a \emph{scope}---written
  $\Scope{\DeclX}$---given by the parts of the program in which it can
  be referenced.
  We will say that:
  \begin{itemize}
  \item A reference $\RefnX$ is \emph{in scope of} a declaration
    $\DeclX$ when $\RefnX \in \Scope{\DeclX}$.
  \item One declaration $\DeclX$ is \emph{more specific than} another
    $\DeclY$ when $\Scope{\DeclX} \subseteq \Scope{\DeclY}$.
  \end{itemize}
\item[Binding]
  A reference is \emph{bound} to the most specific declaration that
  it has the same name as and is in scope of, provided such a unique
  element exists. Again, if there is no (unique) most specific
  declaration, we will say that the reference is \emph{ambiguously
    bound}.
%% \item[Renamability]
%%   The scope of a declaration is determined solely by the shape
%%   of the term it is in (and, in particular, not by the name of any
%%  variable).
\end{description}

These axioms differ from the {\sap} definitions in that the relations
``in scope of'' and ``more specific than'' are defined differently.
However, all of the other definitions of \cref{sec:rscope-sap}
(including binding, shadowing, ambiguous declarations, etc.)
are based on the relations ``in scope of'' and
``more specific than''. Thus all of those definitions apply just as
well to {\sas}.

Furthermore, the two axiomatizations of scope we have presented are closely
related. In fact, {\sap} is simply a normalized form of {\sas}.
To begin with, either form can be converted to the other.

\paragraph{\textit{Conv1}: From {\sap} to {\sas}}
Given a preorder ($\<$), define its conversion into a {\sas}
function $\Scope{}$ by:
\[ \Scope{\DeclY} \Defeq
   \St{\RefnX}{\RefnX \< \DeclY}
   \cup
   \St{\DeclX}{\DeclX \< \DeclY} \]

\paragraph{\emph{Conv2}: From {\sas} to {\sap}}
Given a {\sas} function $\Scope{}$, define its conversion into a
preorder ($\<$) by letting ($\<$) be the least relation such that:
\begin{Table}
  $\RefnXi \< \RefnXi$ \\
  $\RefnX \< \DeclY$
  &when& $\RefnX \in \Scope{\DeclY}$ \\
  $\DeclX \< \DeclY$
  &when& $\Scope{\DeclX} \subseteq \Scope{\DeclY}$
\end{Table}

Both of these conversions preserve all of the binding concepts of
\cref{sec:rscope-sap-defs}:

\begin{lemma}[Binding Preservation] \label{thm:rscope-preservation}
  Both $\ConvA{}$ and $\ConvB{}$ preserve ``in scope of'' and ``more
  specific than'' in both directions
  (and thus all of the concepts of \cref{sec:rscope-rules}).
\end{lemma}
\begin{proof}
  Under both conversions, ``in scope of'' is preserved:
  \begin{Table}
    \hphantom{iff}\; $\RefnX$ in scope of $\DeclY$ (under {\SAS}) \\
    iff\; $\RefnX \in \Scope{\DeclY}$ \\
    iff\; $\RefnX \< \DeclY$ \\
    iff\; $\RefnX$ in scope of $\DeclY$ (under {\SAP})
  \end{Table}

  And under both conversions, ``more specific than'' is preserved:
  \begin{Table}
    \hphantom{iff}\;
          $\DeclX$ more specific than $\DeclY$ (under {\SAS}) \\
    iff\; $\Scope{\DeclX} \subseteq \Scope{\DeclY}$ \\
    iff\; $\DeclX \< \DeclY$ \\
    iff\; $\DeclX$ more specific than $\DeclY$ (under {\SAP})
  \end{Table}
\end{proof}

Furthermore, these conversions are inverses in one direction: from
{\sap} to {\sas} back to {\sap}. This implies that there is a normal
form for {\sas}, given by $\ConvA{\ConvB{\Scope{}}}$, such that the
conversions are exact inverses whenever this normal form is used.

We first show that the conversions are inverses in one direction:
\begin{lemma}[Inverses1] \label{lemma:rscope-inverses}
  For every scope preorder ($\<$),
  $\ConvB{\ConvA{\<}} = (\<)$
\end{lemma}
\begin{proof}
  \begin{Table}
    $\RefnXi \< \RefnXj$
    &when& $i = j$ \\
    $\RefnX \< \DeclY$
    &when& $\RefnX \in \Scope{\DeclY}$ \\
    && iff $\RefnX \in \St{\RefnX}{\RefnX \< \DeclY}$ \\
    && iff $\RefnX \< \DeclY$ \\
    $\DeclX \< \DeclY$
    &when& $\Scope{\DeclX} \subseteq \Scope{\DeclY}$ \\
    && iff $\DeclX \< \DeclY$
  \end{Table}
\end{proof}

And they're inverses in the other direction when $\Scope{}$ is in a
\emph{normal form}. Specifically, we will say that a
{\sas} function $\Scope{}$ is in \emph{normal form} when:
\begin{Table}
  &(i)&  $\Forall{\DeclX} \Scope{\DeclX}$ contains only variables \\
  &(ii)& $\Forall{\DeclX} \Forall{\DeclY}
    \DeclX \in \Scope{\DeclY}
    \text{ iff } \Scope{\DeclX} \subseteq \Scope{\DeclY}$
\end{Table}

\begin{lemma}[Inverses2]
  For any {\sas} function $\Scope{}$ in normal form,\\ %NEWLINE
  $\ConvA{\ConvB{\Scope{}}} = \Scope{}$.
\end{lemma}
\begin{proof}
  \begin{LongTable}
    && $\Scope{\DeclY}$ \\
    &$=$& $\St{\RefnX}{\RefnX \< \DeclY} \cup
           \St{\DeclX}{\DeclX \< \DeclY}$
           &($\ConvA{}$) \\
    &$=$& $\St{\RefnX}{\RefnX \in \Scope{\DeclY}} \cup$ \\
       && $\St{\DeclX}{\Scope{\DeclX} \subseteq \Scope{\DeclY}}$
           &($\ConvB{}$) \\
    &$=$& $\St{\RefnX}{\RefnX \in \Scope{\DeclY}} \cup
           \St{\DeclX}{\DeclX \in \Scope{\DeclY}}$
           &(by (ii)) \\
    &$=$& $\Scope{\DeclY}$ &(by (i))
  \end{LongTable}
\end{proof}

The normal form of a {\sas} scope function can be computed as:
\[ \Norm{\Scope{}} = \ConvA{\ConvB{\Scope{}}} \]

\begin{lemma}
  For any {\sas} scope function $\Scope{}$, $\Norm{\Scope{}}$ is in
  fact in normal form.
\end{lemma}
\begin{proof}
  The first requirement---that the range of $\Scope{}$ only contains
  variables---is immediately fulfilled by $\ConvA{}$. The second
  requirement follows from the definitions of $\ConvA{}$ and $\ConvB{}$: \\
  $\DeclX \in \Scope{\DeclY}$
  iff $\DeclX \< \DeclY$
  iff $\Scope{\DeclX} \subseteq \Scope{\DeclY}$
\end{proof}

Putting a {\sas} scope function in normal form preserves its binding
structure. Furthermore, once it is in normal form, converting it to
{\sap} (and back) have no effect.

\begin{lemma}
  $\Norm{}$ preserves ``in scope of'' and ``more specific than''
  (and thus all of the concepts of \cref{sec:rscope-rules}).
\end{lemma}
\begin{proof}
  Follows directly from \cref{thm:rscope-preservation}.
\end{proof}

This concludes the demonstration that {\sap} is simply a normalized
version of {\sas}, and can effectively be used in its place. The
axioms of {\sas} are basic enough that they ought to apply in
basically any lexically-scoped setting; thus {\sap} should too.
We use {\Sap} in this thesis, however, because it is more canonical (per
normalization) and because it more closely aligns with the binding
language described next.



\section{A Binding Specification Language}
\label{sec:rscope-rules}

The previous section presented definitions for \emph{representing} the
scoping of a term. It did not, however, say how to \emph{determine}
the scoping of a term, i.e., what the specific preorder should be. In
this section, we give a language for specifying \emph{scoping rules}
that, given a term, determine a preorder over its variables.

\subsection{Scoping Rules: Simplified}

The basic idea behind our binding language is that the binding structure of
a term should be determined piecewise by its subterms. Thus every term
constructor (e.g., Lambda or Bind) should specify a \emph{scoping rule}
that gives a preorder amongst itself and its children. A term's {\sap}
can then be found by taking the transitive closure of these local
preorders across the whole term.

As an example, take the term
$\NodeRm{Lambda}{\DeclX[1]\;\NodeRm{Plus}{\RefnX[2]\;3}}$.  To find
the bindings of this term, we must know the scoping rules for Lambda
and Plus. A sensible rule for Plus is that a term
$\NodeRm{Plus}{\PVarA\;\PVarB}$ has preorder
$\PVarA \< \NodeRm{Plus}{\PVarA\;\PVarB}$
and $\PVarB\< \NodeRm{Plus}{\PVarA\;\PVarB}$,
meaning that whatever a Plus term is in scope of, its
children are too.
For brevity, we will typically write $\PVarA \< \ConstRm{Plus}$
and $\PVarB\< \ConstRm{Plus}$ instead.
Likewise, a sensible rule for Lambda is that a term
$\NodeRm{Lambda}{\PVarA\;\PVarB}$ has preorder $(\PVarB \< \PVarA
\< \ConstRm{Lambda})$,
meaning that whatever a Lambda term is in scope of, its
children are too, and that $\PVarB$ (its body) is in
scope of $\PVarA$ (its declaration). Put together, and applied to the
example term, these rules give that:
\[  \NodeRm{Lambda}{\DeclX[1]\;\NodeRm{Plus}{\RefnX[2]\;3}} \]
has preorder:
\[
  \RefnX[2], 3 \<
  \ConstRm{Plus} \< \DeclX[1] \< \ConstRm{Lambda}
\]
Thus $\RefnX[2] \Bound \DeclX[1]$ by \cref{def:rscope-bound}, as it should be.


\subsection{A Problem}

This isn't quite the whole picture, though. Consider the term
\[
\NodeRm{Lambda}{\DeclX[1]\;
  \NodeRm{Let*}{
    \NodeRm{Bind}{\DeclX[2]\;\RefnX[3]\;\ConstRm{EndBinds}}\;
    \RefnX[4]}}
\]

What will these scoping rules look like? Whatever they are, they
should cause $\DeclX[2]$ to shadow $\DeclX[1]$, $\RefnX[3]$ to be
bound by $\DeclX[1]$, and $\RefnX[4]$ to be bound by $\DeclX[2]$.
Formally, we should have:
\[ \DeclX[2] \Bound \DeclX[1] \text{ and }
   \RefnX[3] \Bound \DeclX[1] \text{ and }
   \RefnX[4] \Bound \DeclX[2]
\]
which implies that, at a minimum:
\[ \DeclX[2] \< \DeclX[1] \text{ and }
   \RefnX[3] \< \DeclX[1] \text{ and }
   \RefnX[4] \< \DeclX[2]
\]

This places a set of requirements on the scoping rules for Lambda,
Let*, and Bind. For instance, $\RefnX[3] \< \DeclX[1]$ can only be
achieved if $\RefnX[3] \< \ConstRm{Bind} \< \ConstRm{Let*} \< \DeclX[1]$.
Continuing this way gives the requirements (shown both pictorially and
textually):

\begin{scopeDescription}
\begin{center}
\begin{tikzScopeDiagram}[simple]
  \tikzRoot
    {A}{\tikzParentTwo{Lambda}
      {B}{\tikzChild{$\DeclX[1]$}}
      {C}{\tikzParentTwo{Let*}
        {D}{\tikzParentThree{Bind}
          {E}{\tikzChild{$\DeclX[2]$}}
          {F}{\tikzChild{$\RefnX[3]$}}
          {G}{\tikzChild{EndBinds}}}
        {H}{\tikzChild{$\RefnX[4]$}}}}
      
  \begin{tikzEdges}
    \tikzEdgeL{B}{C};
    \tikzEdge{C}{D};
    \tikzEdgeL{D}{E};
    \tikzEdge{D}{F};
    \tikzEdgeL{E}{D};
    \tikzEdgeL{D}{H};
  \end{tikzEdges}
\end{tikzScopeDiagram}
\end{center}
\begin{Table}
  \ConstRm{Let*} &$\<$& $\DeclX[1]$ \\
  \ConstRm{Bind} &$\<$& \ConstRm{Let*} \\
  $\RefnX[4]$    &$\<$& \ConstRm{Bind} \\
  $\DeclX[2]$    &$\<$& \ConstRm{Bind} \\
  $\RefnX[3]$    &$\<$& \ConstRm{Bind} \\
  \ConstRm{Bind} &$\<$& $\DeclX[2]$
\end{Table}
\end{scopeDescription}

However, this puts $\RefnX[3]$ in scope of $\DeclX[2]$, and as a
result, $\RefnX[3]$ will be bound by $\DeclX[2]$! The problem is that
Bind is trying to provide $\DeclX[2]$, to make it available in the
body of Let*, but in doing so it incidentally makes it available in the
Bind's definition (to $\RefnX[3]$). This is not how scoping
dependencies should flow, and in the next two subsections we present
the full, un-simplified version of our scoping rules that avoid this
problem.

\subsection{The Solution}
\label{sec:rscope-the-solution}

The solution is to separate the bindings a term \emph{imports} (i.e.,
requires) from the bindings it \emph{exports} (i.e., provides). In the
running example, for instance, the Bind \emph{imports} $\DeclX[1]$,
and \emph{exports} $\DeclX[1]$ and $\DeclX[2]$ (with $\DeclX[2]$
shadowing $\DeclX[1]$). We will call imports and exports \emph{ports}.

The scoping rules can now be re-interpreted with this in mind. Given a
term $e$, they will determine a preorder not over the subterms of $e$
(like we have presented it so far), but instead over the \emph{ports}
of the subterms of $e$. With this in mind, we offer four kinds of bindings:\footnote{
  There is a close analogy between ports and attributes in attribute
  grammars~\cite{knuth-attribute-grammar}:
  namely, imports are analogous to inherited
  attributes and exports are analogous to synthesized attributes.
  The paths between imports and exports that are allowed by our
  binding language (e.g., child export to parent export, but not
  child export to parent import) are precisely the relationships
  between inherited and synthesized attributes that are allowed in
  attribute grammars. Most algorithms for evaluating attribute
  grammars disallow cycles, however, while our preorders allow them.
}
\begin{enumerate}
\item[A.] \SpecBind{j}{i}: A term may make its $i$'th child's bindings
  available in its $j$'th child. If so, any declarations
  \emph{exported} by child $i$ will be \emph{imported} by child $j$.
\item[B.] \SpecImpt{i}: A term's $i$'th child may import its
  parent's declarations. If so, it \emph{imports} the declarations \emph{imported}
  by its parent. (This is almost universal:
  declarations in scope at a node in an \textsc{ast} should
  also be in scope at its children. However, we do allow a term to
  hide all bindings from its child, if it so desires.)
\item[C.] \SpecExpt{i}: A term's $i$'th child may export its
  declarations to its parent. If so, the term \emph{exports} child $j$'s
  \emph{exports}.
\item[D.] \SpecSelf: A term may take the declarations it
  \emph{imported}, and \emph{export} them. (This is not terribly useful
  in practice, but we offer it for completion.)
\end{enumerate}
These four kinds of paths may be represented graphically,
showing imports as $\ImSymbCirc$ and exports as $\ExSymbCirc$:

\begin{center}
\begin{tikzScopeDiagram}
  \tikzRoot
    {A}{\tikzParentTwo{Parent}
      {B}{\tikzChild{Child1}}
      {C}{\tikzChild{Child2}}}
  \begin{tikzEdges}
    \tikzEdge{A-}{B-}[][B];
    \tikzEdgeLL{B+}{C-}[][A];
    \tikzEdge{C+}{A+}[][C];
    \tikzEdgeRRR{A-}{A+}[][D];
  \end{tikzEdges}
\end{tikzScopeDiagram}
\end{center}

With these new bindings in mind, the requirements for the example from
the previous subsection become:

\begin{scopeDescription}
\begin{center}
\begin{tikzScopeDiagram}
  \tikzRoot
    {A}{\tikzParentTwo{Lambda}
      {B}{\tikzChild{$\DeclX[1]$}}
      {C}{\tikzParentTwo{Let*}
        {D}{\tikzParentThree{Bind}
          {E}{\tikzChild{$\DeclX[2]$}}
          {F}{\tikzChild{$\RefnX[3]$}}
          {G}{\tikzChild{EndBinds}}}
        {H}{\tikzChild{$\RefnX[4]$}}}}

  \begin{tikzEdges}
    \tikzEdgeLL{B+}{C-};
    \tikzEdge{C-}{D-};
    \tikzEdge{D-}{E-};
    \tikzEdge{D-}{F-};
    \tikzEdgeRRRR{E-}{E+};
    \tikzEdge{E+}{D+};
    \tikzEdgeLL{D+}{H-};
  \end{tikzEdges}
\end{tikzScopeDiagram}
\end{center}
\begin{LongTable}
  \\ \\
  $\im{}$\ConstRm{Let*}       &$\<$& $\ex{}\DeclX[1]$ \\
  $\im{}$\ConstRm{Bind}      &$\<$& $\im{}$\ConstRm{Let*}  \\
  $\im{}\RefnX[4]$ &$\<$& $\ex{}$\ConstRm{Bind} \\
  $\im{}\DeclX[2]$ &$\<$& $\im{}$\ConstRm{Bind} \\
  $\im{}\RefnX[3]$ &$\<$& $\im{}$\ConstRm{Bind} \\
  $\ex{}\DeclX[2]$ &$\<$& $\im{}\DeclX[2]$ \\
  $\ex{}$\ConstRm{Bind}      &$\<$& $\ex{}\DeclX[2]$
\end{LongTable}
\end{scopeDescription}

Under this new preorder, $\RefnX[3] \Bound \DeclX[1]$ and $\RefnX[4]
\Bound \DeclX[2]$ as desired.


\subsection{Scoping Rules: Unsimplified}

We have given an intuition behind our scoping rules;
now we present them formally.

Each port will have one of two \emph{signs} (import or export):
\begin{Table}
$d$ &$::=$& $\ImSymb$ & (import) \\
   &$\mid$& $\ExSymb$ & (export)
\end{Table}
A \emph{port}, then, pairs a term $e$ with a sign:
\begin{Table}
$a,b,c$ &$::=$& $\im{}e \;|\; \ex{}e$ &(port)
\end{Table}

A set of \emph{scope rules} $\Sigma$ gives a relation
for each term constructor $C$ that describes the scoping relationships
between a term constructed with $C$ and its subterms:
\begin{definition} \label{def:rscope-scope-rules}
  A set of \emph{scope rules} $\Sigma$ is a partial map from term
  constructors $C$ of arity $n$ to binary relations over
  $\{1,\,...,\,n,\Exp,\Imp\}$, such that:
  \begin{itemize}
  \item The relation is transitive.
  \item $\Exp$ is a least element ($\NotExists{a} (a,\,\Exp) \in \Sigma[C]$)
  \item $\Imp$ is a greatest element ($\NotExists{a} (\Imp,\,a) \in \Sigma[C]$)
  \end{itemize}
\end{definition}
Here $i$ represents the $i$th child term,
$\Exp$ represents the parent term's exports, and
$\Imp$ represents the parent term's imports.
We will call pairs in the relation (e.g., $(1,\,\Imp)$) \emph{facts},
and will equate them with their description in our binding
language (so that $(1,\,\Imp) = \SpecImpt{1}$).
The sign on the port on $i$ can be determined knowing that the fact it is part of must be
one of the four kinds of bindings described in \cref{sec:rscope-the-solution}.
We will write $e' \Subterm e$ to mean that $e'$ is a subterm of $e$,
and write $a \Subterm e$ to mean
$\Exists{e'} (a=\im{e'} \text{ or } a=\ex{e'}) \text{ and } e' \Subterm e$.

As an example of scope rules, the rules for Lambda are:
\[
\Sigma[\text{Lambda}] = \{(1,\,\Imp),\,(2,\,\Imp),\,(2,\,1)\} = \{\SpecImpt{1},\,\SpecImpt{2},\,\SpecBind{2}{1}\}
\]

These scope rules determine the scoping for individual (sub)terms.
The scoping of a \emph{full} term is found by applying the scoping rules
locally at each subterm, then taking the reflexive transitive closure
of this global relation:
\begin{definition}
The \emph{scoping} of a full term $e$ under scoping rules $\Sigma$
is the set of judgements of the form $\SaysScope{e}{a}{b}$
defined by the ``Declarative Rules'' and ``Shared Rules'' of 
\cref{fig:rscope-SD}.
\end{definition}

The judgments in the figure have the form
$\SaysScope{e}{a}{b}$, which means that
``$a \< b$ in term $e$ using scoping rules $\Sigma$''.
A judgment is \emph{well formed} when $a, b \Subterm e$.
(Later, we will also use judgments of the form $\SaysScope{p}{a}{b}$; these
are governed by identical rules, allowing each term $e$ to instead be
a pattern $p$.)

Rules SD-Import, SD-Export, SD-Bind, and S-ReExport capture the
direct meaning of the scoping rules.
S-Refl, S-Refl2, and SD-Trans give the transitive reflexive
closure.
SD-Decl allows declarations to extend the current scope.
S-Lift says that facts learned about a subterm remain true in
the whole term.

\begin{figure}[ht]
  \begin{flushleft}
    \hspace{5em}\TypeLabel{\SaysScope{e}{a}{b}}
  \end{flushleft}
  \[ \textbf{Declarative Rules} \]
  \[
  \inference[SD-Trans]
      {\SaysScope{e}{a}{b} \quad \SaysScope{e}{b}{c}}
      {\SaysScope{e}{a}{c}}
  \]

  \[
  \inference[SD-Import]
      {\SigPImpt{i}}
      {\SaysScope{\NodeStd}{\im e_i}{\im \NodeStd}}
  \]

  \[
  \inference[SD-Export]
      {\SigPExpt{i}}
      {\SaysScope{\NodeStd}{\ex{}\NodeStd}{\ex e_i}}
  \]
  \[
  \inference[SD-Bind]
      {\SigPBind{i}{j}}
      {\SaysScope{\NodeStd}{\im e_i}{\ex e_j}}
  \]

  \[ \textbf{Shared Rules (Declarative \& Algorithmic)} \]
  \[
  \inference[S-Refl1]
      {}
      {\SaysScope{e}{\im e}{\im e}}
  \qquad
  \inference[S-Refl2]
      {}
      {\SaysScope{e}{\ex e}{\ex e}}
  \]

  \[    
  \inference[S-Lift]
      {\SaysScope{e_i}{a}{b}}
      {\SaysScope{\NodeStd}{a}{b}}
  \]

  \[
  \inference[S-Decl]
      {}
      {\SaysScope{\DeclX}{\ex{}\DeclX}{\im{}\DeclX}}
  \]
  \[
  \inference[S-ReExport]
      {\SigPSelf}
      {\SaysScope{\NodeStd}{\ex{}\NodeStd}{\im{}\NodeStd}}
  \]

  \[ \textbf{Algorithmic Rules} \]
  \[
  \inference[SA-Import]
      {\SaysScope{e_i}{a}{\im e_i} \quad
       \SigPImpt{i}}
      {\SaysScope{\NodeStd}{a}{\im{}\NodeStd}}
  \]
  \[
  \inference[SA-Export]
      {\SigPExpt{i} \quad
       \SaysScope{e_i}{\ex e_i}{a}}
      {\SaysScope{\NodeStd}{\ex{}\NodeStd}{a}}
  \]

  \[
  \inference[SA-Bind]
      {\SaysScope{e_i}{a}{\im e_i} \quad
       \SigPBind{i}{j} \quad
       \SaysScope{e_j}{\ex e_j}{b}}
      {\SaysScope{\NodeStd}{a}{b}}
  \]

\caption{Scope Checking}
\label{fig:rscope-SD}
\end{figure}

These rules are not, however, syntax-directed. We give a
syntax-directed version of the rules in the figure, under
``Algorithmic Rules'' and ``Shared Rules''.
These two rule sets are equivalent:

\begin{theorem}[Algorithmic Scope Checking]\label{thm:rscope-rules}
The declarative and algorithmic scope
checking rules (\cref{fig:rscope-SD}) [with shared rules common to both]
are equivalent.
\end{theorem}
  \begin{proof}\label{proof:rscope-rules}
  We will show that the Algorithmic (SA-) and Declarative (SD-) rules
  are equivalent by giving translations in both directions
  (omitting the symbol $\Sigma$ throughout for brevity).
  \cref{fig:rscope-rules-proof-1,fig:rscope-rules-proof-4} give the translations.
  The translations make use of the fact that the scoping rule relations
  are transitive.
  
  The conversion from SA to SD is straightforward.
  However, the conversion from SD to SA is more difficult,
  as it has to handle the many ways the SD-Trans rule can be used.
  It proceeds by recursively pushing SD-Trans toward the leaves of the
  derivation.
  
  The following table shows which inference rules can occur above
  SD-Trans (and are thus included in
  \cref{fig:rscope-rules-proof-1,fig:rscope-rules-proof-3}),
  and which are impossible (and thus not included):
  \begin{center}
  \begin{tabular}{r | c c c c c}
    left \textbackslash\ right & re-export & export & import & bind & lift \\
    \hline
   re-export & \xmark & \xmark & \xmark & \xmark & \xmark
  \\ export  & \xmark & \xmark & \checkmark & \checkmark & \checkmark
  \\ import  & \xmark & \xmark & \xmark & \xmark & \xmark
  \\ bind    & \xmark & \xmark & \checkmark & \checkmark & \checkmark
  \\ lift    & \xmark & \xmark & \checkmark & \checkmark & \checkmark
  \end{tabular}
  \end{center}
\end{proof}

These scope checking rules say how to find a preorder over all of the
\emph{ports} in a term.
However, \cref{sec:rscope-sap} is based only on preorders over the
\emph{variables} in a term. In fact, {\sap} could be used with a
\emph{different} binding language, so long as it can be used to
extract a preorder.

This is obtained as the restriction of
the entire preorder to variables, as captured by the following rule:
\[
  \TypeLabel{\SaysScope{e}{\VarX}{\VarY}}
  \hspace{3em}
  \inference[S-Var]
     {\SaysScope{e}{\im{}\VarXi}{\im{}\VarYj}}
     {\SaysScope{e}{\VarXi}{\VarYj}}
  \hspace{6em}
\]

  The definitions for binding and shadowing
  (\cref{def:rscope-bound,def:rscope-shadow}) can then be expressed as inference rules:
  \[
    \TypeLabel{\SaysScopeBound{e}{\VarX}{\VarY}}
    \hspace{3em}
    \inference[S-Bound]
        {\DeclX \in \min
          \St{\DeclXi}{\SaysScope{e}{\RefnX}{\DeclXi}}}
        {\SaysScopeBound{e}{\RefnX}{\DeclX}}
    \hspace{5em}
  \]
  \[
    \hspace{5em}
    \inference[S-Shadow]
        {\DeclX[j] \in \min \St{\DeclX[k]}{i \neq k \text{ and } \SaysScope{e}{\DeclX[i]}{\DeclX[k]}}}
        {\SaysScopeBound{e}{\DeclX[i]}{\DeclX[j]}}
  \]
  These definitions form a scope preorder:
\begin{lemma}
  For any set of scoping rules $\Sigma$ and term $e$, the relation
  $\St{(\VarXi, \VarXj)}{\SaysScope{e}{\VarXi}{\VarXj}}$ is a scope
  preorder satisfying the requirements of \cref{def:rscope-scope}.
\end{lemma}
  \begin{proof}[Proof sketch]
      The relation is a preorder by the derivation rules S-Refl1,
      S-Refl2, and SD-Trans. We must also show that references
      are least. Suppose instead that $\VarXi \< \RefnXj$ for some
      $i \neq j$. Then $\im{}\VarXi \< \im{}\RefnXj$ (by S-Var), which is
      syntactically impossible to achieve by the declarative judgements.
  \end{proof}


\subsection{Well-Boundedness}

\Cref{def:rscope-bound} (on being bound) can be used to define $\alpha$-equivalence. Two
terms are $\alpha$-equivalent if (i) each term is ``well-bound'';
(ii) they have the same ``shape'' (i.e.,
they are identical ignoring their variable names); and (iii) for every
binding $\RefnX \Bound \DeclX$ in one term, an analogous binding
exists in the same location in the other term. To formalize what
``same location'' means, we will use a \emph{join} operator
($e \bowtie e'$) that checks that $e$ and $e'$ have the same shape and
finds a bijection between their variable occurrences as a witness to
this fact:
\begin{LongTable}
  $\DeclXi$ &$\bowtie$& $\DeclYj$ &$=$& $\{\DeclXi \leftrightarrow \DeclYj\}$ \\
  $\RefnXi$ &$\bowtie$& $\RefnYj$ &$=$& $\{\RefnXi \leftrightarrow \RefnYj\}$ \\
  $\mathit{const}$ &$\bowtie$& $\mathit{const}$ &$=$& $\emptyset$ \\
  $\Node{C}{\Seqn[1]{e}[n]}$ &$\bowtie$& $\Node{C}{\Seqn[1]{e'}[n]}$
  &$=$& $\bigcup_{i \in 1..n} e_i \bowtie e_i'$ \\
  $e$ &$\bowtie$& $e'$           &$=$& \textsc{undefined} &otherwise
\end{LongTable}

Likewise, to formalize ``well-bound'', we will use the rules to
determine when two declarations
\emph{conflict}; for instance if they have the same name and are both
parameters to the same function.
We will consider terms with conflicting declarations to be ill-bound.
\begin{definition}[Conflicting Declarations]
  \label{def:rscope-conflict}
  Two variable declarations $\DeclXi$ and $\DeclXj$ \emph{conflict} in
  a term $e$ when:
  \[
  \inference[S-Conflict]
    { \SaysScope{e}{a}{\DeclXi}
      \quad \SaysScope{e}{a}{\DeclXj} \\
      \min_{\Sigma,e} \{ \DeclXi, \DeclXj \} = \{ \DeclXi, \DeclXj \}}
    { \SaysScopeConflict{e}{\DeclXi}{\DeclXj} }
  \]
\end{definition} \noindent
(If a variable reference is \emph{ambiguously bound}
(\cref{def:rscope-ambig}), then its bindings declarations must be in conflict.)

A term $e$ is \emph{well-bound} with respect to scoping rules $\Sigma$
when every
reference is bound by exactly one declaration, and there are no
conflicting declarations:
\[
\inference[S-WB]
          { \Forall{\RefnX \!\in\! e} \ExistsUnique{\DeclX \!\in\! e}
            \SaysScopeBound{e}{\RefnX}{\DeclX} \\
            \NotExists{\DeclX[i],\DeclX[j] \!\in\! e}
            \SaysScopeConflict{e}{\DeclX[i]}{\DeclX[j]}}
          { \SaysScopeWB{e} }
\]

The definition of $\alpha$-equivalence with respect to the scoping rules
$\Sigma$ is then:
\begin{definition}[$\alpha$-equivalence]
  \label{def:rscope-eqa}
\[
  \inference[S-$\alpha$-Eqv]
      { \SaysScopeWB{e} \quad \SaysScopeWB{e'} \quad e \bowtie e' = \psi \\
        \Forall{\RefnX,\DeclX}{\SaysScopeBound{e}{\RefnX}{\DeclX} \text{ iff }
        \SaysScopeBound{e'}{\psi(\RefnX)}{\psi(\DeclX)}}}
      {\SaysScopeEqa{e}{e'}}
\]
(We will also talk about $\alpha$-equivalence and well-boundedness of
patterns. The definitions are identical.)
\end{definition}

In \cref{sec:rscope-catalog}, we show a catalog of scoping rules
that can be expressed in our binding language.

%% % Interpretation
%% % (An intermediate formulation that simplifies the above)
%% \begin{Table}
%% $\interpSign[d]{0}{\Node{n}{\Seqn[1]{e}[k]}}$
%%   &$=$& $\port{n}{\negport{d}}$ \\
%% $\interpSign[d]{i}{\Node{n}{\Seqn[1]{e}[k]}}$
%%   &$=$& $\port{e_i}{d}$
%%   &when $i > 0$
%% \end{Table}
%% \begin{Table}
%% ... \\
%% $\interpSign[\im]{j}{s, \seq{e_i}}$
%%   &$\<$& $\interpSign[\ex]{k}{s, \seq{e_i}}$
%%   &for each $s = \Node{n}{\seq{e_i}} \in e$ \\
%% &&&and $[j] \<: [k] \in \sign{n}$ \\
%% ...
%% \end{Table}



\section{Inferring Scope}
\label{sec:rscope-resugar}

In this section we show how to infer scope by lifting scoping
rules from a core language to the surface language.
The input to this inference process is twofold: first, the
core language must have associated scoping rules, and second, the
syntactic sugar must be given as a set of pattern-based rewrite rules.
The output of scope inference is a set of scoping rules for the surface
language.

The process is loosely analogous to type inference:
type inference finds the most general type annotations such that a
program type-checks; scope inference will find the smallest set of
surface scoping rules under which desugaring preserves $\alpha$-equivalence.
More precisely, given a core language with scoping rules $\SigmaCore$,
and a desugaring $\Desugar{}$, our algorithm finds scoping rules $\SigmaSurf$
that preserves $\alpha$-equivalence (\cref{thm:rscope-hygiene}), so that:
\[ \SaysScopeEqa[\SigmaSurf]{e}{e'} \text{\quad implies\quad }
   \SaysScopeEqa[\SigmaCore]{\Desugar{e})}{\Desugar{e'})}
\]
Furthermore, $\SigmaSurf$ will be least, so that if
$\SigmaSurf'$ also has this property, then
$\Forall{C} \SigmaSurf[C] \subseteq \SigmaSurf'[C]$.

The general algorithm for scope inference is given in
\cref{fig:rscope-resugar}. The next three subsections explain our assumptions
about desugaring, and then the algorithm.

\begin{figure*}
\begin{LongTable}
  $\displaystyle \Op{inferScope}{\Sigma,\,\{p_i \To p_i'\}_{i \in 1..n}}$
  &$\Defeq$&
  Let $\SigmaSurf = \displaystyle \Op{solve}{\Sigma,\,
    \bigcup_{i \in 1..n} \Op{genConstrs}{p_i \To p_i'}}$ \\
  && $\Op{checkScope}{\SigmaSurf,\,\{p_i \To p_i'\}_{i \in 1..n}}$ \\
  && Return $\SigmaSurf$
  \vspace{0.9em} \\
  $\displaystyle \Op{genConstrs}{p \To p'}$
  &$\Defeq$&
  $\displaystyle \left\{
  \Op{genConstr}{a \< b,\;p \To p'} \right\}_{a,b \in H}$ \\
  &&where $H = \PVars{p} \cup \PVars{p'} \cup \{\Root\}$
  \vspace{0.9em} \\
  $\Op{genConstr}{a \< b,\; p \To p'}$
  &$\Defeq$&
  $\left(\Op{genConj}{a \< b,\; p}, \Op{genConj}{a \< b,\; p'}\right)$ \\
  &&where $(p, q)$ means a constraint ``$p$ iff $q$''
  \vspace{0.9em} \\
  $\Op{genConj}{a \< b,\; p}$
  &$\Defeq$&
  Smallest $\SigmaSurf$ such that $\SaysScope{p}{a}{b}$. \\
  && (To compute this, take the premises of
     the (unique) \\
  && derivation of $\SaysScope{p}{a}{b}$ using the \\
  && Algorithmic Scope Checking rules (\cref{fig:rscope-SD}).)
  \vspace{0.9em} \\
  $\displaystyle \Op{solve}{\SigmaCore,\,constraints}$
  &$\Defeq$&
  \parbox[t][][t]{0.8\linewidth}{
    Initialize $\SigmaSurf$ = $\SigmaCore$ \\
    Until fixpoint:
    \begin{itemize}
    \item If a fact $F$ in a constraint is in $\SigmaSurf$: \\
      \Indent Delete $F$ from the constraint
    \item If one side of a constraint is empty: \\
      \Indent Delete the constraint \\
      \Indent Add the other side to $\SigmaSurf$ \\
      \Indent\Indent (maintaining transitive closure)
    \item If any fact in $\SigmaSurf$ is in the complement of $\SigmaCore$: \\
      \Indent ERROR: Reject this sugar
    \end{itemize}
    Return $\SigmaSurf$
    }
  \vspace{0.9em} \\
  $\displaystyle \Op{checkScope}{\Sigma,\,\{p_i \To p_i'\}_{i \in 1..n}}$
  &$\Defeq$&
  For each rule $p \To p'$: \\
  &&\quad Assert that if $\SaysScope{p}{a}{b}$
          and $a \in p'$ then $b \in p'$ \\
          && \Indent (otherwise ERROR) \\
  &&\quad Assert that each reference $\RefnX \in p'$ is bound by a \\
          && \Indent unique declaration $\DeclX \in p'$
\end{LongTable}
\caption{Scope Inference Algorithm}
\label{fig:rscope-resugar}
\end{figure*}

\subsection{Assumptions about Desugaring} \label{sec:rscope-des-assumptions}

To briefly recap, we assume that desugaring is given by a set of
rewrite rules of the form $p \To p'$, where $p$ and $p'$ are patterns:
\begin{Table}
pattern $p$ &$::=$& $\Value$ & primitive value \\
  &$|$& $\Node{C}{p_1 \dd p_n}$ & \Sc{ast} node \\
  &$|$& $\Refn[i]{x}$  & variable reference \\
  &$|$& $\Decl[i]{x}$  & variable declaration
\end{Table}
Additionally, references and declarations in the \Sc{rhs} of a rule
must be given fresh names during expansion.  Finally, desugaring must
obey the requirements of \cref{sec:formal-reqs}, with one exception:
the \Sc{lhs}s of rules need not be disjoint.

When desugaring, there may be more than one rewrite rule that applies
to a given term. \emph{None of the results of this chapter depend on
the order of the rewrites; even a non-deterministic desugaring is allowed.}
A more typical choice is to apply rules in outside-in order, as
described in \cref{sec:formal-desugar}, and as is done by Scheme-style
\Code{syntax-rules} macros~\cite{scheme5}.

In general, a rewrite will look like:
\[ p_0[p[e_1,...,e_n]] \To p_0[p'[e_1,...,e_n]] \]
where $p_0$ and $p$ are patterns, and $p[e_1,...,e_n]$ denotes
replacing the $n$ pattern variables of pattern $p$ with
terms $e_1,...,e_n$. (In \cref{sec:rscope-example}, $p$ was
called the \textsc{lhs}, and $p'$ the \textsc{rhs}.)
The outer pattern $p_0$ is important
because when a piece of sugar expands, while its \emph{expansion}
doesn't typically depend on its surrounding context, its binding
structure might. For example, $p_0$ might be $\NodeRm{Lambda}{\DeclX\;\PVar}$,
and $\RefnX$ inside the pattern variable may be unbound without it.


\subsection{Constraint Generation}
\label{sec:rscope-constr}

The first step to scope inference is generating a set of constraints
for each desugaring rule that, if satisfied, ensure that it will
preserve binding structure. 
Specifically, fix a rewrite rule $p \To p'$.
It is important that this rewrite does not change the binding of any
variable \emph{outside} of $p$. To achieve this, it will suffice that
the preorder on the \emph{boundary} of $p$ is the same as the
preorder among the boundary of $p'$. The boundary, here, is the set of
pattern variables in $p$, together with the root (i.e., the whole term).
For example, in $p_0[p[e_1,...,e_n]]$, $\PVar_i$ bounds $e_i$, and $p$ (the root)
bounds $p_0$. In general, we will call this property
\emph{scope-equivalence}:
\begin{definition}[Scope-equivalence of patterns] \WhitePhantom{.}\\ %NEWLINE
  $\SaysScopeEqv{p}{p'}$ means that
  $\Forall{a, b \in \{\PVar_1,...,\PVar_n,\Root\}}$
  \[ \SaysScope{p}{a}{b} \text{ iff } \SaysScope{p'}{a}{b} \]
  where $\Root$ (``root'') stands in for $p$ or $p'$, as appropriate,
  and omitted port signs are determined by what our binding language allows:
  \begin{Table}
    $\SaysScope{p}{\PVar_i}{\PVar_j}$ &$\Defeq$&
    $\SaysScope{p}{\im{}\PVar_i}{\ex{}\PVar_j}$
    \\
    $\SaysScope{p}{\PVar_i}{\Root}$ &$\Defeq$&
    $\SaysScope{p}{\im{}\PVar_i}{\im{}p}$
    \\
    $\SaysScope{p}{\Root}{\PVar_j}$ &$\Defeq$&
    $\SaysScope{p}{\ex{}p}{\ex{}\PVar_j}$
    \\
    $\SaysScope{p}{\Root}{\Root}$ &$\Defeq$&
    $\SaysScope{p}{\ex{}p}{\im{}p}$
  \end{Table}
\end{definition}

When two patterns are scope-equivalent, rewriting one to the other
within a term does not change the scope of the rest of the term:
\begin{definition}[Scope-preservation]
  A rewrite
  \[ p_0[p[e_1,...,e_n]] \To p_0[p'[e_1,...,e_n]] \]
  \emph{preserves scope} relative to a set of scoping rules $\Sigma$
  if $\forall\,a,b \Subterm p_0,e_1,...,e_n$ (i.e.,
  each of $a$ and $b$ lies in one of $p_0,e_1,...,e_n$):
  \[ \SaysScope{p_0[p[e_1,...,e_n]]}{a}{b} \text{ iff }
     \SaysScope{p_0[p'[e_1,...,e_n]]}{a}{b} \]
\end{definition}

\begin{lemma}[Scope-equivalent patterns preserve scope]
  \label{lemma:rscope-scope-preservation}
  If $\SaysScopeEqv{p}{p'}$, then any rewrite
  $p_0[p[e_1,...,e_n]] \To p_0[p'[e_1,...,e_n]]$
  preserves scope.
\end{lemma}
  \begin{proof}
    We will prove the forward implication of the \textit{iff} in
    scope-preservation; the reverse is symmetric.
    View the ($\<$) preorder as a directed graph.
    Our given is that there is a path from $a$ to $b$ in
    $p_0[p[e_1,...,e_n]]$, where neither $a$ nor $b$ lies in $p$.
    Some subpaths of this path may traverse $p$; these subpaths are
    bounded by
    $\im{}e_1,\ex{}e_1,...,\im{}e_n,\ex{}e_n,\im{}p,\ex{}p$.
    The fact that $\SaysScopeEqv{p}{p'}$ means that these subpaths can be
    converted to subpaths in $p'$, bounded instead by
    $\im{}e_1,\ex{}e_1,...,\im{}e_n,\ex{}e_n,\im{}p',\ex{}p'$.
    Replace these subpaths. Now the whole path goes from $a$ to $b$ in
    $p_0[p'[e_1,...,e_n]]$.
  \end{proof}

We can use scope-equivalence to turn a rewrite rule $p \To p'$ into a
set of constraints that hold iff the rewrite rule preserves
scope. There will be one constraint for every pair $(a, b)$ from the
boundary. Each constraint will have the form:
\[f_1 \wedge f_2 \wedge ... f_n \texttt{ iff } f_1' \wedge f_2' \wedge ... f_m' \]
where each $f_i$ is a fact (e.g. $\SigBind{Let}{1}{2}$). This constraint is
found by stating that the premises of the derivation of
$\SaysScope{p}{a}{b}$ hold iff the premises of the derivation
$\SaysScope{p'}{a}{b}$ hold. These derivations are guaranteed to be unique,
and can found efficiently, because the algorithmic scope-checking
rules (\cref{fig:rscope-SD}) are syntax-directed.

As an example of this constraint generation, take the desugaring rule
for Let:
\begin{LongTable}
  $\NodeRm{Let}{\PVarA\;\PVarB\;\PVarC}$
  &$\To$&
  $\NodeRm{Apply}{\NodeRm{Lambda}{\PVarA\;\PVarC}\;\PVarB}$
\end{LongTable}
One of the necessary constraints says that:
\begin{Table}
&& $\SaysScope{\NodeRm{Let}{\PVarA\;\PVarB\;\PVarC}}{\PVarA}{\PVarB}$ \\
&iff& $\SaysScope{\NodeRm{Apply}{\NodeRm{Lambda}{\PVarA\;\PVarC}\;\PVarB}}{\PVarA}{\PVarB}$
\end{Table}

Each side of this ``iff'' has a unique derivation using the
algorithmic scope-checking rules (\cref{fig:rscope-SD}).
Replacing each side with the premises of its derivation
yields the constraint:
\begin{Table}
  $\SigBindRm{Let}{1}{2}$ iff $\SigBindRm{App}{1}{2} \wedge \SigImptRm{Lam}{1}$
\end{Table}

Since the boundary has size four ($\PVarA$, $\PVarB$, $\PVarC$, and $\Root$),
continuing this way leads to a total of $4^2=16$ constraints:
\begin{center}\begin{tabular}
  {r @{\;\;} c @{\;\;} r @{\,} c @{\,} r}

  $\SigBindRm{Let}{1}{1}$
  &iff& && $\SigBindRm{Lam}{1}{1}$ \\

  $\SigBindRm{Let}{1}{2}$
  &iff& $\SigBindRm{App}{1}{2}$ &$\wedge$& $\SigImptRm{Lam}{1}$ \\

  $\SigBindRm{Let}{1}{3}$
  &iff& && $\SigBindRm{Lam}{1}{2}$ \\

  $\SigImptRm{Let}{1}$
  &iff& $\SigImptRm{App}{1}$ &$\wedge$& $\SigImptRm{Lam}{1}$ \\

  $\SigBindRm{Let}{2}{1}$
  &iff& $\SigBindRm{App}{2}{1}$ &$\wedge$& $\SigExptRm{Lam}{1}$ \\

  $\SigBindRm{Let}{2}{2}$
  &iff& $\SigBindRm{App}{2}{2}$ && \\

  $\SigBindRm{Let}{2}{3}$
  &iff& $\SigBindRm{App}{2}{1}$ &$\wedge$& $\SigExptRm{Lam}{2}$ \\

  $\SigImptRm{Let}{2}$
  &iff& $\SigImptRm{App}{2}$ && \\

  $\SigBindRm{Let}{3}{1}$
  &iff& && $\SigBindRm{Lam}{2}{1}$ \\

  $\SigBindRm{Let}{3}{2}$
  &iff& $\SigBindRm{App}{1}{2}$ &$\wedge$& $\SigImptRm{Lam}{2}$ \\

  $\SigBindRm{Let}{3}{3}$
  &iff& && $\SigBindRm{Lam}{2}{2}$ \\

  $\SigImptRm{Let}{3}$
  &iff& $\SigImptRm{App}{1}$ &$\wedge$& $\SigImptRm{Lam}{2}$ \\

  $\SigExptRm{Let}{1}$
  &iff& $\SigExptRm{App}{1}$ &$\wedge$& $\SigExptRm{Lam}{1}$ \\

  $\SigExptRm{Let}{2}$
  &iff& $\SigExptRm{App}{2}$ && \\
  
  $\SigExptRm{Let}{3}$
  &iff& $\SigExptRm{App}{1}$ &$\wedge$& $\SigExptRm{Lam}{2}$ \\
  
  $\SigSelfRm{Let}$
  &iff& $\SigSelfRm{App}$ && \\
\end{tabular}\end{center}

We have just described how to generate constraints---covering the
$\mathit{gen}$ functions in \cref{fig:rscope-resugar}---and the previous lemma shows
that the constraints generated this way capture our aim in scope
inference. We now turn to solving these constraints.


\subsection{Constraint Solving}

These constraints can be solved by searching for their least
fixpoint, starting with the initial knowledge of the scoping rules for
the core language. Finding the \emph{least} fixpoint is sensible,
because by default, declarations should not be in scope.
Since all of the constraints have the form of an
``iff'' between conjunctions, the least fixpoint exists and can be
found by monotonically growing the set of known facts.

Solving for the least fixpoint gives a set of scoping rules for the
surface and core languages such that the desugaring rules preserve
this scope. Since the least fixpoint was seeded with the known scoping
rules for the core language, its output will contain at least those
facts.  However, they may have inferred \emph{additional}, incorrect
facts about the scope of the core language. For instance, consider the
following ``Lambda flip flop'' rule (where $\ConstRm{Flip}$ and
$\ConstRm{Flop}$ are constants, i.e., nodes of arity 0):
\begin{Table}
  $\NodeRm{LambdaFF}{\ConstRm{Flip}\,\PVar_1\,\PVar_2} \To
  \NodeRm{Lambda}{\PVar_1\,\PVar_2}$
  \\ \\
  $\NodeRm{LambdaFF}{\ConstRm{Flop}\,\PVar_1\,\PVar_2} \To
  \NodeRm{Lambda}{\PVar_2\,\PVar_1}$
\end{Table}
In traditional hygienic macro expansion systems this desugaring is
considered to be OK: the scope of a term is \emph{defined} by the
scope of its desugaring, which may vary on things such as the choice
between Flip and Flop constants. However, we will take the opposite
view: this desugaring should be rejected because the scope it produces
for LambdaFF cannot be captured by (reasonable) static scoping rules.

Let us work through scope inference for this example. From the first
rule, we can learn (from the Lambda on the \textsc{rhs}) that
$\SigBind{\text{LambdaFF}}{3}{2}$, and from the second rule, we can learn that
$\SigBind{\text{LambdaFF}}{2}{3}$. Applying either of these facts to the
\emph{other} rule gives that $\SigBind{\text{Lambda}}{1}{2}$: the \emph{body} of
the Lambda is in scope at its \emph{parameter}!  This contradicts the
known signature for Lambda (we know that $\SigNotBind{\text{Lambda}}{1}{2}$),
so these rules would be rejected.
In general, scope inference fails when the least fixpoint contains
facts about the scope of a core language construct that are not part
of that construct's signature.

\subsection{Ensuring Hygiene}

We have described how to infer scope by generating and then solving
constraints. There are two checks we should
perform, however, to ensure that desugaring cannot produce unbound
identifiers. These checks are performed by $\Op{checkScope}{}$
in \cref{fig:rscope-resugar}:
\begin{itemize}

\item
Any references introduced on the \textsc{rhs} of a sugar must
be bound. For instance, a sugar could not simply expand to $\RefnX$,
because that would be unbound.

\item
A sugar cannot delete a pattern variable that might contain a bound
declaration. For instance, it could not rewrite
$\NodeRm{lambda}{\PVarA\,\PVarB}$ to $\PVarB$, because $\PVarB$ might
contain a reference bound by a declaration in $\PVarA$. In general,
if a sugar deletes any pattern variable, then it must also delete all smaller
pattern variables (those that are less in the preorder).

\end{itemize}
These two checks ensure that sugars cannot cause unbound identifier
exceptions. Besides obviously being a problem, we would like to
prevent this because it violates our notion of \emph{hygiene}.
However, these problematic sugars would not be considered unhygienic
in the traditional sense.

Traditionally, research on hygiene has focused on preventing sugars
from accidentally capturing user-defined references and vice versa.
For instance, if a user binds $\DeclXi$ and then uses $\RefnX$
inside a sugar, and the sugar locally binds $\DeclXj$, then $\RefnX$
should not be bound by $\DeclXj$. These hygiene violations are called
``introduced-binder'' and ``introduced-reference'' violations,
respectively. There are also more subtle violations in which
desugaring makes observations about declaration
equality~\cite{adams-hygiene}.

However, there is a simpler goal we can aim for that gets at the heart
of the problem, and subsumes all of these specific properties.
The goal is that if two programs are $\alpha$-equivalent, then they
will still be $\alpha$-equivalent after a desugaring $\Desugar{}$:
\[ \SaysScopeEqa[\SigmaSurf]{e}{e'} \text{ \,implies\, }
   \SaysScopeEqa[\SigmaCore]{\Desugar{e}}{\Desugar{e'}}
\]
(Recall from \cref{def:rscope-eqa} that $\alpha$-equivalence is parameterized
by $\Sigma$. Therefore, in the above antecedent and consequent,
$\alpha$-equivalence is respectively defined by $\SigmaSurf$ and
$\SigmaCore$.)

This prevents accidental variable capture because $\alpha$-renaming
the captured variable would cause it to not be captured, changing the
$\alpha$-equivalence-class of the program. It also prevents the
introduction of unbound identifiers, because a program with an unbound
identifier is not $\alpha$-equivalent to any other program (it is
outside the domain of $\alpha$-equivalence).

Most hygiene papers don't mention this criterion for a simple reason:
$=_\alpha$ is not defined on their surface language, so they cannot even
state the requirement. Recent exceptions to this
rule~\cite{herman-hygiene,stansifer-romeo}
get around it by requiring sugar-writers to supply scoping rules
for the surface language. These scoping rules then define
$\alpha$-equivalence for the surface language.
In contrast, we \emph{infer} scoping rules for the surface language,
and can then ask whether these inferred rules preserve
$\alpha$-equivalence. In \cref{sec:rscope-hygiene} we will show that they do,
so long as inference was successful and $\Op{scopeCheck}{}$ passed.

This covers the $\Op{solve}{}$ algorithm in \cref{fig:rscope-resugar}, and
completes our description of scope inference: (i) find constraints for
every desugaring rule; (ii) find their least fixpoint, starting
with the known scoping rules for the core language; and (iii) check
that none of the sugars can produce unbound identifiers.


\subsection{Discussion of the Hygiene Property}

One may wonder how useful of a property preserving $\alpha$-equivalence is.
To put it strongly: what good is it for
desugaring to preserve $\alpha$-equivalence, when $\alpha$-equivalence
for the surface language was \emph{made up}?  In fact, there is a
situation in which preserving $\alpha$-equivalence is useless.
Suppose that $e_1 =_\alpha e_2$ in the surface language was \emph{defined} to mean
that $\Desugar{e_1} =_\alpha \Desugar{e_2}$ in the core language,
where $\Desugar{}$ is a naive, scope-unaware desugaring. Then
desugaring will preserve $\alpha$-equivalence, despite not being
hygienic!

It is therefore crucial that our binding language cannot express this
(rather insane) notion of surface-language $\alpha$-equivalence. It
is the \emph{weakness} (i.e., sanity) of our binding language that
leads to the \emph{strength} of our hygiene property. While ``sanity''
is a somewhat subjective notion, we can at least show that our binding
language passes one basic litmus test. Any set of scope rules in our
binding language will lead to a notion of $\alpha$-equivalence that is
invariant under permuting variable names (that is, it is
\emph{equivariant}~\cite{nominal-logic}):
\begin{lemma}
  If $\sigma$ is a permutation of variable names and
  $\SaysScopeEqa[\Sigma]{e}{e'}$, then
  $\SaysScopeEqa[\Sigma]{\sigma(e)}{\sigma(e')}$.
\end{lemma}
\begin{proof}[Proof sketch]
  A formula $\phi(a_1, \dd, a_n)$ is \emph{equivariant} if for all
  arguments $a_1, \dd, a_n$, and for all permuations $\sigma$ of variable names,
  $\phi(a_1, \dd, a_n) \text{ iff } \phi(\sigma(a_1), \dd,
  \sigma(a_n))$. We wish to show that $\alpha$-equivalence in our
  binding language is equivariant. The only parts of $\alpha$-equivalence
  (\cref{def:rscope-eqa}) that inspect variable names are the various
  rules that implicitly check that two variables have the same name
  (such as S-Decl and S-WB). Luckily, equality checking is
  equivariant. Furthermore, formulas that do not inspect or modify
  names are equivariant, and any composition of equivariant formulas
  is itself equivariant. Thus $\alpha$-equivalence is equivariant.
\end{proof}
The ``insane'' notion of $\alpha$-equivalence above---that defines
surface language $\alpha$-equivalence in terms of a naive
desugaring---does not respect variable permutation, and thus cannot be
expressed in our binding language.


\subsection{Correctness and Runtime}

The $\Op{inferScope}{}$ algorithm correctly solves the constraints:

\begin{theorem}[Rewrites preserve scope]\label{thm:rscope-resugar}
  \WhitePhantom{.}\\%Newline
  Let $\SigmaSurf =
  \Op{inferScope}{\SigmaCore,\,\{p_i \To p_i'\}_{i \in 1..n}}$.
  Then any rewrite of the form
  $p_0[p_i[e_1,...,e_n]] \To p_0[p_i'[e_1,...,e_n]]$
  will preserve scope.
  Furthermore, $\SigmaSurf$ is least (it is contained in every other
  set of scoping rules that would be preserved).
\end{theorem}
  \begin{proof}
    By construction, the constraints generated by $\Op{inferScope}{}$
    ensure that $\SaysScopeEqv[\SigmaSurf]{p_i}{p_i'}$ for each $i$.
    By \cref{lemma:rscope-scope-preservation},
    this implies that any rewrite
    $p_0[p_i[e_1,...,e_n]] \To p_0[p_i'[e_1,...,e_n]]$ will preserve scope.
    Thus it suffices to show that $\Op{solve}{}$ does, in fact, find the
    least solution to the constraints given $\SigmaCore$.
  
    $\Op{solve}{}$ works with scope signatures $\SigmaCore$ and
    $\SigmaSurf$, and a set of constraints $\mathbb{C}$:
    \begin{Table}
      $F_{11}' \wedge  F_{12}' \wedge ...$ &iff&
      $F_{11}'' \wedge F_{12}'' \wedge ...$ \\
      $F_{21}' \wedge  F_{22}' \wedge ...$ &iff&
      $F_{21}'' \wedge F_{22}'' \wedge ...$ \\
      &...&
    \end{Table}
    At any point during the evaluation of $\Op{solve}{}$, let the
    \emph{meaning} $\phi$ of $\SigmaCore$, $\SigmaSurf$, and $Cs$ be:
    \[ \bigwedge_{F \in \SigmaSurf}\hspace{-0.5em}F \wedge
       \bigwedge_{F \in \SigmaCore}\hspace{-0.5em}F \wedge
       \bigwedge_{F \in \overline \SigmaCore}\hspace{-0.5em}\neg F \wedge
       \bigwedge_i \left(\bigwedge_j F_{ij}' \text{ iff } \bigwedge_k F_{ik}''\right)
    \]
    That is,
    \begin{enumerate}[noitemsep]
      \item Every fact in $\SigmaSurf$ is true
      \item Every fact in $\SigmaCore$ is true
      \item Every fact in the core language but \emph{not} in
        $\SigmaCore$ is false (i.e., we assume that $\SigmaCore$ lists
        \emph{all} scoping rules for the core language)
      \item The constraints hold.
    \end{enumerate}
  
    Upon initialization during $\Op{solve}{}$, $\phi$ follows from
    $\SigmaCore$ and $\mathbb{C}$. Furthermore, every step of $\Op{solve}{}$
    maintains $\phi$. There are three kinds of steps to consider, and
    each follows from a logical equivalence:
    \begin{itemize}[noitemsep]
    \item ``If a fact $F$ in a constraint is in $\SigmaSurf$:
      Delete $F$ from the constraint.''
      \[ F \wedge (F \wedge F_2 \wedge ...
         \text{ iff } F_1' \wedge F_2' \wedge ...)
         \equiv
         F \wedge (F_2 \wedge ...
         \text{ iff } F_1' \wedge F_2' \wedge ...)
      \]
    \item ``If one side of a constraint is empty:
      Delete the constraint;
      Add the other side to $\SigmaSurf$ (maintaining trans. closure).''
      \[ (\mathit{true} \text{ iff } F_1 \wedge ... \wedge F_n)
         \equiv
         F_1 \wedge ... \wedge F_n
      \]
    \item ``If any fact $F$ in $\SigmaSurf$ is in the complement of
      $\SigmaCore$: ERROR''
      \[ F \wedge \neg F \equiv \mathit{false} \]
    \end{itemize}
  
    Finally, when $\Op{solve}{}$ halts, it is because the facts in $\mathbb{C}$
    and $\SigmaSurf$ are disjoint, and every constraint in $\mathbb{C}$ has at
    least one fact on each side. Therefore it is valid to obtain a
    minimal set of facts by setting every
    fact in $\overline\SigmaSurf$ to $\mathit{false}$; doing so will
    satisfy $\mathbb{C}$ by making both sides of every remaining constraint
    false. Thus we have shown that the final $\SigmaSurf$:
    \[ \bigwedge_{F \in \SigmaSurf}\hspace{-0.5em}F \wedge
       \bigwedge_{F \in \overline \SigmaSurf}\hspace{-0.5em}\neg F
    \]
    is a minimal solution to the initial $\SigmaCore$ and $\mathbb{C}$:
    \[ \bigwedge_{F \in \SigmaCore}\hspace{-0.5em}F \wedge
       \bigwedge_{F \in \overline \SigmaCore}\hspace{-0.5em}\neg F \wedge
       \bigwedge_i \left(\bigwedge_j F_{ij}' \text{ iff } \bigwedge_k F_{ik}''\right)
    \]
    Thus the surface language scoping rules $\SigmaSurf$ found by
    $\Op{solve}{}$ are valid and minimal, given the core language scoping rules
    $\SigmaCore$ plus the constraints $\mathbb{C}$.
  \end{proof}
  
  \begin{corollary}[Desugaring preserves scope]
    \WhitePhantom{.}\\%Newline
    Let $\SigmaSurf =
    \Op{inferScope}{\SigmaCore,\,\{p_i \To p_i'\}_{i \in 1..n}}$.
    Then desugaring with the rules $\{p_i \To p_i'\}_{i \in 1..n}$
    will preserve scope.
  \end{corollary}
  \begin{proof}
    Induct on the number of rewrites performed.
  \end{proof}

  Furthermore, scope inference runs in time
  $O(\Sigma_{C \in \mathit{surf}} \Arity{C}^3)$:

  \begin{lemma}
    $\Op{inferScope}{\Sigma,\,\mathbb{C}}$ runs in time
    $O(\Op{size}{\mathbb{C}} + \Sigma_{C \in \mathit{surf}} \Arity{C}^3)$.
  \end{lemma}

  \begin{proof}
    The running time of $\Op{inferScope}{}$ is dominated by
    $\Op{solve}{}$, which in turn is dominated by two operations:
    iterating over the facts in $\mathbb{C}$, and adding facts to $\SigmaSurf$.
    Iterating over the facts in $\mathbb{C}$ takes time $\Op{size}{\mathbb{C}}$, where
    $\Op{size}{\mathbb{C}}$ is the total number of facts in $\mathbb{C}$.  Each fact
    added to $\SigmaSurf$ requires maintaining the transitive closure of
    $\SigmaSurf$, for the constructor $C$ of the fact. This can be done
    with an amortized cost of $O(\Arity{C})$ per $C$-fact added.  (To
    add a fact $\FactP{a}{b}$ that does not appear in $\SigmaSurf$,
    insert it and then recursively add $\FactP{a}{c}$ for every fact
    $\FactP{b}{c}$, and add $\FactP{c}{b}$ for every fact $\FactP{c}{a}$.)
    Since there are $O(\Arity{C}^2)$ possible $C$-facts to add, this
    adds an additional $O(\Sigma_{C \in \mathit{surf}} \Arity{C}^3)$
    running time.
  \end{proof}

The cubic parameter is concerning, but not a problem in practice for a
number of reasons. First, $\Arity{C}$ tends to be small.
Second, this algorithm is run off-line, and once per language.
Finally, as we discuss in \cref{sec:rscope-impl}, in practice the running
time is extremely small.


\section{Implementation and Evaluation}
\label{sec:rscope-impl}
 
We have implemented the scope inference algorithm. Beyond what is
shown in this chapter, the implementation also allows (i) marking
variables as \emph{global references} that should refer to globally
available identifiers in the expanded program, such as \Code{print},
and (ii) a select form of copying a pattern variable, where the pattern variable contains a
declaration and the copy is meant to be a reference of the same name.
The implementation is available at \url{https://github.com/brownplt/scope-graph}.

Besides the examples shown earlier, we have tested this
implementation on sugars from three languages:
\begin{itemize}
  \item All of the sugars that bind values in the Pyret language
    (pyret.org): namely \texttt{for} expressions, \texttt{let}
    statement clustering (nested bindings are grouped into a single
    \texttt{let}), and \texttt{function} declarations.
  \item Haskell list comprehensions, which include guards, generators,
    and local bindings.
  \item All of the sugars that bind values in R5RS
    Scheme~\cite{scheme5}: namely \texttt{let}, \texttt{let*},
    \texttt{letrec}, and \texttt{do}.
\end{itemize}

Some of the desugarings use ellipses in their definition, and thus had
to be translated to match our fixed-arity assumption. (To do so, we
introduced auxiliary \textsc{ast} constructors and used those to express the
equivalent looping.) \texttt{letrec}
required one further adjustment to successfully infer scope.\footnote{
  The change was to have the desugaring distinguish between the letrec
  having zero bindings or one-or-more bindings. This prevented a
  fact of the form \SpecBind{i}{i} from being applied to the binding
  list of the desugared \texttt{let}, which would make its bindings
  recursive. We have not found a principled account for why this
  was necessary.
} After
that, our tool successfully inferred scope for all of the sugars
except for Scheme's \texttt{do}. In the rest of this section, we will
describe many of these sugars in more detail, ending with \texttt{do}.

In practice, the running times are very modest. In our implementation in
Rust (\url{rust-lang.org}) all of the sugars we have tested run in
about 18ms on a generic desktop, of which ~8ms is parsing
time.\marginpar{The original publication gave a total running time of
  130ms because the code was compiled without \Code{cargo -{}-release}.}
Therefore, the speed is even fast enough for scope inference to
be used as part of a language developer's rapid prototyping workflow.

\subsection{Case Study: Pyret \texttt{for} Expressions}
\label{sec:rscope-for-example}
Consider the ``for expressions'' of the Pyret language:
\begin{verbatim}
for fold(p from 1, n from range(1, 6)):
  p * n
end # Produces 5! = 120
\end{verbatim}
  This example desugars into:
\begin{verbatim}
fold(lam(p, n): p * n end, 1, range(1, 6))
\end{verbatim}
  In general, the \Code{for} syntax takes a function
  expression, any number of \Code{from} clauses, and a body. It desugars
  into a call to the function, passing it as arguments (i) a lambda whose
  parameters are the \textsc{lhs}s of each \Code{from} and whose body
  is the body of the \Code{for}, and (ii) the
  \textsc{rhs} of each \Code{from}.

  Our system produces the following scoping rules for \Code{for},
  shown both textually and pictorially:\marginpar{%
In the textual representation of the scoping rules, we give names
to a node's children. Formally, these should be indices.
}

\begin{scopeDescription}
\begin{center}
\begin{tikzScopeDiagram}
  \tikzRoot
    {A}{\tikzParentThree{For}
      {B}{\tikzChild{func}}
      {C}{\tikzChild{froms}}
      {D}{\tikzChild{body}}}
      
  \begin{tikzEdges}
    \tikzEdgeR{C+}{D-};
    \tikzEdge{A-}{D-};
    \tikzEdge{A-}{C-};
    \tikzEdge{A-}{B-};
  \end{tikzEdges}
\end{tikzScopeDiagram}
\end{center}
\begin{ScopeRules}
  \RuleImpt{For}{func} \\
  \RuleImpt{For}{froms} \\
  \RuleImpt{For}{body} \\
  \RuleBind{For}{body}{froms}
\end{ScopeRules}
\end{scopeDescription}

\vspace{1em}
\begin{scopeDescription}
\begin{center}
\begin{tikzScopeDiagram}
  \tikzRoot
    {A}{\tikzParentThree{From}
      {B}{\tikzChild{param}}
      {C}{\tikzChild{arg}}
      {D}{\tikzChild{froms}}}

  \begin{tikzEdges}
    \tikzEdge{A-}{B-};
    \tikzEdge{A-}{C-};
    \tikzEdge{A-}{D-};
    \tikzEdge{B+}{A+};
    \tikzEdge{D+}{A+};
  \end{tikzEdges}
\end{tikzScopeDiagram}
\end{center}
\begin{ScopeRules}
  \RuleImpt{From}{param} \\
  \RuleImpt{From}{arg} \\
  \RuleImpt{From}{froms} \\
  \RuleExpt{From}{param} \\
  \RuleExpt{From}{froms}
\end{ScopeRules}
\end{scopeDescription}


\subsection{Case Study: Haskell List Comprehensions}
\label{sec:list-rscope-example}
Haskell list comprehensions consist of sugar for \emph{boolean
  guards} that filter the list, \emph{generators} that specify the
domain of the elements in the list, and \emph{local bindings}.
To quote the language standard~\cite[section 3.11]{haskell-language}:
``List comprehensions satisfy these identities, which may be used as a
translation into the kernel:''
\begin{Table}
  \texttt{[ e | True ]}
  &=& \texttt{[e]}
  & (Base case)
  \\
  \texttt{[ e | q ]}
  &=& \texttt{[ e | q, True ]}
  & (Base case)
  \\
  \texttt{[ e | b, Q ]}
  &=& \texttt{if b then [ e | Q ] else []}
  & Boolean guards
  \\
  \texttt{[ e | p <- l, Q ]}
  &=& \texttt{let ok p = [ e | Q ]} \\
  &&  \texttt{\phantom{....}ok \_ = []} \\
  &&  \texttt{in concatMap ok l}
  & Generators
  \\
  \texttt{[ e | let decls, Q ]}
  &=& \texttt{let decls in [ e | Q ]}
  & Local bindings
\end{Table}
``where $e$ ranges over expressions,
$p$ over patterns, $l$ over list-valued expressions, $b$ over boolean
expressions, $decls$ over declaration lists, $q$ over qualifiers, and
$Q$ over sequences of qualifiers.''

For example, the perfect numbers (those equal to the sum of their
divisors) can be calculated by:
\[
\texttt{[n | n <- [1..], let d = divisors n, sum d == n]}
\]

Our system successfully infers the scope of these sugars. We will
describe them one at a time. First, list comprehensions
\texttt{[e | Q]} consist of an expression $e$ and a list of
\emph{qualifiers} $Q$. Any declarations exported by $Q$ (such as $n$
above) should be in scope at $e$:

\begin{scopeDescription}
\begin{center}
\begin{tikzScopeDiagram}
  \tikzRoot
    {A}{\tikzParentTwo{\;[e $\vert$ Q]\;}
      {B}{\tikzChild{\;e\tall\;}}
      {C}{\tikzChild{\;Q\;}}}

  \begin{tikzEdges}
    \tikzEdge{A-}{B-};
    \tikzEdge{A-}{C-};
    \tikzEdgeLLL{C+}{B-};
  \end{tikzEdges}
\end{tikzScopeDiagram}
\end{center}
\begin{ScopeRules}
  \RuleImpt{ListComprehension}{e} \\
  \RuleImpt{ListComprehension}{Q} \\
  \RuleBind{ListComprehension}{e}{Q}
\end{ScopeRules}
\end{scopeDescription}

Boolean guards $b, Q$ have a boolean expression $b$ that is used to
filter the list, and a sequence of more qualifiers $Q$. The scope of a
boolean guard expression is simple: besides lexical scope, any
declarations from $Q$ are exported:

\begin{scopeDescription}
\begin{center}
\begin{tikzScopeDiagram}
  \tikzRoot
    {A}{\tikzParentTwo{\;b, Q\;}
      {B}{\tikzChild{\;b\tall\;}}
      {C}{\tikzChild{\;Q\;}}}

  \begin{tikzEdges}
    \tikzEdge{A-}{B-};
    \tikzEdge{A-}{C-};
    \tikzEdge{C+}{A+};
  \end{tikzEdges}
\end{tikzScopeDiagram}
\end{center}
\begin{ScopeRules}
  \RuleImpt{LC\_Guard}{b} \\
  \RuleImpt{LC\_Guard}{Q} \\
  \RuleExpt{LC\_Guard}{Q}
\end{ScopeRules}
\end{scopeDescription}

A generator expression \texttt{p $\leftarrow$ l, Q} binds elements of
list $l$ to pattern $p$. $p$ is bound in $Q$, and the declarations of
both $p$ and $Q$ are exported:

\begin{scopeDescription}
\begin{center}
\begin{tikzScopeDiagram}
  \tikzRoot
    {A}{\tikzParentThree{\;p $\leftarrow$ l, Q\;}
      {B}{\tikzChild{\;p\tall\;}}
      {C}{\tikzChild{\;l\tall\;}}
      {D}{\tikzChild{\;Q\;}}}

  \begin{tikzEdges}
    \tikzEdge{A-}{B-};
    \tikzEdge{A-}{C-};
    \tikzEdge{A-}{D-};
    \tikzEdgeRRR{B+}{D-};
    \tikzEdge{B+}{A+};
    \tikzEdge{D+}{A+};
  \end{tikzEdges}
\end{tikzScopeDiagram}
\end{center}
\begin{ScopeRules}
  \RuleImpt{LC\_Generator}{p} \\
  \RuleImpt{LC\_Generator}{l} \\
  \RuleImpt{LC\_Generator}{Q} \\
  \RuleBind{LC\_Generator}{Q}{p} \\
  \RuleExpt{LC\_Generator}{p} \\
  \RuleExpt{LC\_Generator}{Q}
\end{ScopeRules}
\end{scopeDescription}

Finally, local bindings $decls$ are bound in the rest of the
qualifiers $Q$, and also exported:

\begin{scopeDescription}
\begin{center}
\begin{tikzScopeDiagram}
  \tikzRoot
    {A}{\tikzParentTwo{\;let decls, Q\;}
      {B}{\tikzChild{\;decls\tall\;}}
      {C}{\tikzChild{\;Q\;}}}

  \begin{tikzEdges}
    \tikzEdge{A-}{B-};
    \tikzEdge{A-}{C-};
    \tikzEdgeRR{B+}{C-};
    \tikzEdge{B+}{A+};
    \tikzEdge{C+}{A+};
  \end{tikzEdges}
\end{tikzScopeDiagram}
\end{center}
\begin{ScopeRules}
  \RuleImpt{LC\_Let}{decls} \\
  \RuleImpt{LC\_Let}{Q} \\
  \RuleBind{LC\_Let}{Q}{decls} \\
  \RuleExpt{LC\_Let}{decls} \\
  \RuleExpt{LC\_Let}{Q}
\end{ScopeRules}
\end{scopeDescription}


\subsection{Case Study: Scheme's Named-Let}
\label{sec:rscope-named-let-example}

The Scheme language standard defines two variants of
the \texttt{let} sugar. The regular variant of \texttt{let} has the syntax
\texttt{(let ((x val) ...) body)}, and binds each declaration \texttt{x}
to the corresponding \texttt{val} in \texttt{body}. The scope of this
variant can be inferred similarly to how we inferred the scope of
\texttt{let*} in \cref{sec:rscope-example}.

The other variant is called ``named'' let. Its syntax is
\texttt{(let f ((x val) ...) body)}, and it behaves like the regular
\texttt{let} except that it additionally binds \texttt{f} to
\texttt{(lambda (x ...) body)}. It can thus be used for recursive
computations, such as reversing a list:
\begin{lstlisting}
(define (reverse lst)
  (let rev ([unreversed lst]
            [reversed empty])
    (if (empty? unreversed)
        reversed
        (rev (cdr unreversed)
             (cons (car unreversed) reversed)))))
\end{lstlisting}

Named-let desugars by the rule:\marginpar{%
We describe Racket's desugaring because it is slightly more clear
  (using better variable names, and putting the application inside of
  the \texttt{letrec}). These differences have no effect on scope
  inference.
}
\begin{lstlisting}
(define-syntax-rule
  ;;; The named let sugar:
  (let proc-id ([arg-id init-expr] ...) body)
  ;;; Desugars into:
  (letrec ([proc-id (lambda (arg-id ...) body)])
    (proc-id init-expr ...)))
\end{lstlisting}

We will represent the \textsc{ast} for named-let expressions with the grammar:
\begin{Table}
  $e$ &$::=$&
    $\NodeRm{Let}{\DeclX\;b\;e}$
      & ``Named-let: bind initial values $b$ and recursive function $\DeclX$ in $e$'' \\
    &$|$& $\ldots$ \\
  $b$ &$::=$&
    $\NodeRm{Bind}{\DeclX\;e\;b}$
      & ``Bind $\DeclX$ to $e$, and bind $b$'' \\
    &$|$&   $\ConstRm{EndBinds}$
      & ``No more bindings''
\end{Table}

Translating the desugaring to use this grammar, our system correctly
infers the binding structure:

\begin{scopeDescription}
\begin{center}
\begin{tikzScopeDiagram}
  \tikzRoot
    {A}{\tikzParentThreeWide{Let}
      {B}{\tikzChild{proc-id}}
      {C}{\tikzChild{bindings}}
      {D}{\tikzChild{body}}}

  \begin{tikzEdges}
    \tikzEdge{B+}{C-};
    \tikzEdgeRR{B+}{D-};
    \tikzEdge{C+}{D-};
    \tikzEdge{A-}{D-};
    \tikzEdge{A-}{C-};
    \tikzEdge{A-}{B-};
  \end{tikzEdges}
\end{tikzScopeDiagram}
\end{center}
\hspace{2em} % push
\begin{ScopeRules}
  \RuleImpt{Let}{proc-id} \\
  \RuleImpt{Let}{bindings} \\
  \RuleImpt{Let}{body} \\
  \RuleBind{Let}{bindings}{proc-id} \\
  \RuleBind{Let}{body}{proc-id} \\
  \RuleBind{Let}{body}{bindings}
\end{ScopeRules}
\end{scopeDescription}

\vspace{1em}
\begin{scopeDescription}
\begin{center}
\begin{tikzScopeDiagram}
  \tikzRoot
    {A}{\tikzParentThreeWide{Bind}
      {B}{\tikzChild{arg-id}}
      {C}{\tikzChild{init-expr}}
      {D}{\tikzChild{bindings}}}

  \begin{tikzEdges}
    \tikzEdge{B+}{A+};
    \tikzEdge{D+}{A+};
    \tikzEdge{A-}{D-};
    \tikzEdge{A-}{C-};
    \tikzEdge{A-}{B-};
  \end{tikzEdges}
\end{tikzScopeDiagram}
\end{center}
\hspace{2em} % shove
\begin{ScopeRules}
  \RuleImpt{Bind}{arg-id} \\
  \RuleImpt{Bind}{init-expr} \\
  \RuleImpt{Bind}{bindings} \\
  \RuleExpt{Bind}{arg-id} \\
  \RuleExpt{Bind}{bindings}
\end{ScopeRules}
\end{scopeDescription}

\noindent
While this correctly reflects the scoping of named-let, observe that
it permits the let-bindings to shadow the function name. This follows
because \texttt{(arg-id ...)} can shadow
\texttt{proc-id} in the macro definition. Of course, if a program
actually did this, it would render the named part of the named-let
useless! Nevertheless, we faithfully reflect the language, and indeed
our inferred scope may be a useful diagnostic to the language designer.

\subsection{Case Study: Scheme's \texttt{do}}

Scheme's \texttt{do} expression can be used to perform what
\texttt{do-while} and \texttt{for} loops might do in another
language. For instance, this \texttt{do} expression reads three
numbers off of \texttt{stdin}, before displaying their sum.
\begin{lstlisting}
(do ((sum 0)
     (i 0 (+ i 1)))
  ((= i 3) (display "The sum is: ") (display sum) (newline))
    (set! sum (+ sum (string->number (read-line)))))
\end{lstlisting}

In general, \texttt{do} binds a list of variables [\texttt{sum} and
\texttt{i}] to initial values [\texttt{0} and \texttt{0}], and then
repeatedly evaluates the body of the loop [\texttt{(set! sum ...)}]  and
updates the variables according to optional step expressions
[\texttt{(+ i 1)}] until a condition [\texttt{(= i 3)}] is met, at which
point it evaluates a final sequence of expressions [\texttt{(display
  "The sum is: ") (display sum) (newline)}].

The desugaring of \texttt{do} is given by~\cite[derived forms]{scheme5}:
\begin{lstlisting}
  (define-syntax do (syntax-rules ()
    ((do ((var init step ...) ...)
         (test expr ...)
         command ...)
    (letrec ((loop (lambda (var ...)
                     (if test
                       (begin #f expr ...)
                       (begin command ...
                        (loop (do "step" var step ...) ...))))))
      (loop init ...)))
    ((do "step" x) x)
    ((do "step" x y) y)))
\end{lstlisting}

We will focus on the scope of the binding list, as scope inference
fails on it. Its correct scope is:\\%Newline (strangely necessary)
\begin{scopeDescription}
\begin{center}
\begin{tikzScopeDiagram}
  \tikzRoot
    {A}{\tikzParentFour{DoBind}
      {B}{\tikzChild{var}}
      {C}{\tikzChild{init}}
      {D}{\tikzChild{step}}
      {E}{\tikzChild{binds}}}

  \begin{tikzEdges}
    \tikzEdge{A-}{B-};
    \tikzEdge{A-}{C-};
    \tikzEdge{A-}{D-};
    \tikzEdge{A-}{E-};
    \tikzEdge{B+}{A+};
    \tikzEdge{E+}{A+};
    \tikzEdgeRRR{B+}{D-};
    \tikzEdgeLLL{E+}{D-};
  \end{tikzEdges}
\end{tikzScopeDiagram}
\end{center}
\hspace{3.7em}%Not sure why these overlap by default.
\begin{ScopeRules}
  \RuleImpt{DoBind}{var} \\
  \RuleImpt{DoBind}{init} \\
  \RuleImpt{DoBind}{step} \\
  \RuleImpt{DoBind}{binds} \\
  \RuleExpt{DoBind}{var} \\
  \RuleExpt{DoBind}{binds} \\
  \RuleBind{DoBind}{step}{var} \\
  \RuleBind{DoBind}{step}{binds}
\end{ScopeRules}
\end{scopeDescription}

While our binding language can express this scope, our algorithm is
unable to handle inferring scope for it: it incorrectly infers that
\texttt{var} is in scope at \texttt{init}. In more detail, whatever a
desugaring does, it must at some point take apart the binding
list. However, once one of the declarations \texttt{var} has been
removed from the list, it must have a path to the rest of the
list. Unfortunately that path will put both \texttt{init} \emph{and}
\texttt{step} in scope of it. Therefore we cannot infer scope for this
macro. In general, we cannot handle binding lists in which the
bindings are visible in some expressions within the list
(\texttt{step}) but not others (\texttt{init}).

This can naturally be fixed by putting \texttt{do} in the core
language, but can also be addressed by altering the \emph{syntax}
slightly: separating the \texttt{init} list from the \texttt{step}
list (which are semantically different entities) in the \textsc{ast}
would avoid this unwanted conflation.
More broadly, however, we believe that
extending scope inference to work on desugaring rules with ellipses
can solve this problem directly, as it is only the intermediate
steps where the binding list is deconstructed that pose a problem.
This raises questions that we leave for future work.\footnote{
  What do scope specifications over arbitrary-length lists look like?
  Can the $i$th element be bound in the $(i\!+\!1)$st element (e.g., for
  \texttt{let*})? How about $(i\!+\!1)$ in $i$, or $i$ in $j$ for all $i$ and $j$? How
  does scope inference handle these cases, while still being
  correct, fast, and hygienic?}

  \subsection{Catalog of Scoping Rules (Extended)}
  \label{sec:rscope-catalog}
  In \cref{fig:rscope-catalog}, we demonstrate the expressiveness of our
  scoping rule language by showcasing a small catalog of scoping rules.
  They are shown diagrammatically. Some arrows have been omitted when
  they are implied by transitivity (e.g., the arrow between
  \ConstRm{Lambda} and its \ConstRm{(body)} is omitted). The names written for the
  children of each term (such as ``(arg)'' and ``$\DeclX$'') are
  only expository, but meant to suggest a grammar the language might follow.
  
  \emph{Single-binding Lambda and Let}
  shows scoping rules for Lambda and Let when they have only a single
  binding position. \emph{Multi-argument Lambda} extends this to functions
  of more than one argument. Notice that the scoping rule for Lambda
  is the same in each case: the only difference is that one is meant to
  be used with a single declaration while the other is meant to use a
  Param.
  
  The next three rules describe the binding for Scheme-style Let, Let*,
  and Letrec let-bindings. Akin to Lambda, the rules for various Lets
  are identical; only their bindings differ.
  
  \emph{Pattern Matching} shows a rule for (nestable) pattern
  matching (or deconstructive assignment to) a pair. Its ``left'' and
  ``right'' children would typically be variable declarations, although
  they could themselves be nested pattern matches. Its ``MatchPair''
  construct can be used in any of the binding sites of the other scoping
  constructs.
  
  Finally, we show the (correctly) inferred scope for the three sugars
  in Pyret that bind values:
  \begin{itemize}
  
  \item For ``loops'' are actually syntactic sugar for the application of a
    higher-order function (like \Code{map}).
  
  \item Pyret has both statement and expression forms of \Code{let}. The
    statements desugar into \Code{let} expressions, which then merge
    with adjacent \Code{let}s to form a single binding block.
  
  \item Pyret function declarations desugar into lambdas with explicit
    recursive bindings.
  
  \end{itemize}

  \begin{figure*}
  \begin{Wide}

    \begin{center}
    \begin{tabular}{p{8em} c @{\hspace{-1em}} c @{\hspace{-1em}} c}
  
    \textbf{Single-binding Lambda and Let}&
    \begin{tikzScopeDiagram}
      \tikzRoot
          {A}{\tikzParentTwo{Lambda1}
            {B}{\tikzChild{$\DeclX$}}
            {C}{\tikzChild{(body)}}}
      \begin{tikzEdges}
        \tikzEdge{A-}{B-};
        \tikzEdgeLL{B+}{C-};
      \end{tikzEdges}
    \end{tikzScopeDiagram}&
    \begin{tikzScopeDiagram}
      \tikzRoot
          {A}{\tikzParentThree{Let1}
            {B}{\tikzChild{$\DeclX$}}
            {C}{\tikzChild{(val)}}
            {D}{\tikzChild{(body)}}}
      \begin{tikzEdges}
        \tikzEdge{A-}{B-};
        \tikzEdge{A-}{C-};
        \tikzEdgeLL{B+}{D-};
      \end{tikzEdges}
    \end{tikzScopeDiagram}
  
    \\
  
    \textbf{Multi-argument Lambda}&
    \begin{tikzScopeDiagram}
      \tikzRoot
          {A}{\tikzParentTwo{Lambda}
            {B}{\tikzChild{(params)}}
            {C}{\tikzChild{(body)}}}
      \begin{tikzEdges}
        \tikzEdge{A-}{B-};
        \tikzEdgeLL{B+}{C-};
      \end{tikzEdges}
    \end{tikzScopeDiagram}&
    \begin{tikzScopeDiagram}
      \tikzRoot
          {A}{\tikzParentTwo{Param}
            {B}{\tikzChild{$\DeclX$}}
            {C}{\tikzChild{(params)}}}
      \begin{tikzEdges}
        \tikzEdge{A-}{B-};
        \tikzEdge{A-}{C-};
        \tikzEdge{B+}{A+};
        \tikzEdge{C+}{A+};
      \end{tikzEdges}
    \end{tikzScopeDiagram}
  
    \\
  
    \textbf{Multi-arm Let}&
      \begin{tikzScopeDiagram}
        \tikzRoot
            {A}{\tikzParentTwo{Let}
              {B}{\tikzChild{(binds)}}
              {C}{\tikzChild{(body)}}}
        \begin{tikzEdges}
          \tikzEdge{A-}{B-};
          \tikzEdgeLL{B+}{C-};
        \end{tikzEdges}
      \end{tikzScopeDiagram}&
      \begin{tikzScopeDiagram}
        \tikzRoot
            {A}{\tikzParentThree{Bind}
              {B}{\tikzChild{$\DeclX$}}
              {C}{\tikzChild{(val)}}
              {D}{\tikzChild{(binds)}}}
        \begin{tikzEdges}
          \tikzEdge{A-}{B-};
          \tikzEdge{A-}{C-};
          \tikzEdge{A-}{D-};
          \tikzEdge{B+}{A+};
          \tikzEdge{D+}{A+};
        \end{tikzEdges}
      \end{tikzScopeDiagram}
  
      \\
  
      \textbf{Multi-arm Let* (Pyret)}&
      \begin{tikzScopeDiagram}
        \tikzRoot
            {A}{\tikzParentTwo{Let*}
              {B}{\tikzChild{(binds)}}
              {C}{\tikzChild{(body)}}}
        \begin{tikzEdges}
          \tikzEdge{A-}{B-};
          \tikzEdgeLL{B+}{C-};
        \end{tikzEdges}
      \end{tikzScopeDiagram}&
      \begin{tikzScopeDiagram}
        \tikzRoot
            {A}{\tikzParentThree{Bind*}
              {B}{\tikzChild{$\DeclX$}}
              {C}{\tikzChild{(val)}}
              {D}{\tikzChild{(binds)}}}
        \begin{tikzEdges}
          \tikzEdge{A-}{B-};
          \tikzEdge{A-}{C-};
          \tikzEdgeLL{B+}{D-};
          \tikzEdge{D+}{A+};
        \end{tikzEdges}
      \end{tikzScopeDiagram}
      
      \\
  
      \textbf{Multi-arm Letrec}&
      \begin{tikzScopeDiagram}
        \tikzRoot
            {A}{\tikzParentTwo{Letrec}
              {B}{\tikzChild{(binds)}}
              {C}{\tikzChild{(body)}}}
        \begin{tikzEdges}
          \tikzEdge{A-}{B-};
          \tikzEdgeLL{B+}{C-};
        \end{tikzEdges}
      \end{tikzScopeDiagram}&
      \begin{tikzScopeDiagram}
        \tikzRoot
            {A}{\tikzParentThreeWide{Bindrec}
              {B}{\tikzChild{$\DeclX$}}
              {C}{\tikzChild{(val)}}
              {D}{\tikzChild{(binds)}}}
        \begin{tikzEdges}
          \tikzEdge{A-}{B-};
          \tikzEdge{B+}{A+};
          \tikzEdgeRR{B+}{D-};
          \tikzEdgeLL{D+}{B-};
          \tikzEdge{B+}{C-};
        \end{tikzEdges}
      \end{tikzScopeDiagram}
  
    \\
      
    \textbf{Pattern Matching}&
    \begin{tikzScopeDiagram}
      \tikzRoot
          {A}{\tikzParentTwo{MatchPair}
            {B}{\tikzChild{(left)}}
            {C}{\tikzChild{(right)}}}
      \begin{tikzEdges}
        \tikzEdge{A-}{B-};
        \tikzEdge{A-}{C-};
        \tikzEdge{B+}{A+};
        \tikzEdge{C+}{A+};
      \end{tikzEdges}
    \end{tikzScopeDiagram}
  
    \\
  
    \textbf{For Loops (Pyret)}&
    \begin{tikzScopeDiagram}
      \tikzRoot
        {A}{\tikzParentThreeWide{For}
          {B}{\tikzChild{(iter)}}
          {C}{\tikzChild{(binds)}}
          {D}{\tikzChild{(body)}}}
        \begin{tikzEdges}
          \tikzEdge{A-}{B-};
          \tikzEdge{A-}{C-};
          \tikzEdge{A-}{D-};
          \tikzEdgeRRR{C+}{D-};
        \end{tikzEdges}
    \end{tikzScopeDiagram}&
    \begin{tikzScopeDiagram}
      \tikzRoot
        {A}{\tikzParentThree{Bindfor}
          {B}{\tikzChild{$\DeclX$}}
          {C}{\tikzChild{(val)}}
          {D}{\tikzChild{(binds)}}}
        \begin{tikzEdges}
          \tikzEdge{A-}{B-};
          \tikzEdge{A-}{C-};
          \tikzEdge{A-}{D-};
          \tikzEdge{B+}{A+};
          \tikzEdge{D+}{A+};
        \end{tikzEdges}
    \end{tikzScopeDiagram}
  
    \\
  
    \textbf{Functions (Pyret)}&
     \multicolumn{2}{c}{
    \begin{tikzScopeDiagram}
      \tikzRoot
        {A}{\tikzParentFour{Function}
          {B}{\tikzChild{F}}
          {C}{\tikzChild{(params)}}
          {D}{\tikzChild{(body)}}
          {E}{\tikzChild{(stmts)}}}
         \begin{tikzEdges}
           \tikzEdge{A-}{B-};
           \tikzEdge{A-}{C-};
           \tikzEdge{A-}{D-};
           \tikzEdge{A-}{E-};
           \tikzEdge{B+}{A+};
           \tikzEdge{E+}{A+};
           \tikzEdgeLL{B+}{C-};
           \tikzEdgeLL{C+}{D-};
           \tikzEdgeR{B+}{E-};
         \end{tikzEdges}
    \end{tikzScopeDiagram}
    }
    \\
  
    \end{tabular}
    \end{center}
      
  \end{Wide}
  \caption{Catalog of Scoping Rules %\\ Newline here would be nice, but breaks the build
    (Arrows that follow from transitivity omitted)}
  \label{fig:rscope-catalog}
  \end{figure*}


\section{Proof of Hygiene}
\label{sec:rscope-hygiene}

We will show that our scope inference algorithm (when successful)
always produces surface scoping rules such that desugaring is
hygienic. Again, we say that a desugaring $\Desugar{}$ is hygienic when it
preserves $\alpha$-equivalence:
\[ \SaysScopeEqa[\SigmaSurf]{e}{e'} \text{\quad implies\quad }
   \SaysScopeEqa[\SigmaCore]{\Desugar{e}}{\Desugar{e'}}
\]
We will show this by way of a theorem that provides a necessary and
sufficient condition for hygiene,
assuming that desugaring obeys our assumptions.
Recall that our definition of $\alpha$-equivalence is
strong, including that both terms are well-bound; thus we will need
to show that the result of desugaring remains well-bound (so long as
its input is).

To discuss the properties required for this theorem,
we will divide variables into categories:
variables in $\Desugar{e}$ are either New (fresh) or Copied from $e$,
and variables in $e$ are either Used (if they were copied) or Unused
otherwise. Formally,
let $\phi$ be the mapping from \emph{copied} variables in $\Desugar{e}$ to
their sources in $e$, and:
\begin{Table}
  $\mathit{Used}$   &$\Defeq$& $\Range{\phi}$ \\
  $\mathit{Unused}$ &$\Defeq$& $\Op{vars}{e} - \Range{\phi}$ \\
  $\mathit{Copied}$ &$\Defeq$& $\Domain{\phi}$ \\
  $\mathit{New}$    &$\Defeq$& $\Op{vars}{\Desugar{e}} - \Domain{\phi}$
\end{Table}
(where $\Op{vars}{e}$ is the set of \emph{all} variables in $e$).

We now turn to the properties required for hygiene.
We will first list some clearly necessary properties, and then show
that they are also sufficient.

First, $\Desugar{}$ must avoid variable capture; thus $\Desugar{e}$ cannot contain
bindings between new variables (introduced by $\Desugar{}$) and copied
variables (taken from $e$):

\begin{property} \label{rscope-prop1}
  \WhitePhantom{.}
\begin{LongTable}
  $\forall{\RefnX \!\in\! \mathit{Copied}}.$
  && $\forall{\DeclX \!\in\! \mathit{New}}.$
  && $\SaysScopeNotBound{\Desugar{e}}{\RefnX}{\DeclX}$ \\

  $\forall{\RefnX \!\in\! \mathit{New}}.$
  && $\forall{\DeclX \!\in\! \mathit{Copied}}.$
  && $\SaysScopeNotBound{\Desugar{e}}{\RefnX}{\DeclX}$ \\
\end{LongTable}
\end{property}

Second, $\Desugar{}$ must preserve binding structure among the variables
it copies:\marginpar{%
\Cref{rscope-prop2} does not appear in prior work on hygiene because such
  work typically assumes that the binding structure of a surface term
  is \emph{defined by} the binding structure of its desugaring,
  causing this property to be true by definition.
}

\begin{property} \label{rscope-prop2}
  \WhitePhantom{.}
\begin{LongTable}
  $\forall{\RefnX,\DeclX \!\in\! \mathit{Copied}}.$ \\
     $\SaysScopeBound{\Desugar{e}}{\RefnX}{\DeclX}$ iff 
     $\SaysScopeBound{e}{\phi(\RefnX)}{\phi(\DeclX)}$
\end{LongTable}
\end{property}

Finally, $\Desugar{}$ must preserve well-boundedness. Thus, it must not cause a
reference to become unbound or introduce a new unbound reference:

\begin{property} \label{rscope-prop3}
  \WhitePhantom{.}
\begin{LongTable}
  $\forall{\RefnX \!\in\! \mathit{Used}}.$
  && $\forall{\DeclX \!\in\! \mathit{Unused}}.$
  && $\SaysScopeNotBound{e}{\RefnX}{\DeclX}$ \\

  $\forall{\RefnX \!\in\! \mathit{New}}.$
  && $\exists!{\DeclX \!\in\! \mathit{New}}.$
  && $\SaysScopeBound{\Desugar{e}}{\RefnX}{\DeclX}$ \\
\end{LongTable}
\end{property}

While these three properties are clearly necessary, it is by no means
clear that they are sufficient to guarantee $\alpha$-equivalence
preservation. However, the following theorem shows that they
are both necessary and sufficient to ensure that
$\Desugar{}$ preserves $\alpha$-equivalence:

\begin{theorem}[Fundamental Hygiene Theorem] \label{thm:rscope-hygiene}
Let $\Desugar{}$ be a desugaring function over terms with respect to
scoping rules $\Sigma$ that obeys the assumptions of \cref{sec:rscope-prelim}.
Then $\Desugar{}$ respects $\alpha$-equivalence iff
for every (well-bound) input term $e$ (numbering by property number):
\begin{LongTable}
  1. $\forall{\RefnX \!\in\! \mathit{Copied}}.$
  && $\forall{\DeclX \!\in\! \mathit{New}}.$
  && $\SaysScopeNotBound{\Desugar{e}}{\RefnX}{\DeclX}$ \\

  1. $\forall{\RefnX \!\in\! \mathit{New}}.$
  && $\forall{\DeclX \!\in\! \mathit{Copied}}.$
  && $\SaysScopeNotBound{\Desugar{e}}{\RefnX}{\DeclX}$ \\

  2. $\forall{\RefnX \!\in\! \mathit{Copied}}.$
  && $\forall{\DeclX \!\in\! \mathit{Copied}}.$
  && $\SaysScopeBound{\Desugar{e}}{\RefnX}{\DeclX}$ \\
  &&&& iff\; $\SaysScopeBound{e}{\phi(\RefnX)}{\phi(\DeclX)}$ \\

  3. $\forall{\RefnX \!\in\! \mathit{Used}}.$
  && $\forall{\DeclX \!\in\! \mathit{Unused}}.$
  && $\SaysScopeNotBound{e}{\RefnX}{\DeclX}$ \\

  3. $\forall{\RefnX \!\in\! \mathit{New}}.$
  && $\exists!{\DeclX \!\in\! \mathit{New}}.$
  && $\SaysScopeBound{\Desugar{e}}{\RefnX}{\DeclX}$
\end{LongTable}
where $\phi$ is the mapping from \emph{copied} variables in $\Desugar{e}$ to
their sources in $e$.
\end{theorem}

  \begin{proof}\label{proof:rscope-hygiene}
    $\Desugar{}$ respects $\alpha$-equivalence iff (i) $\Desugar{}$ maps well-bound terms
    to well-bound terms, and (ii) $\alpha$-renaming any declaration
    $\DeclX$ in any well-bound term $e$ does not change the binding
    structure of $\Desugar{e}$.
  
    Address parts (i) and (ii) in reverse order.
  
    For part (ii), do a case analysis on $\DeclX$. In both cases, let
    $e$ be the input term.
    \begin{enumerate}
    \item ($\DeclX \in \mathit{Unused}$) $\alpha$-varying $\DeclX$ renames
      both $\DeclX$ and also $\RefnX$ for every $\RefnX \Bound \DeclX$
      in $e$.
      Renaming $\DeclX$ itself is fine: since it doesn't appear in
      $\Desugar{e}$, it cannot change the binding structure of $\Desugar{e}$.
      Now do case analysis on every such $\RefnX$:
      \begin{enumerate}
        \item ($\RefnX \in \mathit{Unused}$) Similarly, this case is OK
          because $\RefnX$ does not appear in $\Desugar{e}$.
        \item ($\RefnX \in \mathit{Used}$) This case is problematic:
          renaming $\RefnX$ in $e$ will rename $\Image{\RefnX}$ in $\Desugar{e}$,
          which will cause its binding to change. Thus:
          \begin{equation}
            \Forall{\RefnX \in \mathit{Used}}
            \Forall{\DeclX \in \mathit{Unused}}
            \SaysScopeNotBound{e}{\RefnX}{\DeclX}
          \end{equation}
      \end{enumerate}
    \item ($\DeclX \in \mathit{Used}$) $\alpha$-varying $\DeclX$
      renames both $\DeclX$ and $\RefnX$ for every $\RefnX \Bound \DeclX$
      in $e$. Renaming $\DeclX$ is problematic iff
      $\Exists{\RefnXi}$ s.t. $\SaysScopeBound{\Desugar{e}}{\RefnXi}{\DeclXi}$
      where $\DeclXi \in \Image{\DeclX}$ but $\RefnXi \not\in
      \Image{\RefnX}$ for some $\RefnX \Bound \DeclX$
      (hence $\RefnXi$'s binding will change when $\DeclX$ is renamed).
      Thus:
      \begin{equation}
        \text{If } \SaysScopeBound{\Desugar{e}}{\RefnXi}{\DeclXi}
        \text{ and } \phi(\DeclXi) \text{ exists},
        \text{ then } \phi(\RefnXi) \text{ exists and }
        \phi(\RefnXi) \Bound \phi(\DeclXi)
      \end{equation}
  
      That dealt with the direct effects of renaming $\DeclX$.
      Now consider the effects of renaming $\RefnX$ for some
      $\SaysScopeBound{e}{\RefnX}{\DeclX}$. Do case analysis on every such $\RefnX$:
      \begin{enumerate}
        \item ($\RefnX \in \mathit{Unused}$) Renaming $\RefnX$ in $e$ is
          OK since it does not appear in $\Desugar{e}$.
        \item ($\RefnX \in \mathit{Used}$) Consider some
          $\RefnXi \in \Image{\RefnX}$.
          Let $\RefnXi \Bound \DeclXi \in \Desugar{e}$ (it must be bound, since
          $\Desugar{e}$ is well-bound).
          For the binding of $\RefnXi$ to not change when $\DeclX$ (and $\RefnX$) is
          renamed, $\DeclXi$ must be in $\Image{\DeclX}$. Thus:
          \begin{equation}
            \text{If } \SaysScopeBound{\Desugar{e}}{\RefnXi}{\DeclXi}
            \text{ and } \phi(\RefnXi) \text{ exists }, \text{ then } \phi(\DeclXi) \text{ exists, and }
            \phi(\RefnXi) \Bound \phi(\DeclXi)
          \end{equation}
      \end{enumerate}
    \end{enumerate}
    
    For part (i), $\Desugar{e}$ must be well-bound. The only bindings
    not covered by the above cases are bindings from
    $\mathit{New}$ to $\mathit{New}$. Thus:
    \begin{equation}
      \Forall{\RefnX \in \mathit{New}}
      \ExistsUnique{\DeclX \in \mathit{New}}
      \RefnX \Bound \DeclX \in \Desugar{e}
    \end{equation}
  
    This gives four equations that hold iff $\Desugar{}$ is hygienic.
    Equation (1) corresponds to the first part of the theorem's Property
    (3). Equations (2) and (3) together correspond to Properties (1) and
    (2). Finally, equation (4) corresponds to the second part of
    Property (3).
  \end{proof}

We can now see that $\Op{inferScope}{}$ is hygienic. Property 1 is
easily ensured by giving variables fresh names, which is one of our
assumptions about desugaring. Property 2 follows from our inference
process: since desugaring preserves scope by \cref{thm:rscope-resugar},
bindings between variables must not change. Finally, property 3 is
exactly what $\Op{checkScope}{}$ checks.

%% NOTE: The lemma given here holds, but it's not the most useful
%% thing to show, and it's involved, so I'm leaving it out of the
%% thesis. (The proof is a bit out of date, and referencing the wrong
%% places, but I believe it is otherwise sound.)

%% \section{Hygiene for Evaluation Resugaring}
%% \label{sec:rscope-eval-hygiene}

%% We can also use this hygiene theorem to show how to make the
%% resugaring system from the previous chapter (\Resugarer) hygienic,
%% both during expansion and during unexpansion. Two things must be
%% done. First, when {\Resugarer} desugars, it must (unsurprisingly) give
%% newly introduced variables fresh names, and when resugaring it should
%% account for these fresh names when matching. Second, this chapter's scope
%% inference must be applied. This, unfortunately, disallows sugars that
%% contain ellipses, although we are currently working to lift this
%% restriction. Remember that scope inference serves a dual % TODO!
%% purpose: it not only infers scope rules, it also rejects bad sugars
%% for which no valid scope rules exist.

%% While it may seem strange to use a scope inference process simply to
%% make a system hygienic, keep in mind that we use a very strong
%% definition of hygiene in this chapter. This definition refers to
%% properties such as ``desugaring may not cause a variable to become
%% unbound'' that are not even defined without the existence of surface
%% scope rules. Presumably, a weaker version of hygiene could be
%% achieved more easily (although it is not quite as simple as using an
%% off-the-shelf hygienic expansion system, as it isn't immediately clear
%% how that should interact with unexpansion).

%% With that said, we can state and prove the lemma:
%% \begin{lemma}
%%   Let $\Rs$ be a set of desugaring rules, and let $\SigmaCore$ be
%%   scope rules for its core langauge. Suppose that:
%%   \begin{itemize}[noitemsep]
%%   \item $\ExpandRecf$ gives fresh names to introduced variables (that
%%     is, variables on the \Sc{rhs}).
%%   \item $\UnexpandRecf$ allows any variable name to match against an
%%     introduced variable.
%%   \item No rule in $\Rs$ drops a pattern variable.
%%   \item $\SigmaSurf = \Op{inferScope}{\SigmaCore,\,\Rs}$.
%%   \end{itemize}
%%   Then both $\ExpandRecf$ and $\UnexpandRecf$ preserve
%%   $\alpha$-equivalence.
%% \end{lemma}
%% \begin{proof}[Proof sketch]
%%   We will apply the hygiene theorem in both directions: both forward
%%   (during expansion) and in reverse (during unexpansion). There are
%%   two sets of requirements for the theorem: the assumptions about
%%   resugaring in \cref{sec:rscope-prelim}, and properties
%%   (1)--(3) in this section. We must show that both sets of
%%   requirements hold both forward and in reverse:
%%   \subsubsection{Assumptions about desugaring (\cref{sec:rscope-prelim})}
%%   The three assumptions in \cref{sec:formal-reqs} are also assumed for
%%   evaluation resugaring (in \cref{sec:reval-wf}), so the
%%   \emph{forward} versions of them hold. Now consider the reverse versions:
%%   \begin{itemize}[noitemsep]
%%   \item[] \emph{Assumption (1, reverse)} is a basic well-formedness
%%     requirement (a pattern variable on the \Sc{rhs} but not the
%%     \Sc{lhs} does not make sense).
%%   \item[] \emph{Assumption (2, reverse)} is identical to Assumption (2, forward).
%%   \item[] \emph{Assumption (3, reverse)} does not hold: rules'
%%     \Sc{rhs}s often \emph{do} contain variables. Thus we must argue
%%     that unexpanding a rule whose \Sc{rhs} contains variables does not
%%     modify $\alpha$-equivalence. Since the variables in the term being
%%     unexpanded were freshly named during desugaring, no other
%%     variables in the term could possibly be bound by or bound to
%%     them. Thus removing them (all at once) preserves $\alpha$-equivalence.
%%   \item[] \emph{Assumption (4, reverse)} is vacuously true, since the
%%     \Sc{lhs} is assumed to not contain variables (\cref{sec:reval-wf}).
%%   \end{itemize}
%%   \subsubsection{Properties (1)--(3)}
%%   \begin{itemize}[noitemsep]
%%   \item[] \emph{Property (1, forward)} holds because we assume
%%     variables are given fresh names.
%%   \item[] \emph{Property (1, reverse)} is vacuously true because we
%%     assume that pattern variables are not dropped, so
%%     $\mathit{Unused}$ is empty.
%%   \item[] \emph{Property (2, forward)} follows from our scope
%%     inference process: since desugaring preserves scope by
%%     \cref{thm:rscope-resugar}, bindings between variables must not
%%     change.
%%   \item[] \emph{Property (2, reverse)} also follows from our scope
%%     inference process, which shows that desugaring rules preserve
%%     scope. Preserving scope is a symmetric property, and will hold
%%     just as well if you run the rule in reverse.
%%   \item[] \emph{Property (3, forward)} is checked by the
%%     \Op{checkScope}{} subroutine of the \Op{inferScope}{} function; if the
%%     check failed, then \Op{inferScope}{} would have failed.
%%   \item[] \emph{Property (3, reverse)}: consider the two equations of
%%     this property in turn. The first is equivalent to the first
%%     equation in Property (1, forward), which we already showed. The
%%     second is vacuously true (since $\mathit{Unused}$ is empty), as
%%     well as following from the surface term being well-bound.
%%   \end{itemize}
%%   Thus the hygiene theorem applies in both directions, so both
%%   $\ExpandRecf$ and $\UnexpandRecf$ are hygienic.
%% \end{proof}



%%%%%%%%%%%%%%%%%%%%%%%%%% Future work %%%%%%%%%%%%%%%%%%%%%%%%%%%%%%%

%% \section{Future Work} \label{sec:future}

%% We have shown how scope can be resugared in a principled way, modulo
%% certain assumptions. However, some languages violate these
%% assumptions. Here, we discuss three important directions for bridging
%% the gap to these languages.

%% \subsection{Breaking Hygiene} \label{sec:future-hygiene}
%% Certain language constructs break hygiene by definition. Most
%% commonly, a construct may \emph{implicitly} declare a variable. For
%% instance, \Code{this} may be implicitly bound in the body of a
%% method. We believe that allowing selective hygiene violations
%% like this should be a mostly straightforward extension of the theory.
%% As a roadmap:
%% \begin{itemize}
%% \item The definition of {\sap} should change, to allow constant
%%   variable names to be part of the partial order. For instance,
%%   \Code{method} may have $(\Code{this}, 2) \in \Sigma[\Code{method}]$,
%%   assuming that $2$ names the method's body.
%% \item The declarative and algorithmic scope checking rules will need
%%   to be extended with rules for constant names; this will require a
%%   large expansion of the proof of \cref{thm:rules}.
%% \item We believe that scope resugaring will be largely unmodified:
%%   instead of lifting partial orders over holes, it will lift partial orders over
%%   holes plus constant names (desugaring rules will need to declare
%%   whether variables they introduce are meant to be capturing or
%%   non-capturing, as already happens in some macro systems~\cite{syntax-case}).
%% \end{itemize}

%% \subsection{Extending Sugar Rules}
%% In this paper, we assume that desugaring rules are pattern-based and
%% name-agnostic (\cref{def:name-agnostic}).
%% While the name-agnosticism assumption is part and
%% parcel of hygiene, the pattern restriction can potentially be
%% lifted. A particularly attractive direction is to develop a
%% binding-safe meta programming language based on {\sap}. The primary
%% difficulty is that the embedded language's scope partial order must be
%% expressed in the host language's type system (for instance,
%% a pair $(a, b)$ where $a$ and $b$ are embedded-language terms
%% and $b$ is in scope of $a$ must have a \emph{different type} than a
%% pair of terms $(a, b)$ where neither $a$ nor $b$ are in scope of
%% the other).

%% We have developed an exploratory prototype of such a system in
%% Haskell, encoding scope partial orders in Haskell's type
%% system. It aims to ensure that functions on terms in the embedded
%% language preserve binding (i.e., given a term as input containing a
%% bound reference $x$, no function can produce an output where that same
%% variable $x$ is bound by a different declaration). We will submit it
%% for artifact evaluation.

%% \subsection{Extending the Binding Language}
%% While our binding language compares favorably to others
%% (\refsec{sec:related}), there are still binding constructs
%% it cannot describe.
%% \begin{description}
%% \item[Modules] ``Modules'', as described in Neron et al.s'
%%   scope graphs~\cite{neron-scope},
%%   cover not only importing and exporting modules,
%%   but also other uses of ``explicit name spaces'' such as OO classes.
%%   While we have found a small extension of our binding language to
%%   handle modules, it relies on breaking the transitivity of {\sap},
%%   which is an essential axiom used, e.g., in \cref{thm:rules}.
%%   Therefore we do not yet know how namespacing constructs can be
%%   resugared.

%% \item[Visibility Annotations] It is sometimes useful to declare how
%%   visible a declaration should be, e.g., private to a module or public
%%   to other modules. This seems closely linked to the above
%%   ``modules'', since the visibility is usually relative to some
%%   namespace boundary (such as the body of the module it is declared
%%   in).
%% \item[Records] Another useful extension would be handling records and
%%   dot-lookup. Doing so clearly requires a type system for the surface
%%   language, which is far beyond the scope of this work. Ideally, the
%%   type system would be acquired by resugaring the core language type
%%   system. Such a ``type resugaring'' would be very valuable in other
%%   settings too.
%% \end{description}


\section{Related Work}
\label{sec:rscope-related}


Below, we discuss work related to two aspects of our approach to
scope inference. However, none of them infer scope through syntactic
sugar; we therefore believe that the central contribution of this
chapter is novel.

\subsection{Hygienic Expansion}

The real goal of hygienic expansion is to preserve
$\alpha$-equivalence: $\alpha$-renaming a program should not change
its meaning. Typically, however, $\alpha$-equivalence is only
\emph{defined} for the core language. Thus, traditional approaches to
hygiene have had to focus on avoiding specific issues like
variable capture~\cite{hygienic-macros}.
Recent work by Adams advances the theory by
giving an algorithm-independent set of issues to
avoid~\cite{adams-hygiene}. However, even this work lacks the ground truth of
$\alpha$-equivalence preservation to base its claims on.

In contrast, Herman et al.
advocate that sugar specify the binding structure of the constructs
they introduce, and build a system that does so~\cite{herman-hygiene}.
Stansifer et al. follow with a more powerful system
called Romeo based on the same approach~\cite{stansifer-romeo}.
(We will discuss the binding languages used by these two tools in the
next subsection.)
Since we \emph{infer} scope rules for the surface language, we can
verify that desugaring preserves $\alpha$-equivalence without
requiring scope annotations on sugars.

Erdweg et al. put forward an interesting alternative approach to hygiene
with the \emph{name-fix} algorithm~\cite{erdweg-hygiene}.
\emph{Name-fix} assumes that the scoping for the surface language is
known. Instead of using this information to \emph{avoid} unwanted
variable capture in the first place, \emph{name-fix} uses it
to \emph{detect} variable capture and rename variables as necessary to
repair it after the fact. Erdweg et al.\ prove that \emph{name-fix}
preserves $\alpha$-equivalence, but for a \emph{weaker} definition
of $\alpha$-equivalence than ours that doesn't include
well-boundedness (thus allowing desugaring to produce unbound variables).

This chapter's work differs from the above work: we assume that scope is defined
\emph{only} for the core language, and not for the
surface language (\`a la Erdweg) or for individual rewrite rules (\`a la Herman).
This assumption is also made by traditional capture-avoiding work on
hygiene. However, by inferring scoping rules from the core to
the surface language, we gain two benefits: (i) we can prove that our
approach is correct with respect to the \emph{ground truth} of
$\alpha$-equivalence preservation (\cref{thm:rscope-hygiene}), and (ii)
we can produce a set of standalone scoping rules for the surface
language. To our knowledge, this approach has not been taken before.


\subsection{Scope}

We will divide related work on scoping into two main categories.
First, ``Modeling Scope'' discusses ways in which the scope of a term
can be \emph{represented}. Our description of scope as a
preorder (\cref{sec:rscope-sap}) falls in this category.
Second, ``Binding Specification Languages'' discusses ways in which scope
can be \emph{determined} for a given term. Our binding
language (\cref{sec:rscope-rules}) falls in this category.

\paragraph{Modeling Scope}

Our description of {\sap} is similar to the view
expressed by Flatt in ``Binding as Sets of Scopes''~\cite{flatt:scope}.\marginpar{%
{\Sap} and Flatt's Sets of Scopes were discovered
  independently: {\sap} arose from some
  of the ideas from Romeo~\cite{stansifer-romeo}.
}
In fact, Flatt's notion of scope can be expressed as a preorder,
as we show in \cref{sec:rscope-sos}.

Neron et al. describe \emph{scope graphs}, which are also based on a
similar view, but have a more complicated set of definitions~\cite{neron-scope}.
Unlike {\sap}, however, scope graphs include mechanisms for handling
module scope, which gives it the ability to model both modules and
also other constructs like objects and field lookup.
Our {\sap} binding language can actually be extended to
handle modules, but doing so breaks our transitivity
assumption, which we need to infer scope, so we have left it out of
this chapter and consider this a problem for
future work.

\paragraph{Binding Specification Languages}

Our preorder-based binding specification language is novel, but
similar in expressiveness to many others.  It is perhaps most similar
to Stansifer et al.'s Romeo~\cite{stansifer-romeo}.
The primary difference between the two is that
Romeo has slightly more expressive power: given two declarations
$\DeclX[1]$ and $\DeclX[2]$, it is possible in Romeo for $\DeclX[1]$
to shadow $\DeclX[2]$ in one part of a term, but $\DeclX[2]$ to shadow
$\DeclX[1]$ in a different part of a term.\footnote{
  In Romeo, this would be expressed using the $\rhd$ combinator,
  as $\beta_1 \rhd \beta_2$ and $\beta_2 \rhd \beta_1$.
} It is not clear
if this power has any practical applications, but we choose to avoid
it both for aesthetic reasons (we do not believe two declarations
should be allowed to shadow one another), and to simplify scope
inference (which would otherwise have to manipulate formulas over
Romeo's combinators, instead of merely preorders).

In a similar vein, Sewell et al. present a semantics engineering
workbench called Ott, which includes a comparable binding
specification language~\cite{sewell-ott-jfp}. Like Romeo, Ott would allow two declarations
to each shadow one another in different places. Furthermore, it gives
additional power, by allowing terms to \emph{name} what they provide.
For instance, a term could export \emph{two} binding lists, one named
``value-bindings'' and one named ``type-bindings''.

Weirich et al. present a binding specification language called
Unbound, which can be expressed using {\sap} (and
hence is no more expressive than it)~\cite{weirich-scope}.
They implement Unbound in Haskell, and give language-agnostic
implementations of operations such as constructing and deconstructing
terms, determining $\alpha$-equivalence, and performing substitution.
In Unbound, binding is specified via a set of \emph{binding combinators}.
  These binding combinators can be expressed as a preorder.\footnote{
  The translation of Unbound to {\sap} is as follows. \Code{Name}
  constructs a declaration or reference, depending on whether it is a
  term or Pattern. Patterns $P$ have scoping rules that state
  $\SigPExpt{i}$ and $\SigPImpt{i}$ for every $i$. terms $e$ have
  scoping rules that state $\SigImpt{e}{i}$ for every $i$. Finally,
  each of the four binding combinators obey the scoping rule for
  patterns or for terms, as appropriate, in addition to the following
  facts:
  \begin{center}
  \begin{tabular}{l l @{\quad} l l}
    Bind $P$ $e$   & $\{\SigBind{\text{Bind}}{2}{1}\}$ &
    Embed $e$      & $\emptyset$ \\
    Rebind $P$ $P$ & $\{\SigBind{\text{Rebind}}{2}{1}\}$ &
    Rec $P$        & $\{\SigBind{\text{Rec}}{1}{1}\}$
  \end{tabular}
  \end{center}
}

There are many more binding specification
languages~\cite{aczel-church-rosser,pottier-caml,nabl}.
We have chosen what we believe to be a representative sample for
comparison. We have shown that our binding specification language
compares favorably in expressiveness, while simultaneously being
simple enough to enable scope inference.

\section{Discussion and Conclusion}
We have presented what we believe is the first algorithm for
inferring scoping rules through syntactic sugar. It makes use
of our description of scope as a preorder in
\cref{sec:rscope-sap}, and our binding language for specifying the
scope of a programming language in \cref{sec:rscope-rules}. The case studies
in \cref{sec:rscope-impl} show that all of the aspects of this work
are able to deal with many interesting scoping constructs from real
languages.

We see three clear directions in which to try to extend scope
inference. First, support for ellipses in sugar definitions would make
writing sugars easier. Second, allowing named imports and exports---\`a
la Ott~\cite{sewell-ott-jfp}---would make sugars like \texttt{do}
inferable.  Third, modules---\`a la Scope Graphs~\cite{neron-scope}---are
necessary for inferring scope for modules and for classes. These last
two changes are relatively straightforward extensions to our binding
language, but research questions when applied to scope inference.

\newpage

\begin{figure*}[h!]
  \begin{sideways}
  \begin{minipage}{18cm}
    \small
    \begin{center}\textbf{SA $\to$ SD}\end{center}
    \begin{Table}&&
      \inference[SA-Import]
        {\SaysScopeS{e_i}{a}{\im e_i} \quad
         \SigPImpt{i}}
        {\SaysScopeS{\NodeStd}{a}{\im{}\NodeStd}}
    \\ \\ &$\Longrightarrow$&
    \inference[SD-Trans]
      {\inference[S-Lift]
        {\SaysScopeS{e_i}{a}{\im e_i}}
        {\SaysScopeS{\NodeStd}{a}{\im e_i}}
      \quad
      \inference[SD-Import]
        {\SigPImpt{i}}
        {\SaysScopeS{\NodeStd}{\im e_i}{\im \NodeStd}}}
      {\SaysScopeS{\NodeStd}{a}{\im \NodeStd}}
    \\ \\ &&
      \inference[SA-Export]
        {\SigPExpt{i} \quad
         \SaysScopeS{e_i}{\ex e_i}{a}}
        {\SaysScopeS{\NodeStd}{\ex{}\NodeStd}{a}}
    \\ \\ &$\Longrightarrow$&
    \inference[SD-Trans]
      {\inference[SD-Export]
        {\SigPExpt{i}}
        {\SaysScopeS{\NodeStd}{\ex \NodeStd}{\ex e_i}}
      \quad
      \inference[S-Lift]
        {\SaysScopeS{e_i}{\ex e_i}{a}}
        {\SaysScopeS{\NodeStd}{\ex e_i}{a}}\phantom{}}
      {\SaysScopeS{\NodeStd}{\ex \NodeStd}{a}}
    \\ \\ &&
    \inference[SA-Bind]
        {\SaysScopeS{e_i}{a}{\im e_i} \quad
         \SigPBind{i}{j} \quad
         \SaysScopeS{e_j}{\ex e_j}{b}}
        {\SaysScopeS{\NodeStd}{a}{b}}
    \\ \\ &$\Longrightarrow$&
    \inference[SD-Trans$^2$]
      {\hspace{-3em}
        \inference[S-Lift]
        {\SaysScopeS{e_i}{a}{\im e_i}}
        {\SaysScopeS{\NodeStd}{a}{\im e_i}}
      \quad
      \inference[SD-Bind]
        {\SigPBind{i}{j}}
        {\SaysScopeS{\NodeStd}{\im e_i}{\ex e_j}}
      \quad
      \inference[S-Lift]
        {\SaysScopeS{e_j}{\ex e_j}{b}}
        {\SaysScopeS{\NodeStd}{\ex e_j}{b}}\phantom{}}
      {\SaysScopeS{\NodeStd}{a}{b}}
      \\ \\
    \end{Table}

  \end{minipage}
  \end{sideways}
  \caption{Proof of \cref{thm:rscope-rules}}
  \label{fig:rscope-rules-proof-1}
\end{figure*}

\begin{figure*}[h!]
  \begin{sideways}
  \begin{minipage}{18cm}
    \small
    
    \begin{center}\textbf{SD $\to$ SA}\end{center}
  
    \begin{Table}&&
    \inference[SD-Export]{
      \SigPExpt{i}
    }{
      \SaysScopeS{\NodeStd}{\ex \NodeStd}{\ex e_i}
    }
    \quad$\Longrightarrow$\quad
    \inference[SA-Export]{
      \SigPExpt{i} \quad
      \inference[S-Refl2]{
      }{
        \SaysScopeS{e_i}{\ex e_i}{\ex e_i}
      }
    }{
      \SaysScopeS{\NodeStd}{\ex \NodeStd}{\ex e_i}
    }
    \\ \\ &&
    \inference[SD-Import]{
      \SigPImpt{i}
    }{
      \SaysScopeS{\NodeStd}{\im e_i}{\im \NodeStd}
    }
    \quad$\Longrightarrow$\quad
    \inference[SA-Import]{
      \SigPImpt{i}
      \quad
      \inference[S-Refl1]{
      }{
        \SaysScopeS{e_i}{\im e_i}{\im e_i}
      }
    }{
      \SaysScopeS{\NodeStd}{\im e_i}{\im \NodeStd}
    }
    \\ \\ &&
    \inference[SD-Trans]{
      \inference[S-Refl1]{}{\SaysScopeS{e}{\im e}{\im e}}
      \quad
      \inference{D}{\SaysScopeS{e}{\im e}{a}}\phantom{}
    }{
      \SaysScopeS{e}{\im e}{a}
    }
    \quad$\Longrightarrow$\quad
    \inference{D}{\SaysScopeS{e}{\im e}{a}}
    \qquad \text{(likewise for S-Refl2)}
    \\ \\ &&
    \inference[SD-Bind]{
      \SigPBind{i}{j}
    }{
      \SaysScopeS{\NodeStd}{\im e_i}{\ex e_j}
    }
    \quad$\Longrightarrow$\quad
    \inference[SA-Bind]{
      \inference[S-Refl1]{}{\SaysScopeS{e_i}{\im e_i}{\im e_i}}
      \quad
      \SigPBind{i}{j}
      \quad\phantom{}
      \inference[S-Refl2]{}{\SaysScopeS{e_j}{\ex e_j}{\ex e_j}}\phantom{}
    }{
      \SaysScopeS{\NodeStd}{\im e_i}{\ex e_j}
    }
    \\ \\ &&
    \inference[SD-Trans]{
      \hspace{-4em}
      \inference[SA-Bind]{
        \SaysScopeS{e_i}{a}{\im e_i}
        \quad
        \SigPBind{i}{j}
        \quad
        \SaysScopeS{e_j}{\ex e_j}{b}
      }{
        \SaysScopeS{\NodeStd}{a}{b}
      }
      \quad
      \inference[SA-Bind]{
        \SaysScopeS{e_j}{b}{\im e_j}
        \quad
        \SigPBind{j}{k}
        \quad
        \SaysScopeS{e_k}{\ex e_k}{c}
      }{
        \SaysScopeS{\NodeStd}{b}{c}
      }\phantom{}
    }{
      \SaysScopeS{\NodeStd}{a}{c}
    }
    \\ \\ &$\Longrightarrow$&
    \inference[SA-Bind]{
      \SaysScopeS{e_i}{a}{\im e_i}
      \quad
      \inference{
        \SigPBind{i}{j}
        \quad
        \SigPBind{j}{k}
      }{
        \SigPBind{i}{k}
      }
      \quad
      \SaysScopeS{e_k}{\ex e_k}{c}
    }{
      \SaysScopeS{\NodeStd}{a}{c}
    }
    \\ \\ &&
    \inference[SD-Trans]{
      \inference[SA-Bind]{
        \SaysScopeS{e_i}{a}{\im e_i}
        \quad
        \SigPBind{i}{j}
        \quad
        \SaysScopeS{e_j}{\ex e_j}{b}
      }{
        \SaysScopeS{\NodeStd}{a}{b}
      }
      \quad
      \inference[S-Lift]{
        \SaysScopeS{e_j}{b}{c}
      }{
        \SaysScopeS{\NodeStd}{b}{c}
      }\phantom{}
    }{
      \SaysScopeS{\NodeStd}{a}{c}
    }
    \\ \\ &$\Longrightarrow$&
    \inference[SA-Bind]{
      \SaysScopeS{e_i}{a}{\im e_i}
      \quad
      \SigPBind{i}{j}
      \quad
      \inference[SD-Trans]{
        \SaysScopeS{e_j}{\ex e_j}{b}
        \quad
        \SaysScopeS{e_j}{b}{c}
      }{
        \SaysScopeS{e_j}{\ex e_j}{c}
      }
    }{
      \SaysScopeS{\NodeStd}{a}{c}
    }
    \end{Table}
    
  \end{minipage}
  \end{sideways}
  \caption{Proof of \cref{thm:rscope-rules} (continued)}
  \label{fig:rscope-rules-proof-2}
\end{figure*}
  
  
\begin{figure*}[h!]
  \begin{sideways}
  \begin{minipage}{18cm}
    \small
  
    \begin{center}\textbf{SD $\to$ SA (cont.)}\end{center}

    \begin{Table}&&
    \inference[SD-Trans]{
      \inference[S-Lift]{
        \SaysScopeS{e_i}{a}{b}
      }{
        \SaysScopeS{\NodeStd}{a}{b}
      }
      \quad
      \inference[SA-Bind]{
        \SaysScopeS{e_i}{b}{\im e_i}
        \quad
        \SigPBind{i}{j}
        \quad
        \SaysScopeS{e_j}{\ex e_j}{c}
      }{
        \SaysScopeS{\NodeStd}{b}{c}
      }\phantom{}
    }{
     \SaysScopeS{\NodeStd}{a}{c}
    }
    \\ \\ &$\Longrightarrow$&
    \inference[SD-Bind]{
      \inference[SD-Trans]{
        \SaysScopeS{e_i}{a}{b}
        \quad
        \SaysScopeS{e_i}{b}{\im e_i}
      }{
        \SaysScopeS{e_i}{a}{\im e_i}
      }
      \quad
      \SigPBind{i}{j}
      \quad
      \SaysScopeS{e_j}{\ex e_j}{c}
    }{
      \SaysScopeS{\NodeStd}{a}{c}
    }
    \\ \\ &&
    \inference[SD-Trans]{
      \inference[S-Lift]{
        \SaysScopeS{e_i}{a}{b}
      }{
        \SaysScopeS{\NodeStd}{a}{b}
      }
      \quad
      \inference[S-Lift]{
        \SaysScopeS{e_i}{b}{c}
      }{
        \SaysScopeS{\NodeStd}{b}{c}
      }\phantom{}
    }{
      \SaysScopeS{\NodeStd}{a}{c}
    }
    \\ \\ &$\Longrightarrow$&
    \inference[S-Lift]{
      \inference[SD-Trans]{
        \SaysScopeS{e_i}{a}{b}
        \quad
        \SaysScopeS{e_i}{b}{c}
      }{
        \SaysScopeS{e_i}{a}{c}
      }
    }{
      \SaysScopeS{\NodeStd}{a}{c}
    }
    \\ \\ &&
    \inference[SD-Trans]{
      \inference[SA-Export]{
        \SigPExpt{i}
        \quad
        \SaysScopeS{e_i}{\ex e_i}{a}
      }{
        \SaysScopeS{\NodeStd}{\ex \NodeStd}{a}
      }
      \quad
      \inference[SA-Import]{
        \SaysScopeS{e_i}{a}{\im e_i}
        \quad
        \SigPImpt{i}
      }{
        \SaysScopeS{\NodeStd}{a}{\im \NodeStd}
      }\phantom{}
    }{
      \SaysScopeS{\NodeStd}{\ex \NodeStd}{\im \NodeStd}
    }
    \\ \\ &$\Longrightarrow$&
    \inference[S-ReExport]{
      \inference{
        \SigPExpt{i}
        \quad
        \SigPImpt{i}
      }{
        \SigPSelf
      }
    }{
      \SaysScopeS{\NodeStd}{\ex \NodeStd}{\im \NodeStd}
    }
    \\ \\ &&
    \inference[SD-Trans]{
      \inference[SA-Export]{
        \SigPExpt{i}
        \quad
        \SaysScopeS{e_i}{\ex e_i}{a}
      }{
        \SaysScopeS{\NodeStd}{\ex \NodeStd}{a}
      }
      \quad
      \inference[S-Lift]{
        \SaysScopeS{e_i}{a}{b}
      }{
        \SaysScopeS{\NodeStd}{a}{b}
      }\phantom{}
    }{
      \SaysScopeS{\NodeStd}{\ex \NodeStd}{b}
    }
    \\ \\ &$\Longrightarrow$&
    \inference[SA-Export]{
      \SigPExpt{i}
      \quad
      \inference[SD-Trans]{
        \SaysScopeS{e_i}{\ex e_i}{a}
        \quad
        \SaysScopeS{e_i}{a}{b}
      }{
        \SaysScopeS{e_i}{\ex e_i}{b}
      }
    }{
      \SaysScopeS{\NodeStd}{\ex \NodeStd}{b}
    }
    \end{Table}
  \end{minipage}
  \end{sideways}
  \caption{Proof of \cref{thm:rscope-rules} (continued)}
  \label{fig:rscope-rules-proof-3}
\end{figure*}

\begin{figure*}[h!]
  \begin{sideways}
  \begin{minipage}{18cm}
    \small
  
    \begin{center}\textbf{SD $\to$ SA (cont.)}\end{center}
    
    \begin{Table}
    \\ \\ &&
    \inference[SD-Trans]{
      \inference[S-Lift]{
        \SaysScopeS{e_i}{a}{b}
      }{
        \SaysScopeS{\NodeStd}{a}{b}
      }
      \quad
      \inference[SA-Import]{
        \SaysScopeS{e_i}{b}{\im e_i}
        \quad
        \SigPImpt{i}
      }{
        \SaysScopeS{\NodeStd}{b}{\im \NodeStd}
      }\phantom{}
    }{
      \SaysScopeS{\NodeStd}{a}{\im \NodeStd}
    }
    \\ \\ &$\Longrightarrow$&
    \inference[SA-Import]{
      \inference[SD-Trans]{
        \SaysScopeS{e_i}{a}{b}
        \quad
        \SaysScopeS{e_i}{b}{\im e_i}
      }{
        \SaysScopeS{e_i}{a}{\im e_i}
      }
      \quad
      \SigPImpt{i}
    }{
      \SaysScopeS{\NodeStd}{a}{\im \NodeStd}
    }
    \\ \\ &&
    \inference[SD-Trans]{
      \inference[SA-Bind]{
        \SaysScopeS{e_i}{a}{\im e_i}
        \quad
        \SigPBind{i}{j}
        \quad
        \SaysScopeS{e_j}{\ex e_j}{b}
      }{
        \SaysScopeS{\NodeStd}{a}{b}
      }
      \quad
      \inference[SA-Import]{
        \SaysScopeS{e_j}{b}{\im e_j}
        \quad
        \SigPImpt{j}
      }{
        \SaysScopeS{\NodeStd}{b}{\im \NodeStd}
      }\phantom{}
    }{
      \SaysScopeS{\NodeStd}{a}{\im \NodeStd}
    }
    \\ \\ &$\Longrightarrow$&
    \inference[SA-Import]{
      \SaysScopeS{e_i}{a}{\im e_i}
      \quad
      \inference{
        \SigPBind{i}{j}
        \quad
        \SigPImpt{j}
      }{
        \SigPImpt{i}
      }
    }{
      \SaysScopeS{\NodeStd}{a}{\im \NodeStd}
    }
    \\ \\ &&
    \inference[SD-Trans]{
      \inference[SA-Export]{
        \SigPExpt{i}
        \quad
        \SaysScopeS{e_i}{\ex e_i}{a}
      }{
        \SaysScopeS{\NodeStd}{\ex \NodeStd}{a}
      }
      \quad
      \inference[SA-Bind]{
        \SaysScopeS{e_i}{a}{\im e_i}
        \quad
        \SigPBind{i}{j}
        \quad
        \SaysScopeS{e_j}{\ex e_j}{b}
      }{
        \SaysScopeS{\NodeStd}{a}{b}
      }\phantom{}
    }{
      \SaysScopeS{\NodeStd}{\ex \NodeStd}{b}
    }
    \\ \\ &$\Longrightarrow$&
    \inference[SA-Export]{
      \inference{
        \SigPExpt{i}
        \quad
        \SigPBind{i}{j}
      }{
        \SigPExpt{j}
      }
      \quad
      \SaysScopeS{e_j}{\ex e_j}{b}
    }{
      \SaysScopeS{\NodeStd}{\ex \NodeStd}{b}
    }
    \end{Table}
  
  \end{minipage}
  \end{sideways}
  \caption{Proof of \cref{thm:rscope-rules} (continued)}
  \label{fig:rscope-rules-proof-4}
  \end{figure*}
